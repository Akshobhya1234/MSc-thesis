\chapter{Universal Algebra: An Overview}
By the early 18th century, mathematicians had discovered how to solve polynomial
equation of up-to degree 4. There was no general rule to solve a polynomial
equation of any degree. It wasn't until, mathematician Evariste Galois
discovered a tool called \emph{group} to solve polynomial equation of any
degree. This tool was later discovered to be useful in other fields of
mathematics such as modulus theory and geometry\cite{enwiki:1107380309}. Knowing
the usefulness of this tool, mathematicians abstracted out the axioms of the
group into a general tool. Thus evolved the structure \emph{group} that we know
today. As group theory, the study of \emph{group} structures evolved, other
abstract structures were invented to solve problems. This gave rise to a new
field in mathematics called \emph{abstract algebra}. Abstract algebra is the
study of algebraic structure and its models or examples
\cite{enwiki:1107380309}. An algebraic structure is a tuple containing a
'carrier' set, A, a set of operations that act on A, and a set of axioms
involving the operations and A. Some mathematicians were only interested in
studying the structures themselves that is the arbitrary interpretation of the
language and not the models that includes a theory that holds in the structure.
Universal algebra is the study of algebraic structures and its properties. In
the recent years, \emph{universal algebra} has seen an exponential growth in its
study of theories and applications \cite{sankappanavar1981course}. 

Algebraic structures, like monoids, loops, groups and rings have similar
properties. Universal algebra studies these structures by abstracting out the
specific definitions and properties of algebraic structures. Universal algebra
will deal with these algebraic structures as axiomatic theories in equational
first-order logic \cite{YSharoda}.

\section{Relation and function}
In order to understand algebraic structures, it is essential to know some basics
of relations and functions. In this section, we define relations and functions.
We can start with defining a set.
\begin{itemize}
\item 
A set is a well-defined collection of objects. The elements or members of the
set can be mathematical object of any kind such as numbers, symbols, geometrical
shapes, or even other sets. If $x$ is an object in set $S$ then we say x is an
element of S and is denoted as $x\inS$.

\item The \emph{Cartesian product} between two sets $X$ and $Y$,  $X \ \times\
Y$ is defined as a pair ${(x,y) :\ x \ \in\ X$,\ $y\ \in\ Y }$. 

\item
A \emph{binary relation} is a subset of the Cartesian product of two sets that
is a mapping between one set called \textit{domain} to the other set called the
\textit{codomain}. A relation assigns elements of domain to some elements in
the codomain. A binary relation $R$ on the set $X$ to $Y$ is denoted as an
ordered pair $(x,y)$ or $xRy$ and element $x$ in $X$ and $y$ in $Y$.

\item 
When talking about a set, we discuss about how different elements of the set can
be related in some way. For example, in set of integers $Z$, we may say that
some $s,y \in Z$ are related if $x-y$ is divisible by 2. In other way, $x$ and
$y$ are related only if they are both odd or both even. This idea of expressing
same relation in different way can be formalized as \textit{equivalence
relation}. A relation $R$ is equivalence if it satisfies:
\begin{enumerate}
    \item Reflexive: A \emph{reflexive relation} $R$ on set $X$ is a
subset on \(X \times X\)  is defined as \( R : \{(x,x) : x \in X\}\) and can be
denoted as $xRx$

    \item Symmetric: A \emph{symmetric relation} R on set X is a subset of \(X
\times X\) is defined as \(R: \forall x y \in X: xRy ⟺ yRx\)

    \item Transitive: A relation R is said to be \emph{transitive} on set X, is
a subset of \(X \times X\) such that \(∀ x y z \in X \) if \((x,y) \in R\) and
\((y,z) \in R \) then \((x,z) \in R\)
\end{enumerate}
In other words, a relation R is \emph{equivalence} if it is reflexive, symmetric and transitive.

\item
If in a relation, if every element in domain is mapped to only one element in
the codomain, then we call it a \emph{function}. In other words, function is a
map $f$ from set $X$ (domain) to $Y$ (codomain) is a rule that assigns each
element of $X$ to a unique element in $Y$. This can be expressed using the notation:
\[f:\ X \rightarrow\ Y\]
\[x \ \rightarrow\ f(x)\]
For example, on set of natural numbers $N$ to $N$ we can define a function as:
\[f:\ N \rightarrow N\]
\[x \ \rightarrow\ x^{2}\]   
\item Let $X$ and $Y$ be two sets, and $f:X\rightarrowY$ be the function then:
\begin{enumerate}
    \item The function $f$ is \textit{identity} if $X=Y$ and $f(x)\ =\ x,\ \forall x\ \in
    \ Y$. The identity function can be denoted as $f=Id_s$.
    \item A function f is \emph{injective} if f maps distinct elements of domain to
    distinct elements of codomain. $f$ is injective if $f(x)=f(y) \Rightarrow x = y \forallx,y \in X.$
    \item A function is called \emph{surjective} if given $y \in Y$, there
    exists $x\in X$ such that $f(x) = y$.
    \item A function is called \emph{bijective} if it is both injective and surjective.
\end{enumerate}
\item
An \emph{operation} is defined as a function that can take zero or more inputs
and maps it to a well-defined output value. The number of operands is the arity
of the operation.
\end{itemize}

\section{Universe, type and signature}
The naive set theory defines a set as well-defined collection of objects. If a
set is defined using unrestricted comprehensive
principle\cite{enwiki:1125383109}, then it leads to contradiction. This was
first discovered by mathematician Bertrand Russell, and the paradox is called
Russell's paradox. The paradox defines the set of all sets that are not the
member of themselves \cite{russelPara}. This develops to two kinds of
contradiction:
\begin{enumerate}
\item If the set contains itself, then it should not be a member of itself by
definition
\item If the set does not contain itself then it is not a member of itself.
\end{enumerate}

The signature of an algebraic structure can be defined as a collection of
relation and operations with their arity on the set of an algebraic structure. A
structure with \(\Omega\) signature is called as \(\Omega\) algebra.

A Formal definition of algebra is given in \cite{sankappanavar1981course} as:
For $A$ a nonempty set and $n$ a non-negative integer we define $A_0$ =
\{\(\emptyset\)\}, and, for $n > 0$, $A_n$ is the set of n-tuples of elements
from $A$. An n-ary operation (or function) on $A$ is any function $f$ from $A_n$
to $A$; $n$ is the arity (or rank) of $f$. A finitary operation is an n-ary
operation, for some n. The image of $<a_1,a_2,...,a_n>$ under an n-ary operation
$f$ is denoted by $f(a_1,a_2,...,a_n)$. An operation $f$ on $A$ is called a
nullary operation (or constant) if its arity is zero; it is completely
determined by the image $f(\emptyset)$ in $A$ of the only element \(\emptyset\)
in $A_0$, and as such it is convenient to identify it with the element
$f(\emptyset)$. Thus, a nullary operation is thought of as an element of $A$. An
operation $f$ on $A$ is unary, binary, or ternary if its arity is 1,2, or 3,
respectively.

For example, a group G is an algebra with one nullary (1), one unary
(\textsuperscript{-1}) and one binary (∙) operation represented as $(G, ∙,
\textsuperscript{-1}, 1)$ which satisfy the following axioms. 
\begin{enumerate}
\item Associativity - \( ∀ x y z \in G, x ∙ (y ∙ z) ≈ (x ∙ y) ∙ z \)
\item Identity - \(∀ x \in G, x ∙ 1 ≈ 1 ∙ x ≈ x\)
\item Inverse - \( ∀ x \in G, x ∙ x \textsuperscript{-1} ≈  x
\textsuperscript{-1} ∙ x ≈ 1\)
\end{enumerate}
Where ≈ is the equivalence relation.

The type (or language) of the algebra is a set of function symbols. Each member
of this set is assigned a positive number that is the arity of the member. For
example an algebra of type (2,0) denotes an algebra with one binary operation
and one nullary operation. The group structure defined in previous section is of
type (2,1,0). That is ∙ is a binary operation, \textsuperscript{-1} is a unary
operation and 1 is the nullary operation.

\section{Constructions}
Universal algebra provides definitions of constructions related to algebraic
structures. In this section, we will describe some of these constructions. 
\begin{itemize}
    \item The \textit{congruence} relation for an algebraic structure can be
    defined as an equivalence relation that is compatible with the structure
    such that the operations are well-defined on the equivalence class. A more
    formal definition is for an algebra $A$ of type $F$ is given as, congruence
    relation \(\theta\) on $A$ is defined using compatibility property that
    states that for each n-ary function symbol $f \ \in\ F$ and $x_i,\ y_i\ \in\
    A$, If $x_i\ \theta\ y_i$ holds for \(1\leq i \leq n\) then
    $f^{A}(x_1,...,x_n)\ \theta\ f^{A}(y_1,....,y_n)$ holds
    \cite{sankappanavar1981course}.

    For example, consider group structure $(G, ∙, \textsuperscript{-1}, 1)$. A
    congruence relation on $G$ with binary operation $∙$ is an equivalence
    relation $\equiv$ on $G$ such that, \[g_1\equiv g_2\ \text{and}\ h_1 \equiv h_2
    \Rightarrow g_1 ∙ h_1 \equiv g_2 ∙ h_2\]

    \item A \textit{morphism} is a structure preserving map between two
    algebraic structures. It is an abstraction that generalizes the map between
    two structures or mathematical objects in general. If $A$ and $B$ are two
    algebras of same type $F$, then a homomorphism is defined as a mapping
    \(\alpha\) from algebra $A$ to $B$ such that: \[ \alpha
    f^{A}(a_1,a_2,....,a_n)\ =\ f^{B}(\alpha a_1,\alpha a_2,....,\alpha a_n)\]
    For each n-ary $f$ in $F$ and each sequence $a_1,a_2,....,a_n$ from $A$.
    
    As an example, consider $G_1 = {1,-1,i,-i}$, which is a group under
    multiplication, and $G_2$ = group of all integers under addition. A mapping
    $f$ from $G_1$ to $G_2$ such that $f(x) = i^{n} \ \forall n \in G_2$ is a
    homomorphism.
    
    In \cite{sankappanavar1981course}, the author proves that if \(\alpha: A
    \rightarrow B\) and \(\beta: A \rightarrow B\) are homomorphism on algebra
    $A$ to $B$ such that \(\alpha (a) = \beta (a) \) then \(\alpha\) = \(\beta\)

    Some variants of homomorphism are:
    \begin{enumerate}
        \item  Monomorphism: For two algebras $A$ and $B$, if \(\alpha : A
        \rightarrow B \) is a homomorphism from $A$ to $B$, and if \(\alpha\)
        satisfies one-to-one mapping (i.e., \(\alpha\) is injective) then the
        morphism \(\alpha\) is called a \textit{monomorphism}.

        \item Isomorphism: For two algebras $A$ and $B$, if \(\alpha : A \rightarrow B \)  is
        a monomorphism from $A$ to $B$, and if \(\alpha\) is a bijection from
        $A$ to $B$, then \(\alpha\) is called an \textit{isomorphism}.  

        \item Endomorphism: A homomorphism from an algebra $A$ to itself is
        called \textit{endomorphism}. In other words, if $f$ is a homomorphism on $A$
        such that $f:A\rightarrow A$ then, f is endomorphism.

        \item Automorphism: An isomorphism from an algebra $A$ to itself is
        called \textit{automorphism}.

        \item Epimorphism: For two algebras $A$ and $B$, if \(\alpha : A
        \rightarrow B \) is a homomorphism from $A$ to $B$, and if \(\alpha\) is
        surjective then the morphism \(\alpha\) is called a
        \textit{epimorphism}.
    \end{enumerate}

    \item For algebras $A$, $B$, and $C$ the \textit{composition of morphisms} $f:\ A \
    \rightarrow \ B$ and $g:\ B \rightarrow\ C$ is denoted by the function $g\
    \circ\ f\ :\ A\ \rightarrow \C$ and is defined as $(g\ \circ\ f)\ a = \ g(f\
    a) \ \forall\ a\ \in\ A$. In \cite{sankappanavar1981course}, the author
    proves that the composite of two homomorphism (monomorphism/isomorphism) is
    also a homomorphism (monomorphism/isomorphism).

    \item The \textit{direct product} between two algebras $A$ and $B$ is
    defined as \[A\ \cross\ B\ =\ {(a,b)\ |\ a \ \in \ A\ b \ \in\ B}\] If
    $x,y\in A$ and $u,v \in B$, there is a natural binary operation on $A \cross
    B$ such that: $(a,u)\cdot(b,v):=(ab,uv)$. 

\end{itemize}
