\chapter{Universal Algebra: An Overview}

By the eary 18th century, mathematicians had discovered how to solve polynomial equation of up-to degree 4. There was no general rule to solve a polynomial equation of any degree. It wasn't until, mathematician Evariste Galois discovered a tool called \emph{group} to solve polynomial equation of any degree. This tool was later discovered to be useful in other fields of mathematics such as modulus theory and geometry\cite{enwiki:1107380309}. Knowing the usefulness of this tool, mathematicians abstracted out the axioms of the group into a general tool. Thus evolved the structure \emph{group} that we know today. As group theory, the study of \emph{group} structures evolved, other abstract structures were invented to solve problems. This gave rise to a new field in mathematics called \emph{abstract algebra}. Abstract algebra is the study of algebraic structure and its models or examples \cite{enwiki:1107380309}.An algebraic structure is a tuple containing a 'carrier' set, A, a set of operations that act on A, and a set of axioms involving the operations and A. Some mathematicians were only interested in studying the structures themselves that is the arbitrary interpretation of the language and not the models that includes a theory that holds in the structure. Universal algebra is the study of algebraic structures and its properties. In the recent years, \emph{universal algebra} has seen an exponential growth in its study of theories and applications \cite{sankappanavar1981course}. 

\section{Relation and function}
In order to understand algebraic structures, it is essential to know some basics of relations and functions. In this section, we define relations and functions. We can start with defining a set. \\

A set is a well defined collection of objects. The \emph{Cartesian product} between two sets X and Y,  X \(\times\) Y is defined as a pair \{(x,y) : x \(\in\) X and y \(\in\) Y \}. \\

A \emph{binary relation} is a subset of the Cartesian product of two sets that is a mapping between one set called domain to the other set called the co-domain. A binary relation R on the set X to Y is denoted as an ordered pair (x,y) or xRy and element x in X and y in Y. \\

A \emph{reflexive relation} R on set X is a subset on \(X \times X\)  is defined as \( R : \{(x,x) : x \in X\}\) and can be denoted as xRx \\

A \emph{symmetric relation} R on set X is a subset of \(X \times X\) is defined as \(R: \forall x y \in X: xRy ⟺ yRx\)  \\

A relation R is said to be \emph{transitive} on set X, is a subset of \(X \times X\) such that \(∀ x y z \in X \) if \((x,y) \in R\) and \((y,z) \in R \) then \((x,z) \in R\)\\

A relation R is \emph{equivalence} if it is reflexive, symmetric and transitive. \\

If in a relation, if every element in domain is mapped to only one element in the co-domain, then we call it a \emph{function}.\\

\emph{Image} of the function is the set of all elements in co-domain that is reachable from the function f in its domain.\\

A function f is \emph{injective} if f maps distinct elements of domain to distinct elements of co-domain. That is function f on domain X in injective if \(f(a) = f(b)\) then a = b.\\

A function is called \emph{surjective} if the image of the function is same as its co-domain.\\

A function is called \emph{bijective} if it is both injective and surjective.\\

An \emph{operation} is defined as a function that can take zero or more inputs and maps it to a well defined output value. The number of operands is the arity of the operation.

\section{Universe, type and signature}
The naive set theory defines a set as well-defined collection of objects. If a set is defined using unrestricted comprehensive principle\cite{enwiki:1125383109}, then it leads to contradiction. This was first discovered by mathematician Bertrand Russell, and the paradox is called Russell's paradox. The paradox defines the set of all sets that are not the member of themselves  \cite{russelPara}. This develops to two kinds of contradiction:
\begin{enumerate}
\item If the set contains itself, then it should not be a member of itself by definition
\item If the set does not contain itself the it is not a member of itself.
\end{enumerate}

A Formal definition of algebra is given in \cite{sankappanavar1981course} as \\ "For A a nonempty set and n a nonnegative integer we define A\textsuperscript{0} = \{\(\emptyset\)\} , and, for n > 0, A\textsuperscript{n} is the set of n-tuples of elements from A. An n-ary operation (or function) on A is any function f from A\textsuperscript{n} to A; n is the arity (or rank) of f. A finitary operation is an n-ary operation, for some n. The image of <a\textsubscript{1}, . . ., a\textsubscript{n}> under an n-ary operation f is denoted by f(a\textsubscript{1}, . . ., a\textsubscript{n}). An operation f on A is called a nullary operation (or constant) if its arity is zero; it is completely determined by the image f(\(\emptyset\)) in A of the only element \(\emptyset\) in A\textsuperscript{0}, and as such it is convenient to identify it with the element f(\(\emptyset\)). Thus a nullary operation is thought of as an element of A. An operation f on A is unary, binary, or ternary if its arity is 1,2, or 3, respectively"\\ \\
In previous section we saw that group structure is one of the early structures studied in universal algebra. A group G is an algebra with one nullary (1), one unary (\textsuperscript{-1}) and one binary (∙) operation represented as (G, ∙, \textsuperscript{-1}, 1) which satisfy the following axioms. 
\begin{enumerate}
\item Associativity - \( ∀ x y z \in G, x ∙ (y ∙ z) ≈ (x ∙ y) ∙ z \)
\item Identity - \(∀ x \in G, x ∙ 1 ≈ 1 ∙ x ≈ x\)
\item Inverse - \( ∀ x \in G, x ∙ x \textsuperscript{-1} ≈  x \textsuperscript{-1} ∙ x ≈ 1\)
\end{enumerate}
Where ≈ is the equivalence relation.

The type (or language) of the algebra is a set of function symbols. Each member of this set is assigned a positive number that is the arity of the member. For example an algebra of type (2,0) denotes an algebra with one binary operation and one nullary operation. The group structure defined in previous section is of type (2,1,0). That is ∙ is a binary operation, \textsuperscript{-1} is a unary operation and 1 is the nullary operation. \\

The signature of an algebraic structure can be defined as a collection of relation and operations with their arity on the set of an algebraic structure. A structure with \(\Omega\) signature is called as \(\Omega\) algebra.

\section{Congruence and Morphism}
The congruence relation for an algebraic structure can be defined as an equivalence relation that is compatible with the structure such that the operations are well defined on the equivalence class. A more formal definition is for an algebra A of type F, congruence relation \(\theta\) on A is defined using compatibility property that states that for each n-ary function symbol f \(\in\) F and x\textsubscript{i} , y\textsubscript{i} \(\in\) A, If x\textsubscript{i} \(\theta\) y\textsubscript{i}  holds for \(1\leq i \leq n\) then f\textsuperscript{A}(x\textsubscript{1},...,x\textsubscript{n})\(\theta\)f\textsuperscript{A}(y\textsubscript{1},....,y\textsubscript{n}) holds\cite{sankappanavar1981course}.\\

A Morphism is a structure preserving map between two algebraic structures. It is an abstraction that generalizes the map between two structures or mathematical objects in general. \\
If A and B are two algebras of same type F, then a homomorphism is defined as a mapping \(\alpha\) from algebra A to B such that
\begin{center}
\(\alpha f\textsuperscript{A}(a\textsubscript{1},a\textsubscript{2},....,a\textsubscript{n}) = f\textsuperscript{B}(\alpha a\textsubscript{1},\alpha a\textsubscript{2},....,\alpha a\textsubscript{n})\)\\
\end{center}
for each n-ary f in F and each sequence a\textsubscript{1},a\textsubscript{2},....,a\textsubscript{n} from A. In \cite{sankappanavar1981course}, the author proves that if \(\alpha\): A \(\rightarrow\) B and \(\beta\): A \(\rightarrow\) B are homomorphisms on algebra A to B such that \(\alpha (a) = \beta (a) \) then \(\alpha\) = \(\beta\)
\\

For two algebras A and B, if \(\alpha : A \rightarrow B \)  is a homomorphism from A to B, if \(\alpha\) satisfies one-to-one mapping (i.e., \(\alpha\) is injective) then the morphism \(\alpha\) is called monomorphism. \\

For two algebras A and B, if \(\alpha : A \rightarrow B \)  is a Monomorphism from A to B, if \(\alpha\) is a bijection from A to B, then \(\alpha\) is called isomorphism.

In \cite{sankappanavar1981course}, the author proves that the composite of two homomorphism (monomorphism/isomorphism) is also a homomorphism (monomorphism/isomorphism).
