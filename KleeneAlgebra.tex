\chapter{Theory of Kleene Algebra in Agda}
Kleene algebra is an algebraic structure named after Stephen Cole Kleene, for his invention of finite automata and regular expressions. Kleene algebras are used in various contexts such as relational algebra, automata and formal theory, design and analysis of algorithms and program analysis and compiler optimization \cite{kozen1997kleene}. Kleene algebra generalizes operations from regular expressions. The axiomization of the algebra if regular events was recently proposed in 1966 but it was in 1984, a completeness theorem for relational algebra with a proper subclass of Kleene algebra was given. \cite{kozen1994completeness}. Although there are some differences in axioms of kleene algebra, in this chapter we consider the axioms defined in \cite{kozen1994completeness}

\section{Definition}
A set S with two binary operations + and * generally called addition and multiplication such that (S,+) is a commutative monoid, (S,*) is a monoid and + distributes over * with annhiliating zero is called a semiring. A semiring satisfying idempotent property is called idempotent semiring. A Kleene Algebra over set S is idempotent semiring with \( \star \) operator that satisfies the following axioms.
\begin{equation}\label{eq_starrightexpansive}
1 + (x ∙ (x *)) \leq x *
\end{equation}
\begin{equation}\label{eq_starleftexpansive}
1 + (x *) ∙ x \leq x *
\end{equation}
\begin{equation}\label{eq_starleftdestructive}
\forall a, b, x \in S: If b + a  ∙ x \leq x then, (a *)  ∙ b \leq x
\end{equation}
\begin{equation}\label{eq_starrightdestructive}
\forall a, b, x \in S: If b + x  ∙ a \leq x then, b  ∙ (a *) \leq x
\end{equation}
In Agda, strong axioms of \(\star\) are given. That is equivalance is directly given and kleene algebra with partial and pre order structures are defined in Algebra.Ordered hierarchy. 
\begin{Verbatim}[commandchars=\\\{\},samepage=true]
StarRightExpansive : A → Op\textsubscript{2} A → Op\textsubscript{2} A 
	→ Op\textsubscript{1} A → Set _
StarRightExpansive e _+_ _∙_ _* = ∀ x → (e + (x ∙ (x *))) ≈ (x *)

StarLeftExpansive : A → Op\textsubscript{2} A → Op\textsubscript{2} A
	 → Op\textsubscript{1} A → Set _
StarLeftExpansive e _+_ _∙_ _* = ∀ x →  (e + ((x *) ∙ x)) ≈ (x *)

StarLeftDestructive : Op\textsubscript{2} A → Op\textsubscript{2} A
	 → Op\textsubscript{1} A → Set _
StarLeftDestructive _+_ _∙_ _* = ∀ a b x → (b + (a ∙ x)) ≈ x 
	→ ((a *) ∙ b) ≈ x

StarRightDestructive : Op\textsubscript{2} A → Op\textsubscript{2} A
	 → Op\textsubscript{1} A → Set _
StarRightDestructive _+_ _∙_ _* = ∀ a b x → (b + (x ∙ a)) ≈ x 
	→ (b ∙ (a *)) ≈ x
\end{Verbatim}
The Kleene algebra can be structurally derived from idempotent semiring. 
\begin{Verbatim}[commandchars=\\\{\},samepage=true]
record IsKleeneAlgebra (+ * : Op\textsubscript{2} A) (⋆ : Op\textsubscript{1} A)
		 (0# 1# : A) : Set (a ⊔ ℓ) where
  field
    isIdempotentSemiring  : IsIdempotentSemiring + * 0# 1#
    starExpansive         : StarExpansive 1# + * ⋆
    starDestructive       : StarDestructive + * ⋆

  open IsIdempotentSemiring isIdempotentSemiring public
\end{Verbatim}
The bundle version of kleene algebra is defined as: 
\begin{Verbatim}[commandchars=\\\{\},samepage=true]
record KleeneAlgebra c ℓ : Set (suc (c ⊔ ℓ)) where
  infix  8 _⋆
  infixl 7 _*_
  infixl 6 _+_
  infix  4 _≈_
  field
    Carrier               : Set c
    _≈_                   : Rel Carrier ℓ
    _+_                   : Op₂ Carrier
    _*_                   : Op₂ Carrier
    _⋆                    : Op₁ Carrier
    0#                    : Carrier
    1#                    : Carrier
    isKleeneAlgebra       : IsKleeneAlgebra _≈_ _+_ _*_ _⋆ 0# 1#

  open IsKleeneAlgebra isKleeneAlgebra public

  idempotentSemiring : IdempotentSemiring _ _
  idempotentSemiring = record \{ isIdempotentSemiring = isIdempotentSemiring \}

  open IdempotentSemiring idempotentSemiring public
    using
    ( _≉_; +-rawMagma; +-magma; +-unitalMagma; +-commutativeMagma
    ; +-semigroup; +-commutativeSemigroup
    ; *-rawMagma; *-magma; *-semigroup
    ; +-rawMonoid; +-monoid; +-commutativeMonoid
    ; *-rawMonoid; *-monoid
    ; nearSemiring; semiringWithoutOne
    ; semiringWithoutAnnihilatingZero
    ; rawSemiring; semiring
    )
\end{Verbatim}
\section{Direct Product}
The direct product K \(\times\) L of two kleene algebra structures K and NL is defined as a pair (k,l) where k\(\in\) K and l \(\in\) L.
\begin{Verbatim}
kleeneAlgebra : KleeneAlgebra a ℓ\textsubscript{1} 
	→ KleeneAlgebra b ℓ\textsubscript{2} →
	 KleeneAlgebra (a ⊔ b) (ℓ\textsubscript{1} ⊔ ℓ\textsubscript{2})
kleeneAlgebra K L = record
  { isKleeneAlgebra = record
      { isIdempotentSemiring = IdempotentSemiring.isIdempotentSemiring 
		(idempotentSemiring K.idempotentSemiring L.idempotentSemiring)
      ; starExpansive = (λ x → (K.starExpansiveˡ  , L.starExpansiveˡ) <*> x)
                      , (λ x → (K.starExpansiveʳ  , L.starExpansiveʳ) <*> x)
      ; starDestructive = (λ a b x x₁ →
		 (K.starDestructiveˡ  , L.starDestructiveˡ)
		 <*> a <*> b <*> x <*> x₁)
                        , (λ a b x x₁ → 
		(K.starDestructiveʳ  , L.starDestructiveʳ)
		 <*> a <*> b <*> x <*> x₁)
      }
  } where module K = KleeneAlgebra K;  module L = KleeneAlgebra L
\end{Verbatim}

\section{Properties}
In this section we prove some of the properties of Kleene algebra

Let (K, +, *, \(\star\), 0, 1) be a Kleene algebra then,
a) 0⋆ = 1\\
b) 1⋆ = 1\\
c) \(\forall x \in K\): 1 + x⋆ = x⋆\\
d) \(\forall x \in K\): x + x * x⋆ = x⋆\\
e) \(\forall x \in K\): x + x⋆ * x = x⋆\\
f) \(\forall x \in K\): x + x⋆ = x⋆\\
g) \(\forall x \in K\): 1 + x + x⋆ = x⋆\\
h) \(\forall x \in K\): 0 + x + x⋆ = x⋆\\
i) \(\forall x \in K\): x⋆ * x⋆ = x⋆\\
j) \(\forall x \in K\): x⋆⋆ = x⋆\\
k) \(\forall x , y \in K\): If x = y then, x⋆ = y⋆\\
l) \(\forall a, b, x \in K\): If a * x = x * b then, a⋆ * x = x * b⋆\\
m) \(\forall x, y \in K\): (x * y) ⋆ * x ≈ x * (y * x) ⋆\\
Proof:
\begin{Verbatim}
a)
0⋆≈1 : 0# ⋆ ≈ 1#
0⋆≈1 = begin
  0# ⋆           ≈⟨ sym (starExpansiveˡ 0#) ⟩
  1# + 0# ⋆ * 0# ≈⟨ +-congˡ ( zeroʳ (0# ⋆)) ⟩
  1# + 0#        ≈⟨ +-identityʳ 1# ⟩
  1#             ∎
\end{Verbatim}

\begin{Verbatim}
b)
1+11≈1 : 1# + 1# * 1# ≈ 1#
1+11≈1 = begin
  1# + 1# * 1#  ≈⟨ +-congˡ ( *-identityʳ 1#) ⟩
  1# + 1#       ≈⟨ +-idem 1# ⟩
  1#            ∎

1⋆≈1 : 1# ⋆ ≈ 1#
1⋆≈1 = begin
  1# ⋆       ≈⟨ sym (*-identityʳ (1# ⋆)) ⟩
  1# ⋆ * 1#  ≈⟨ starDestructiveˡ 1# 1# 1# 1+11≈1 ⟩
  1#         ∎
\end{Verbatim}

\begin{Verbatim}
c)
1+x⋆≈x⋆ : ∀ x → 1# + x ⋆ ≈ x ⋆
1+x⋆≈x⋆ x = sym (begin
  x ⋆                   ≈⟨ sym (starExpansiveʳ x) ⟩
  1# + x * x ⋆          ≈⟨ +-congʳ (sym (+-idem 1#)) ⟩
  1# + 1# + x * x ⋆     ≈⟨ +-assoc 1# 1# ((x * x ⋆ )) ⟩
  1# + (1# + x * x ⋆)   ≈⟨ +-congˡ (starExpansiveʳ x) ⟩
  1# + x ⋆              ∎)
\end{Verbatim}

\begin{Verbatim}
d)
x⋆+xx⋆≈x⋆ : ∀ x → x ⋆ + x * x ⋆ ≈ x ⋆
x⋆+xx⋆≈x⋆ x = begin
  x ⋆ + x * x ⋆         ≈⟨ +-congʳ (sym (1+x⋆≈x⋆ x)) ⟩
  1# + x ⋆ + x * x ⋆    ≈⟨ +-congʳ (+-comm 1# ((x ⋆))) ⟩
  x ⋆ + 1# + x * x ⋆    ≈⟨ +-assoc ((x ⋆)) 1# ((x * x ⋆ )) ⟩
  x ⋆ + (1# + x * x ⋆)  ≈⟨ +-congˡ (starExpansiveʳ x) ⟩
  x ⋆ + x ⋆             ≈⟨ +-idem (x ⋆) ⟩
  x ⋆                   ∎
\end{Verbatim}

\begin{Verbatim}
e)
x⋆+x⋆x≈x⋆ : ∀ x → x ⋆ + x ⋆ * x ≈ x ⋆
x⋆+x⋆x≈x⋆ x = begin
  x ⋆ + x ⋆ * x         ≈⟨ +-congʳ (sym (1+x⋆≈x⋆ x)) ⟩
  1# + x ⋆ + x ⋆ * x    ≈⟨ +-congʳ (+-comm 1# (x ⋆)) ⟩
  x ⋆ + 1# + x ⋆ * x    ≈⟨ +-assoc (x ⋆) 1# (x ⋆ * x) ⟩
  x ⋆ + (1# + x ⋆ * x)  ≈⟨ +-congˡ (starExpansiveˡ x) ⟩
  x ⋆ + x ⋆             ≈⟨ +-idem (x ⋆) ⟩
  x ⋆                   ∎
\end{Verbatim}

\begin{Verbatim}
f)
x+x⋆≈x⋆ : ∀ x → x + x ⋆ ≈ x ⋆
x+x⋆≈x⋆ x = begin
  x + x ⋆                  ≈⟨ +-congˡ (sym (starExpansiveʳ x)) ⟩
  x + (1# + x * x ⋆)       ≈⟨ +-congʳ (sym (*-identityʳ x)) ⟩
  x * 1# + (1# + x * x ⋆)  ≈⟨ sym (+-assoc (x * 1#) 1# (x * x ⋆)) ⟩
  x * 1# + 1# + x * x ⋆    ≈⟨ +-congʳ (+-comm (x * 1#) 1#) ⟩
  1# + x * 1# + x * x ⋆    ≈⟨ +-assoc 1# (x * 1#) (x * x ⋆) ⟩
  1# + (x * 1# + x * x ⋆)  ≈⟨ +-congˡ (sym (distribˡ x 1# ((x ⋆)))) ⟩
  1# + x * (1# + x ⋆)      ≈⟨ +-congˡ (*-congˡ (1+x⋆≈x⋆ x)) ⟩
  1# + x * x ⋆             ≈⟨ (starExpansiveʳ x) ⟩
  x ⋆                      ∎
\end{Verbatim}

\begin{Verbatim}
g)
1+x+x⋆≈x⋆ : ∀ x → 1# + x + x ⋆ ≈ x ⋆
1+x+x⋆≈x⋆ x = begin
  1# + x + x ⋆    ≈⟨ +-assoc 1# x (x ⋆) ⟩
  1# + (x + x ⋆)  ≈⟨ +-congˡ (x+x⋆≈x⋆ x) ⟩
  1# + x ⋆        ≈⟨ 1+x⋆≈x⋆ x ⟩
  x ⋆             ∎
\end{Verbatim}

\begin{Verbatim}
h)
0+x+x⋆≈x⋆ : ∀ x → 0# + x + x ⋆ ≈ x ⋆
0+x+x⋆≈x⋆ x = begin
  0# + x + x ⋆    ≈⟨ +-assoc 0# x (x ⋆) ⟩
  0# + (x + x ⋆)  ≈⟨ +-identityˡ ((x + x ⋆)) ⟩
  (x + x ⋆)       ≈⟨ x+x⋆≈x⋆ x ⟩
  x ⋆             ∎
\end{Verbatim}

\begin{Verbatim}
i)
x⋆x⋆≈x⋆ : ∀ x → x ⋆ * x ⋆ ≈ x ⋆
x⋆x⋆≈x⋆ x = starDestructiveˡ x (x ⋆) (x ⋆) (x⋆+xx⋆≈x⋆ x)
\end{Verbatim}

\begin{Verbatim}
j)
1+x⋆x⋆≈x⋆ : ∀ x → 1# + x ⋆ * x ⋆ ≈ x ⋆
1+x⋆x⋆≈x⋆ x = begin
  1# + x ⋆ * x ⋆  ≈⟨ +-congˡ (x⋆x⋆≈x⋆ x) ⟩
  1# + x ⋆        ≈⟨ 1+x⋆≈x⋆ x ⟩
  x ⋆             ∎

x⋆⋆≈x⋆ : ∀ x → (x ⋆) ⋆ ≈ x ⋆
x⋆⋆≈x⋆ x = begin
  (x ⋆) ⋆        ≈⟨ sym (*-identityʳ ((x ⋆) ⋆)) ⟩
  (x ⋆) ⋆ * 1#   ≈⟨ starDestructiveˡ (x ⋆) 1# (x ⋆) (1+x⋆x⋆≈x⋆ x) ⟩
  x ⋆            ∎
\end{Verbatim}

\begin{Verbatim}
k)
x≈y⇒1+xy⋆≈y⋆ : ∀ x y → x ≈  y → 1# + x * y ⋆ ≈ y ⋆
x≈y⇒1+xy⋆≈y⋆ x y eq = begin
  1# + x * y ⋆  ≈⟨ +-congˡ (*-congʳ (eq)) ⟩
  1# + y * y ⋆  ≈⟨ starExpansiveʳ y ⟩
  y ⋆           ∎

x≈y⇒x⋆≈y⋆ : ∀ x y → x ≈ y → x ⋆ ≈ y ⋆
x≈y⇒x⋆≈y⋆ x y eq = begin
  x ⋆       ≈⟨ sym (*-identityʳ (x ⋆)) ⟩
  x ⋆ * 1#  ≈⟨ (starDestructiveˡ x 1# (y ⋆) (x≈y⇒1+xy⋆≈y⋆ x y eq)) ⟩
  y ⋆       ∎
\end{Verbatim}

\begin{Verbatim}
l)
ax≈xb⇒x+axb⋆≈xb⋆ : ∀ x a b → a * x ≈ x * b → x + a * (x * b ⋆) ≈ x * b ⋆
ax≈xb⇒x+axb⋆≈xb⋆ x a b eq = begin
  x + a * (x * b ⋆)       ≈⟨ +-congˡ (sym(*-assoc a x (b ⋆))) ⟩
  x + a * x * b ⋆         ≈⟨ +-congʳ (sym (*-identityʳ x)) ⟩
  x * 1# + a * x * b ⋆    ≈⟨ +-congˡ (*-congʳ (eq)) ⟩
  x * 1# + x * b * b ⋆    ≈⟨ +-congˡ (*-assoc x b (b ⋆) ) ⟩
  x * 1# + x * (b * b ⋆)  ≈⟨ sym (distribˡ x 1# (b * b ⋆)) ⟩
  x * (1# + b * b ⋆)      ≈⟨ *-congˡ (starExpansiveʳ b) ⟩
  x * b ⋆                 ∎

ax≈xb⇒a⋆x≈xb⋆ : ∀ x a b → a * x ≈ x * b → a ⋆ * x ≈ x * b ⋆
ax≈xb⇒a⋆x≈xb⋆ x a b eq = starDestructiveˡ a x ((x * b ⋆)) (ax≈xb⇒x+axb⋆≈xb⋆ x a b eq)
\end{Verbatim}

\begin{Verbatim}
m)
[xy]⋆x≈x[yx]⋆ : ∀ x y → (x * y) ⋆ * x ≈ x * (y * x) ⋆
[xy]⋆x≈x[yx]⋆ x y = ax≈xb⇒a⋆x≈xb⋆ x (x * y) (y * x) (*-assoc x y x)
\end{Verbatim}