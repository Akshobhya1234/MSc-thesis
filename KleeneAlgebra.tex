\chapter{Theory of Kleene Algebra in Agda}
Kleene algebra is an algebraic structure named after Stephen Cole Kleene, for
his invention of finite automata and regular expressions. Kleene algebras are
used in various contexts such as relational algebra, automata and formal theory,
design and analysis of algorithms and program analysis and compiler optimization
\cite{kozen1997kleene}. Kleene algebra generalizes operations from regular
expressions. The axiomization of the algebra if regular events was recently
proposed in 1966 but it was in 1984, a completeness theorem for relational
algebra with a proper subclass of Kleene algebra was given.
\cite{kozen1994completeness}. Although there are some differences in axioms of
kleene algebra, in this chapter we consider the axioms defined in
\cite{kozen1994completeness}

\section{Definition}
A set S with two binary operations + and ∙ generally called addition and
multiplication such that (S,+) is a commutative monoid, (S,∙) is a monoid and +
distributes over ∙ with annhiliating zero is called a semiring. A semiring
satisfying idempotent property is called idempotent semiring. A Kleene Algebra
over set S is idempotent semiring with \textsuperscript{*} operator that satisfies the
following axioms.
\begin{equation}\label{eq_starrightexpansive}
1\ +\ (x\ ∙\ (x^{*}))\ \leq\ x^{*}
\end{equation}
\begin{equation}\label{eq_starleftexpansive}
1\ +\ (x^{*})\ ∙\ x\ \leq\ x^{*}
\end{equation}
\begin{equation}\label{eq_starleftdestructive}
\forall\ a,\ b,\ x\ \in\ S:\ \text{If} \ b\ +\ a\  ∙\ x\ \leq\ x\ \text{then},\ (a^{*})\ ∙\ b\ \leq\ x
\end{equation}
\begin{equation}\label{eq_starrightdestructive}
\forall\ a,\ b,\ x\ \in\ S:\ \text{If} \ b\ +\ x\ ∙\ a\ \leq\ x \  \text{then},\ b\ ∙\ (a^{*})\ \leq\ x
\end{equation}
In Agda, strong axioms of operator \ \textsuperscript{*} is given. That is instead of
partial order, equivalance is given in the above equations. Kleene algebra with
partial and pre order structures are defined in "Algebra.Ordered.Structures" in
Agda standard library.\footnote{Agda standard library uses operator + for
addition, * for multiplication and $\star$ for star operation.} 
\begin{minted}[breaklines,samepage]{Agda}
StarRightExpansive : A → Op₂ A → Op₂ A → Op₁ A → Set _
StarRightExpansive e _+_ _∙_ _* = ∀ x → (e + (x ∙ (x *))) ≈ (x *)
\end{minted}
\begin{minted}[breaklines,samepage]{Agda}
StarLeftExpansive : A → Op₂ A → Op₂ A → Op₁ A → Set _
StarLeftExpansive e _+_ _∙_ _* = ∀ x →  (e + ((x *) ∙ x)) ≈ (x *)
\end{minted}
\begin{minted}[breaklines,samepage]{Agda}
StarExpansive : A → Op₂ A → Op₂ A → Op₁ A → Set _
StarExpansive e _+_ _∙_ _* = (StarLeftExpansive e _+_ _∙_ _*) × (StarRightExpansive e _+_ _∙_ _*)
\end{minted}
\begin{minted}[breaklines,samepage]{Agda}
StarLeftDestructive : Op₂ A → Op₂ A → Op₁ A → Set _
StarLeftDestructive _+_ _∙_ _* = ∀ a b x → (b + (a ∙ x)) ≈ x → ((a *) ∙ b) ≈ x
\end{minted}
\begin{minted}[breaklines,samepage]{Agda}
StarRightDestructive : Op₂ A → Op₂ A → Op₁ A → Set _
StarRightDestructive _+_ _∙_ _* = ∀ a b x → (b + (x ∙ a)) ≈ x → (b ∙ (a *)) ≈ x
\end{minted}
\begin{minted}[breaklines,samepage]{Agda}
StarDestructive : Op₂ A → Op₂ A → Op₁ A → Set _
StarDestructive _+_ _∙_ _* = (StarLeftDestructive _+_ _∙_ _*) × (StarRightDestructive _+_ _∙_ _*)
\end{minted}
The Kleene algebra can be structurally derived from idempotent semiring. 
\begin{minted}[breaklines,samepage]{Agda}
record IsKleeneAlgebra (+ * : Op₂ A) (⋆ : Op₁ A) (0# 1# : A) : Set (a ⊔ ℓ) where
field
  isIdempotentSemiring  : IsIdempotentSemiring + * 0# 1#
  starExpansive         : StarExpansive 1# + * ⋆
  starDestructive       : StarDestructive + * ⋆

open IsIdempotentSemiring isIdempotentSemiring public
\end{minted}
The bundled version of kleene algebra is defined as: 
\begin{minted}[breaklines,samepage]{Agda}
record KleeneAlgebra c ℓ : Set (suc (c ⊔ ℓ)) where
infix  8 _⋆
infixl 7 _*_
infixl 6 _+_
infix  4 _≈_
field
  Carrier               : Set c
  _≈_                   : Rel Carrier ℓ
  _+_                   : Op₂ Carrier
  _*_                   : Op₂ Carrier
  _⋆                    : Op₁ Carrier
  0#                    : Carrier
  1#                    : Carrier
  isKleeneAlgebra       : IsKleeneAlgebra _≈_ _+_ _*_ _⋆ 0# 1#

open IsKleeneAlgebra isKleeneAlgebra public
\end{minted}
\section{Morphism}
A morphism of Kleene algebra is a function between two Kleene algebras that
preserves the algebraic structure of the underlying semiring and the Kleene star
operation. Morphisms of Kleene algebra are important in the study of regular
languages and automata, as they allow us to relate the behavior of different
automata and regular expressions to each other. Morphism of Kleene algebra help
to generalize the theory of regular languages and finite automata to more
general algebraic structures.
\begin{minted}[samepage,breaklines]{Agda}
  record IsKleeneAlgebraHomomorphism (⟦_⟧ : A → B) : Set (a ⊔ ℓ₁ ⊔ ℓ₂) where
  field
    isSemiringHomomorphism : IsSemiringHomomorphism ⟦_⟧
    ⋆-homo :  Homomorphic₁ ⟦_⟧ _⋆₁ _⋆₂

  open IsSemiringHomomorphism isSemiringHomomorphism public
\end{minted}
A Kleene algebra homomorphism which is injective gives a monomorphism. 
\begin{minted}[samepage,breaklines]{Agda}
record IsKleeneAlgebraMonomorphism (⟦_⟧ : A → B) : Set (a ⊔ ℓ₁ ⊔ ℓ₂) where
  field
    isKleeneAlgebraHomomorphism   : IsKleeneAlgebraHomomorphism ⟦_⟧
    injective                     : Injective ⟦_⟧

  open IsKleeneAlgebraHomomorphism isKleeneAlgebraHomomorphism public
\end{minted}

A surjective monomorphism of a Kleene algebra gives isomorphism.
\begin{minted}[samepage,breaklines]{Agda}
record IsKleeneAlgebraIsomorphism (⟦_⟧ : A → B) : Set (a ⊔ b ⊔ ℓ₁ ⊔ ℓ₂) where
  field
    isKleeneAlgebraMonomorphism   : IsKleeneAlgebraMonomorphism ⟦_⟧
    surjective                    : Surjective ⟦_⟧

  open IsKleeneAlgebraMonomorphism isKleeneAlgebraMonomorphism public
\end{minted}

\section{Morphism composition}
If $f$ is a morphism such that $f\ :\ a \ \rightarrow \ b$ and $g$ is a morphism
on same Kleene algebra structure such that $g\ :\ b\ \rightarrow \ c$, then
composition of morphism can be defined as $g \ ∘\ f\ :\ a \ \rightarrow \ c$.
\begin{minted}[samepage,breaklines]{Agda}
  isKleeneAlgebraHomomorphism
  : IsKleeneAlgebraHomomorphism K₁ K₂ f
  → IsKleeneAlgebraHomomorphism K₂ K₃ g
  → IsKleeneAlgebraHomomorphism K₁ K₃ (g ∘ f)
isKleeneAlgebraHomomorphism f-homo g-homo = record
  { isSemiringHomomorphism = isSemiringHomomorphism ≈₃-trans F.isSemiringHomomorphism G.isSemiringHomomorphism
  ; ⋆-homo              = λ x → ≈₃-trans (G.⟦⟧-cong (F.⋆-homo x)) (G.⋆-homo (f x))
  } where module F = IsKleeneAlgebraHomomorphism f-homo; module G = IsKleeneAlgebraHomomorphism g-homo
\end{minted}
The composition of monomorphism and isomorphism can be defined similar to
homomorphism and can be found in Agda standard library.

\section{Direct Product}
The \textit{direct product} $K \ \times \ L$ of two kleene algebra structures
$K$ and $L$ is defined as a pair $(k,l)$ where $k \ \in \ K$ and $l \ \in \ L$.
\begin{minted}[breaklines,samepage]{Agda}
kleeneAlgebra : KleeneAlgebra a ℓ₁ → KleeneAlgebra b ℓ₂ → KleeneAlgebra (a ⊔ b) (ℓ₁ ⊔ ℓ₂)
kleeneAlgebra K L = record
  { isKleeneAlgebra = record
      { isIdempotentSemiring = IdempotentSemiring.isIdempotentSemiring (idempotentSemiring K.idempotentSemiring L.idempotentSemiring)
      ; starExpansive = (λ x → (K.starExpansiveˡ  , L.starExpansiveˡ) <*> x)
                      , (λ x → (K.starExpansiveʳ  , L.starExpansiveʳ) <*> x)
      ; starDestructive = (λ a b x x₁ → (K.starDestructiveˡ  , L.starDestructiveˡ) <*> a <*> b <*> x <*> x₁)
                        , (λ a b x x₁ → (K.starDestructiveʳ  , L.starDestructiveʳ) <*> a <*> b <*> x <*> x₁)
      }
  } where module K = KleeneAlgebra K;  module L = KleeneAlgebra L
\end{minted}

\section{Properties}
In this section we prove some properties of kleene algebra

Let (K, +, *, \textsuperscript{*}, 0, 1) be a kleene algebra then:
\begin{enumerate}
\item $0^{*}\ =\ 1$
\item $1^{*}\ =\ 1$
\item $\forall \ x\ \in\ K:\ 1\ +\ x^{*}\ =\ x^{*}$
\item $\forall \ x\ \in\ K:\ x\ +\ x\ *\ x^{*} = x^{*}$
\item $\forall \ x\ \in\ K:\ x\ +\ x^{*}\ *\ x\ =\ x^{*}$
\item $\forall \ x\ \in\ K:\ x\ +\ x^{*}\ =\ x^{*}$
\item $\forall \ x\ \in\ K:\ 1\ + x\ + x^{*}\ =\ x^{*}$
\item $\forall \ x\ \in\ K:\ 0\ + x\ +\ x^{*}\ =\ x^{*}$
\item $\forall \ x\ \in\ K:\ x^{*}\ *\ x^{*}\ =\ x^{*}$
\item $\forall \ x\ \in\ K:\ x^{**}\ =\ x^{*}$
\item $\forall\ x,\ y\ \in\ K:\ \text{If}\ x\ =\ y\ \text{then},\ x^{*}\ =\ y^{*}$
\item $\forall\ a,\ b,\ x\ \in\ K:\ \text{If}\ a\ *\ x\ =\ x\ *\ b\ \text{then},\ a^{*}\ *\ x\ =\ x\ *\ b^{*}$
\item $\forall\ x,\ y\ \in\ K:\ (x\ *\ y)^{*}\ *\ x\ =\ x\ *\ (y\ *\ x)^{*}$
\end{enumerate}
Proof:
\begin{enumerate}
\item
\begin{minted}[breaklines,samepage]{Agda}
0⋆≈1 : 0# ⋆ ≈ 1#
0⋆≈1 = begin
  0# ⋆           ≈⟨ sym (starExpansiveˡ 0#) ⟩
  1# + 0# ⋆ * 0# ≈⟨ +-congˡ ( zeroʳ (0# ⋆)) ⟩
  1# + 0#        ≈⟨ +-identityʳ 1# ⟩
  1#             ∎
\end{minted}
\item
\begin{minted}[breaklines,samepage]{Agda}
1+11≈1 : 1# + 1# * 1# ≈ 1#
1+11≈1 = begin
  1# + 1# * 1#  ≈⟨ +-congˡ ( *-identityʳ 1#) ⟩
  1# + 1#       ≈⟨ +-idem 1# ⟩
  1#            ∎

1⋆≈1 : 1# ⋆ ≈ 1#
1⋆≈1 = begin
  1# ⋆       ≈⟨ sym (*-identityʳ (1# ⋆)) ⟩
  1# ⋆ * 1#  ≈⟨ starDestructiveˡ 1# 1# 1# 1+11≈1 ⟩
  1#         ∎
\end{minted}
\item
\begin{minted}[breaklines,samepage]{Agda}
1+x⋆≈x⋆ : ∀ x → 1# + x ⋆ ≈ x ⋆
1+x⋆≈x⋆ x = sym (begin
  x ⋆                   ≈⟨ sym (starExpansiveʳ x) ⟩
  1# + x * x ⋆          ≈⟨ +-congʳ (sym (+-idem 1#)) ⟩
  1# + 1# + x * x ⋆     ≈⟨ +-assoc 1# 1# ((x * x ⋆ )) ⟩
  1# + (1# + x * x ⋆)   ≈⟨ +-congˡ (starExpansiveʳ x) ⟩
  1# + x ⋆              ∎)
\end{minted}
\item
\begin{minted}[breaklines,samepage]{Agda}
x⋆+xx⋆≈x⋆ : ∀ x → x ⋆ + x * x ⋆ ≈ x ⋆
x⋆+xx⋆≈x⋆ x = begin
  x ⋆ + x * x ⋆         ≈⟨ +-congʳ (sym (1+x⋆≈x⋆ x)) ⟩
  1# + x ⋆ + x * x ⋆    ≈⟨ +-congʳ (+-comm 1# ((x ⋆))) ⟩
  x ⋆ + 1# + x * x ⋆    ≈⟨ +-assoc ((x ⋆)) 1# ((x * x ⋆ )) ⟩
  x ⋆ + (1# + x * x ⋆)  ≈⟨ +-congˡ (starExpansiveʳ x) ⟩
  x ⋆ + x ⋆             ≈⟨ +-idem (x ⋆) ⟩
  x ⋆                   ∎
\end{minted}
\item
\begin{minted}[breaklines,samepage]{Agda}
x⋆+x⋆x≈x⋆ : ∀ x → x ⋆ + x ⋆ * x ≈ x ⋆
x⋆+x⋆x≈x⋆ x = begin
  x ⋆ + x ⋆ * x         ≈⟨ +-congʳ (sym (1+x⋆≈x⋆ x)) ⟩
  1# + x ⋆ + x ⋆ * x    ≈⟨ +-congʳ (+-comm 1# (x ⋆)) ⟩
  x ⋆ + 1# + x ⋆ * x    ≈⟨ +-assoc (x ⋆) 1# (x ⋆ * x) ⟩
  x ⋆ + (1# + x ⋆ * x)  ≈⟨ +-congˡ (starExpansiveˡ x) ⟩
  x ⋆ + x ⋆             ≈⟨ +-idem (x ⋆) ⟩
  x ⋆                   ∎
\end{minted}
\item
\begin{minted}[breaklines,samepage]{Agda}
x+x⋆≈x⋆ : ∀ x → x + x ⋆ ≈ x ⋆
x+x⋆≈x⋆ x = begin
  x + x ⋆                  ≈⟨ +-congˡ (sym (starExpansiveʳ x)) ⟩
  x + (1# + x * x ⋆)       ≈⟨ +-congʳ (sym (*-identityʳ x)) ⟩
  x * 1# + (1# + x * x ⋆)  ≈⟨ sym (+-assoc (x * 1#) 1# (x * x ⋆)) ⟩
  x * 1# + 1# + x * x ⋆    ≈⟨ +-congʳ (+-comm (x * 1#) 1#) ⟩
  1# + x * 1# + x * x ⋆    ≈⟨ +-assoc 1# (x * 1#) (x * x ⋆) ⟩
  1# + (x * 1# + x * x ⋆)  ≈⟨ +-congˡ (sym (distribˡ x 1# ((x ⋆)))) ⟩
  1# + x * (1# + x ⋆)      ≈⟨ +-congˡ (*-congˡ (1+x⋆≈x⋆ x)) ⟩
  1# + x * x ⋆             ≈⟨ (starExpansiveʳ x) ⟩
  x ⋆                      ∎
\end{minted}
\item
\begin{minted}[breaklines,samepage]{Agda}
1+x+x⋆≈x⋆ : ∀ x → 1# + x + x ⋆ ≈ x ⋆
1+x+x⋆≈x⋆ x = begin
  1# + x + x ⋆    ≈⟨ +-assoc 1# x (x ⋆) ⟩
  1# + (x + x ⋆)  ≈⟨ +-congˡ (x+x⋆≈x⋆ x) ⟩
  1# + x ⋆        ≈⟨ 1+x⋆≈x⋆ x ⟩
  x ⋆             ∎
\end{minted}
\item
\begin{minted}[breaklines,samepage]{Agda}
0+x+x⋆≈x⋆ : ∀ x → 0# + x + x ⋆ ≈ x ⋆
0+x+x⋆≈x⋆ x = begin
  0# + x + x ⋆    ≈⟨ +-assoc 0# x (x ⋆) ⟩
  0# + (x + x ⋆)  ≈⟨ +-identityˡ ((x + x ⋆)) ⟩
  (x + x ⋆)       ≈⟨ x+x⋆≈x⋆ x ⟩
  x ⋆             ∎
\end{minted}
\item
\begin{minted}[breaklines,samepage]{Agda}
x⋆x⋆≈x⋆ : ∀ x → x ⋆ * x ⋆ ≈ x ⋆
x⋆x⋆≈x⋆ x = starDestructiveˡ x (x ⋆) (x ⋆) (x⋆+xx⋆≈x⋆ x)
\end{minted}
\item
\begin{minted}[breaklines,samepage]{Agda}
1+x⋆x⋆≈x⋆ : ∀ x → 1# + x ⋆ * x ⋆ ≈ x ⋆
1+x⋆x⋆≈x⋆ x = begin
  1# + x ⋆ * x ⋆  ≈⟨ +-congˡ (x⋆x⋆≈x⋆ x) ⟩
  1# + x ⋆        ≈⟨ 1+x⋆≈x⋆ x ⟩
  x ⋆             ∎

x⋆⋆≈x⋆ : ∀ x → (x ⋆) ⋆ ≈ x ⋆
x⋆⋆≈x⋆ x = begin
  (x ⋆) ⋆        ≈⟨ sym (*-identityʳ ((x ⋆) ⋆)) ⟩
  (x ⋆) ⋆ * 1#   ≈⟨ starDestructiveˡ (x ⋆) 1# (x ⋆) (1+x⋆x⋆≈x⋆ x) ⟩
  x ⋆            ∎
\end{minted}
\item
\begin{minted}[breaklines,samepage]{Agda}
x≈y⇒1+xy⋆≈y⋆ : ∀ x y → x ≈  y → 1# + x * y ⋆ ≈ y ⋆
x≈y⇒1+xy⋆≈y⋆ x y eq = begin
  1# + x * y ⋆  ≈⟨ +-congˡ (*-congʳ (eq)) ⟩
  1# + y * y ⋆  ≈⟨ starExpansiveʳ y ⟩
  y ⋆           ∎

x≈y⇒x⋆≈y⋆ : ∀ x y → x ≈ y → x ⋆ ≈ y ⋆
x≈y⇒x⋆≈y⋆ x y eq = begin
  x ⋆       ≈⟨ sym (*-identityʳ (x ⋆)) ⟩
  x ⋆ * 1#  ≈⟨ (starDestructiveˡ x 1# (y ⋆) (x≈y⇒1+xy⋆≈y⋆ x y eq)) ⟩
  y ⋆       ∎
\end{minted}
\item
\begin{minted}[breaklines,samepage]{Agda}
ax≈xb⇒x+axb⋆≈xb⋆ : ∀ x a b → 
	a * x ≈ x * b → x + a * (x * b ⋆) ≈ x * b ⋆
ax≈xb⇒x+axb⋆≈xb⋆ x a b eq = begin
  x + a * (x * b ⋆)       ≈⟨ +-congˡ (sym(*-assoc a x (b ⋆))) ⟩
  x + a * x * b ⋆         ≈⟨ +-congʳ (sym (*-identityʳ x)) ⟩
  x * 1# + a * x * b ⋆    ≈⟨ +-congˡ (*-congʳ (eq)) ⟩
  x * 1# + x * b * b ⋆    ≈⟨ +-congˡ (*-assoc x b (b ⋆) ) ⟩
  x * 1# + x * (b * b ⋆)  ≈⟨ sym (distribˡ x 1# (b * b ⋆)) ⟩
  x * (1# + b * b ⋆)      ≈⟨ *-congˡ (starExpansiveʳ b) ⟩
  x * b ⋆                 ∎

ax≈xb⇒a⋆x≈xb⋆ : ∀ x a b → a * x ≈ x * b → a ⋆ * x ≈ x * b ⋆
ax≈xb⇒a⋆x≈xb⋆ x a b eq = 
	starDestructiveˡ a x ((x * b ⋆)) (ax≈xb⇒x+axb⋆≈xb⋆ x a b eq)
\end{minted}
\item
\begin{minted}[breaklines,samepage]{Agda}
[xy]⋆x≈x[yx]⋆ : ∀ x y → (x * y) ⋆ * x ≈ x * (y * x) ⋆
[xy]⋆x≈x[yx]⋆ x y = ax≈xb⇒a⋆x≈xb⋆ x (x * y) (y * x) (*-assoc x y x)
\end{minted}
\end{enumerate}