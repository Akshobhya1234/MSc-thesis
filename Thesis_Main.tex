\documentclass[12pt]{report} % Times New Roman, 12pt
%\usepackage{gscale_thesis_singlespace} % Single spaced thesis
\usepackage{style/gscale_thesis_singlespace} % Double spaced thesis
\usepackage{style/fancyheadings} % Header and footer styling
\usepackage{setspace} % Allows double spacing but skips headers/footers
\setcounter{tocdepth}{1} % Limits the TOC to chapter and section names

% Additional packages
\usepackage{graphicx} % Allows the inclusion of figures
\usepackage{subcaption} % Allows captions to be added to subfigures
\usepackage[justification=centering]{caption} % Centres caption text
\usepackage{array} % Used for table formatting
\newcolumntype{P}[1]{>{\raggedright\let\newline\\\arraybackslash\hspace{0pt}}m{#1}}
\usepackage{booktabs} % Fancy-style tables
\usepackage{longtable} % Allows for tables that are more than one page long
\usepackage{float} % Better figure placement control
\usepackage{enumerate} % Numbered lists
\usepackage[shortlabels]{enumitem} % For controlling enumerate labels
\usepackage[shortcuts]{extdash} % Allows manual hyphenation of hypenated words
\usepackage{amsmath} % Non-standard math symbols
\usepackage{amsfonts} % Extended fonts for mathematics

\usepackage[hidelinks]{hyperref} % Linking to LaTeX labels and external URLs
\usepackage{newunicodechar}
\usepackage{fourier}
\usepackage{comment}

% For nice captions and floating environments, such as for my code snippets
\usepackage{caption}

\usepackage[newfloat,outputdir=build]{minted}
% Credits to Arash Esbati (https://tex.stackexchange.com/a/254177) for the
% listings-related component of minted usage.

\usepackage[most]{tcolorbox}

%textmarker style from colorbox doc
\tcbset{textmarker/.style={%
        enhanced,
        parbox=false,boxrule=0mm,boxsep=0mm,arc=0mm,
        outer arc=0mm,left=6mm,right=3mm,top=7pt,bottom=7pt,
        toptitle=1mm,bottomtitle=1mm,oversize}}


% define new colorboxes
\newtcolorbox{hintBox}{textmarker,
    borderline west={6pt}{0pt}{yellow},
    colback=yellow!10!white}
\newtcolorbox{importantBox}{textmarker,
    borderline west={6pt}{0pt}{red},
    colback=red!10!white}
\newtcolorbox{noteBox}{textmarker,
    borderline west={6pt}{0pt}{green},
    colback=green!10!white}

% define commands for easy access
\newcommand{\note}[1]{\begin{noteBox} \textbf{Note:} #1 \end{noteBox}}
\newcommand{\warning}[1]{\begin{hintBox} \textbf{Warning:} #1 \end{hintBox}}
\newcommand{\important}[1]{\begin{importantBox} \textbf{Important:} #1 \end{importantBox}}

\usemintedstyle{colorful}
\newcommand{\inline}[1]{\mintinline{agda}|#1|}
\newcommand{\inlineCode}[2]{\mintinline{#1}|#2|}


\newunicodechar{λ}{\ensuremath{\mathnormal\lambda}}
\newunicodechar{∙}{\ensuremath{\mathnormal\cdot}}
\newunicodechar{⊔}{\ensuremath{\mathnormal\sqcup}}
\newunicodechar{ℓ}{\ensuremath{\mathnormal\ell}}
\newunicodechar{≈}{\ensuremath{\mathnormal\approx}}
\newunicodechar{←}{\ensuremath{\mathnormal\from}}
\newunicodechar{→}{\ensuremath{\mathnormal\to}}
\newunicodechar{ε}{\ensuremath{\mathnormal\epsilon}}
\newunicodechar{∀}{\ensuremath{\mathnormal\forall}}
\newunicodechar{⟦}{\ensuremath{\mathnormal\llbracket}}
\newunicodechar{⟧}{\ensuremath{\mathnormal\rrbracket}}
\newunicodechar{∘}{\ensuremath{\mathnormal\circ}}
\newunicodechar{∎}{\ensuremath{\mathnormal\blacksquare}}
\newunicodechar{⋆}{\ensuremath{\mathnormal\star}}
\newunicodechar{₂}{\ensuremath{\textsubscript{2}}}
\newunicodechar{₁}{\ensuremath{\textsubscript{1}}}
\newunicodechar{ʳ}{\ensuremath{\textsuperscript{r}}}
\newunicodechar{ˡ}{\ensuremath{\textsuperscript{l}}}
\newunicodechar{⇒}{\ensuremath{\mathnormal\Rightarrow}}
\newunicodechar{⟺}{\ensuremath{\mathnormal\Longleftrightarrow}}
\newunicodechar{≉}{\ensuremath{\mathnormal\not\approx}}	
\newunicodechar{⁻}{\ensuremath{\textsuperscript{-}}}	
\newunicodechar{₃}{\ensuremath{\textsubscript{3}}}
\newunicodechar{‿}{\ensuremath{\mathnormal\smile}}
\newunicodechar{∼}{\ensuremath{\mathnormal\sim}}
\newunicodechar{×}{\ensuremath{\mathnormal\times}}
\newunicodechar{¹}{\ensuremath{\textsuperscript{1}}}
\newunicodechar{ₙ}{\ensuremath{\textsubscript{n}}}
\newunicodechar{≡}{\ensuremath{\mathnormal\equiv}}
\newunicodechar{⟨}{\ensuremath{\mathnormal\langle}}
\newunicodechar{⟩}{\ensuremath{\mathnormal\rangle}}
\newunicodechar{◦}{\ensuremath{\mathnormal\circ}}
\newunicodechar{₀}{\ensuremath{\textsubscript{0}}}

\numberwithin{equation}{section} % Numbers equations based on their section

% ********************************
\begin{document}
\title{Algebraic Structures in Proof Assistant Systems}
\halftitle{Algebraic Structures in Proof Assistant Systems} % 60 Characters Max. Including Spaces

\author{Akshobhya Katte Madhusudana}
\shortauthor{Akshobhya K M} % Used for page header

\dept{Department of Computing and Software}
\field{Computer Science} % What field your thesis is in (e.g. Software Engineering)

\prevdegreeone{B.Eng. (Computer Science and Engineering),\\ Bangalore University, Bangalore, India}
\prevdegreetwo{B.Eng.} % Just your degree's field

\submitdate{April 2023} % Use the month's full spelling e.g. November
\copyrightyear{2023} % Year you are submitting this, usually your graduation 
%year

\doctype{Thesis} % ``Report'' or ``Thesis'' or whatever you need
\degree{Masters of Science} % The degree you get when you submit this
\degreeabbrv{M.Sc.}
\principaladviser{Dr. Jacques Carette} % Your Supervisor
 % LaTeX variables for preface pages/headers
    
\beforepreface % Half title page, title page, declaration page   
  \prefacesection{Abstract}
Building a standard library of mathematical knowledge for a proof system is a
complex task that relies on human effort. By conducting a survey on the standard library
of four proof systems (Agda, Idris, Lean, and Coq), we define the scope for our
research to study types of algebraic structures in proof systems. From the
result of the survey, we establish our focus to contribute to the Agda standard
library. 

Universal algebra studies structures by abstracting out the specific definitions
and properties of algebraic structures. Providing an extensive and
well-defined library of algebraic structures and theorems in Agda will
enable researchers to explore new domains and build upon existing definitions
(and theorems). We explore capturing a select subset of algebraic structures
such as quasigroups, loops, semigroups, rings, and Kleene algebra with some of
their constructs. Constructs like morphisms and direct products are given to us
by universal algebra which provides a way to relate different structures in a
systematic and rigorous way. Morphisms allow us to understand how different
structures are related.

During our exploration of capturing these structures in Agda, we encountered
several issues. We categorized these issues into five classes and analyzed each
problem to provide plausible solutions. As part
of this research, we define more than 20 algebraic structures and add more than
40 proofs to the Agda standard library % Abstract
  %\thispagestyle{empty}
\null\vfill
\begin{center}
%\textbf{Dedications}
%\linebreak
\textsl{To all my teachers \\ You are my greatest blessing}
\end{center}
\vfill
 % Dedication
  \prefacesection{Acknowledgements}

I would like to thank Dr.\  Jacques Carette for the guidance, encouragement, contributions and support he has provided during my studies as his student. Your expertise and feedback were invaluable in shaping my research. I have been very lucky to have you as my supervisor and I have learned a lot from you. I would also like to thank Dr.\  Ridha Khedri. Your course on algebraic structures in software engineering really motivated me to pursue my research in this field. Many thanks to Dr. Wolfarm Khal for your advice and feedback. Your project RATH-AGDA was one of the best resources I had for my research.\\

I would like to express my sincere gratitude to all the maintainers and contributors of Agda standard library. Your code review was very helpful for critical thinking and pushed me to understand the subject better. \\

Thanks to my parents Shanthala and Madhusudana, my sister Utpala and my twin Anagha for their endless love and support of me while I continue my education. Finally, Thanks to my family and friends for all the motivation and support. % Acknowledgements
  \referencepages % Table of Contents, List of Figures, List of Tables
  \academicstatement{academicachievementdeclaration}
\afterpreface
        
  \include{Introduction}                  
        \setcounter{figure}{0}
        \setcounter{equation}{0}
        \setcounter{table}{0}

      \chapter{Background Universal Algebra}

In the recent years, universal algebra has seen an exponential growth in its study of theories and applications \cite{sankappanavar1981course}. Universal algebra is the study of algebraic structures and its properties. Algebraic structure contains a set A called the carrier set and some operations and axioms such that the operation satisfies the axioms. \\ \\
Formal definition for algebra is given in \cite{sankappanavar1981course} as \\ "For A a nonempty set and n a nonnegative integer we define A\textsuperscript{0} = \{\(\emptyset\)\} , and, for n > 0, A\textsuperscript{n} is the set of n-tuples of elements from A. An n-ary operation (or function) on A is any function f from A\textsuperscript{n} to A; n is the arity (or rank) of f. A finitary operation is an n-ary operation, for some n. The image of <a\textsubscript{1}, . . ., a\textsubscript{n}> under an n-ary operation f is denoted by f(a\textsubscript{1}, . . ., a\textsubscript{n}). An operation f on A is called a nullary operation (or constant) if its arity is zero; it is completely determined by the image f(\(\emptyset\)) in A of the only element \(\emptyset\) in A\textsuperscript{0}, and as such it is convenient to identify it with the element f(\(\emptyset\)). Thus a nullary operation is thought of as an element of A. An operation f on A is unary, binary, or ternary if its arity is 1,2, or 3, respectively"\\ \\
Group structure is one of the early structures studied in universal algebra. A group G is an algebra with one nullary, one unary and one binary operation represented as (G, ∙, \textsuperscript{-1}, 1) which satisfy the following axioms. \\
Associativity - ∀ x y z \(\in\) G, x ∙ (y ∙ z) ≈ (x ∙ y) ∙ z)\\
Identity - ∀ x \(\in\) G, x ∙ 1 ≈ 1 ∙ x ≈ x\\
Inverse - ∀ x \(\in\) G, x ∙ x \textsuperscript{-1} ≈  x \textsuperscript{-1} ∙ x ≈ 1\\
Where ≈ is the equivalence relation defined later in the chapter.

\section{Relation and function}
In this section a brief overview of relation and function is discussed\\
The Cartesian product between two sets X and Y,  X \(\times\) Y is defined as \{(x,y) : x \(\in\) X and y \(\in\) Y \} \\
A binary relation is a subset of the Cartesian product of two sets that is a mapping between one set called domain to the other set called the codomain. A binary relation R on the set X to Y is denoted as an ordered pair (x,y) or xRy and element x in X and y in Y. \\
A reflexive relation R on set X is a subset on X \(\times\) X is defined as R : \{(x,x) : x \(\in\) X \} and can be denoted as xRx \\
A symmetric relation R on set X is a subset of X \(\times\) X is defined as R : \(\forall x y \in \) X: xRy ⟺ yRx \\
A relation R is said to be transitive on set X, is a subset of X \(\times\) X if ∀ x y z  \(\in\) X if (x,y) in R and (y,z) in R then (x,z) is in R\\
A relation R is equivalence if it is reflexive, symmetric and transitive. \\
If in a relation, if every element in domain is mapped to only one element in the codomain, then we call it a function.\\
A function f is injective if f maps distinct elements of domain to distinct elements of codomain.
Image of the function is the set of all elements in codomain that is reachable from the function f in its domain.\\
A function is called surjective if the image of the function is same as its codomain.\\
A function is called bijective if it is both injective and surjective.

\section{Universe and type}
According to Russel's paradox \cite{russell2020principles} the collection of all set is not a set. The naive set theory defines a set as well defined collection of objects. The paradox defines the set of all sets that are not the member of themselves. This develops to two kinds of contradiction. \cite{russelPara}\\
1. If the set contains itself, then it should not be a member of itself by definition\\
2. If the set does not contain itself the it is not a member of itself.\\
For this reason, in Agda not every type is a set and the set type can be defined using keyword Set\textsubscript{1}. "A type whose elements are types are called universe" \cite{universeagda}. This primitive type is useful to define and prove theorems about functions that operate on large set.\\
\\
The type of an algebra is also called the language of the algebra is a set of function symbols. Each member of this set is assigned a positive number that is the arity of the member. For example an algebra of type(2,0) denotes an algebra with one binary operation and one nullary operation. The group structure defined in previous section is of type (2,1,0). That is ∙ is a binary operation, \textsuperscript{-1} is a unary operation and 1 is the nullary operation. 

\section{Congruence and Morphism}
For an algebra A of type F, congruence relation \(\theta\) on A is defined using compatibility property that states that for each n-ary function symbol f \(\in\) F and x\textsubscript{i} , y\textsubscript{i} \(\in\) A, If x\textsubscript{i} \(\theta\) y\textsubscript{i}  holds for \(1\leq i \leq n\) then f\textsuperscript{A}(x\textsubscript{1},...,x\textsubscript{n})\(\theta\)f\textsuperscript{A}(y\textsubscript{1},....,y\textsubscript{n}) holds. \cite{sankappanavar1981course}
\\
                  
        \setcounter{figure}{0}
        \setcounter{equation}{0}
        \setcounter{table}{0}

      \chapter{Background Agda}
Agda is a functional programming language developed by Ulf Norell. Agda is used as a proof assistant system that supports higher order logic with dependently typed pattern matching that is used to prove theorems about inductive types. 

\section{Agda Notation}
In this section we discuss the programming notations in Agda that will be used in the later chapters of the thesis.
A → B denotes a function from set A to set B\\
a : A denotes a is an element in A \\
f : A → B denotes that f is a some map from set A to set B\\
Op₁ denotes the unary operation on the set.
\{a : A\} denotes the implicit argument a in A. 
\begin{Verbatim}
Op₁ : ∀ {ℓ} → Set ℓ → Set ℓ
Op₁ A = A → A
\end{Verbatim}
\textunderscore ∙ denotes unary operation ∙ and underscore denotes the placement of the argument. \\
Op₂ denotes binary operation on the set
\begin{Verbatim}
Op₂ : ∀ {ℓ} → Set ℓ → Set ℓ
Op₂ A = A → A → A
\end{Verbatim}
level (or universe) is a primitive type in Agda. \\
a ⊔ ℓ denotes the least upper bound of the levels a and ℓ.
\textunderscore ∙\textunderscore  denotes binary operation ∙ . Underscore denotes the placement of the arguments.\\

\section{Structure definition}
Algebraic structures are defined as record types in Agda. Record types are defined using the keyword record. Records types are used to group values together and they provide named fields to generalise dependent product types. The structures are obtained by wrapping the predicates that are expressed as "is-a" relation. ~\citep{hu2021formalizing}
\begin{Verbatim}
record IsMagma (∙ : Op₂ A) : Set (a ⊔ ℓ) where
  field
    isEquivalence : IsEquivalence _≈_
    ∙-cong        : Congruent₂ ∙

  open IsEquivalence isEquivalence public
\end{Verbatim}
In the above example structure IsMagma is defined as a record type with fields isEquivalence and ∙-cong. ∙ is a binary operation on set A. a ⊔ ℓ is the least upper bound for the set. \textunderscore  ≈ \textunderscore is the binary operation argument for IsEquivalence. If a relation P on set A is equivalent to relation Q on set B, then we say f preserves p for some map f from set A to B. Congruent₂ ∙ represents that the binary operation ∙ preserves equivalence relation. IsEquivalence and Congruent₂ are predicates defined in standard library.\\
We open the module isEquivalence to be able to use it in defining other structures in the algebra hierarchy. The open statement is made public using the keyword public to be able to re-export the names from another module.\\
Morphisms of the structure are defined as record type in Agda standard library.\\
Agda standart library defines the bundled version of the structures that contains the operations of the structures, sets and axioms. 
\begin{Verbatim}
record Magma c ℓ : Set (suc (c ⊔ ℓ)) where
  infixl 7 _∙_
  infix  4 _≈_
  field
    Carrier : Set c
    _≈_     : Rel Carrier ℓ
    _∙_     : Op₂ Carrier
    isMagma : IsMagma _≈_ _∙_

  open IsMagma isMagma public

  rawMagma : RawMagma _ _
  rawMagma = record { _≈_ = _≈_; _∙_ = _∙_ }

  open RawMagma rawMagma public
    using (_≉_)
\end{Verbatim}
Above is the bundled version of IsMagma structure. RawMagma is the raw version of the magma with only the operators and set. infix<l,r> denotes the fixity and precedence of the operator. using keyword is used to export only the fields that are mentioned in its arguments. \\
When exporting the modules we may need to rename the fields to avoid having duplicate names.
\begin{Verbatim}
  open IsMagma *-isMagma public
    using ()
    renaming
    ( ∙-congˡ  to *-congˡ
    ; ∙-congʳ  to *-congʳ
    )
\end{Verbatim} 
Keyword renaming is used to rename the fields. In the above sample code, we rename ∙-congˡ  to *-congˡ and ∙-congʳ  to *-congʳ.

\subsection{Proofs in Agda}
A setoid is a set with equivalence relation. In Agda beginning of the proof is given using "begin". IsRelatedTo is a type defined to infer arguments even if the underlying equality evaluates. 
\begin{Verbatim}
begin_ : ∀ {x y} → x IsRelatedTo y → x ∼ y
begin relTo x∼y = x∼y
\end{Verbatim}
standard step to relation is defined as step-∼
\begin{Verbatim}
step-∼ : ∀ x {y z} → y IsRelatedTo z → x ∼ y → x IsRelatedTo z
step-∼ _ (relTo y∼z) x∼y = relTo (trans x∼y y∼z)
\end{Verbatim}
step using equality is given as
\begin{Verbatim}
step-≈ = Base.step-∼
syntax step-≈ x y≈z x≈y = x ≈⟨ x≈y ⟩ y≈z
\end{Verbatim}
The termination of the proof is given using \textunderscore ∎
\begin{Verbatim}
_∎ : ∀ x → x IsRelatedTo x
x ∎ = relTo refl
\end{Verbatim}
Agda supports quantifiers. Universal quantifier is denoted as \(\forall\) and existential quantifier is denoted as \(\exists\) 

Below is the example to the proposition x ∙ (y ∙ z) = (x ∙ y) ∙ z  for a semigroup i.e., a Magma with associative property (x ∙ (y ∙ z) = (x ∙ y) ∙ z) 
\begin{Verbatim}
x∙yz≈xy∙z : ∀ x y z → x ∙ (y ∙ z) ≈ (x ∙ y) ∙ z
x∙yz≈xy∙z x y z = begin 
  x ∙ (y ∙ z) ≈⟨ sym (assoc x y z) ⟩ 
  (x ∙ y) ∙ z ∎
\end{Verbatim}
In the proposition x y z are in set S that is a semigroup. "sym (assoc x y z)" is the reasoning for the proof.

                  
        \setcounter{figure}{0}
        \setcounter{equation}{0}
        \setcounter{table}{0}

  \include{algebrasurvey}                  
        \setcounter{figure}{0}
        \setcounter{equation}{0}
        \setcounter{table}{0}

  \include{quasigrouploop}                  
        \setcounter{figure}{0}
        \setcounter{equation}{0}
        \setcounter{table}{0}

  \include{semigroupring}                  
        \setcounter{figure}{0}
        \setcounter{equation}{0}
        \setcounter{table}{0}

  \include{kleenealgebra}                  
        \setcounter{figure}{0}
        \setcounter{equation}{0}
        \setcounter{table}{0}

  \include{Probleminprogramalgebra}                  
       \setcounter{figure}{0}
       \setcounter{equation}{0}
       \setcounter{table}{0}

  \chapter{Conclusion And Future Work}
The primary of this work was to study types of algebraic structures in proof
assistant systems. To define the scope of the work, we do a survey on the
coverage of types of algebraic structures in four proof assistant systems which
are Agda, Idris, Coq, and Lean. The thesis shows how to define a structure with
some of its constructs and properties in Agda. We divided this into three main
chapters based on the closeness of structures that is quasigroup and loop,
semigroup and ring, and Kleene algebra. We then analyzed five problems that arise
when defining types of algebraic structures in proof systems.

In section ~\ref{contribution}, we summarize the contributions of this work and
how it refers to the research outline described in Chapter 1. Section
~\ref{future} discuss some extensions or future work of this work. 

\section{Summary Of Contributions}
\label{contribution}
Universal algebra is a well-studied and evolving branch of mathematics. Proof
systems are useful in automated reasoning and becoming popular in research and
applications more than ever. With an introduction to universal algebra in
Chapter 1 and Agda in Chapter 2, and Chapter 3 provide an overview of the
quantitative use of algebraic structures in proof assistant systems. We create a
clickable table that takes to the definition of structures in the standard
libraries of the systems studied (Agda, Idris, Lean, and Coq).

This leads to defining the scope of contribution to the Agda standard library.
Chapter 5 is dedicated to studying the structures quasigroup, loop, and their
variations. Chapter 6 provides an overview of semigroup and ring structures with
definitions of their constructs and prove their properties. Chapter 7 is
dedicated to the study of Kleene algebra and its properties in Agda. Along with
these structures, we define structures unital magma, invertible magma,
invertible unital magma, idempotent magma, alternate magma, flexible magma,
semimedial magma, medial magma, with their constructs.

Our approach to defining these structures led us to encounter and analyze some
problems such as ambiguity in naming, equivalent and identical structures.
Chapter 8 discussed how these problems become more evident in proof systems that
might be ignored in classical the 'pen-and-paper' technique.

\section{Future Work}
\label{future}
Our work can be extended in different ways. The direct products defined in this
thesis do not clearly differentiate between direct products, products, and
co-products of algebraic structures. There is currently a discussion on the Agda
standard library to overcome this issue, but the changes are yet to come. The
current solution adapted in the Agda standard library to remove the redundant
field will only remove the equivalence. However, there can be other redundant
fields. For example, in commutative monoid, right identity can be obtained from
left identity and commutativity. These problems are yet to be addressed. The
current work will rely on human efforts in building strong libraries in the
field of abstract algebra. A more robust and reliable generative library will be
helpful to reduce human efforts. 
        \setcounter{figure}{0}
        \setcounter{equation}{0}
        \setcounter{table}{0}

\begin{appendix}

\end{appendix}

\ignorecitefornumbering
% The bibliography is set up to allow for multiple bib files
\bibliographystyle{ieeetran}
\nocite{*}
\bibliography{references}

\label{NumDocumentPages}

\end{document}
% ********************************
