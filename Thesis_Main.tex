\documentclass[12pt]{report} % Times New Roman, 12pt
%\usepackage{gscale_thesis_singlespace} % Single spaced thesis
\usepackage{style/gscale_thesis_doublespace} % Double spaced thesis
\usepackage{style/fancyheadings} % Header and footer styling
\usepackage{setspace} % Allows double spacing but skips headers/footers
\setcounter{tocdepth}{1} % Limits the TOC to chapter and section names
\usepackage{breakcites}
\usepackage{microtype}
% Additional packages
\usepackage{graphicx} % Allows the inclusion of figures
\usepackage{subcaption} % Allows captions to be added to subfigures
\usepackage[justification=centering]{caption} % Centres caption text
\usepackage{array} % Used for table formatting
\newcolumntype{P}[1]{>{\raggedright\let\newline\\\arraybackslash\hspace{0pt}}m{#1}}
\usepackage{booktabs} % Fancy-style tables
\usepackage{longtable} % Allows for tables that are more than one page long
\usepackage{float} % Better figure placement control
\usepackage{enumerate} % Numbered lists
\usepackage[shortlabels]{enumitem} % For controlling enumerate labels
\usepackage[shortcuts]{extdash} % Allows manual hyphenation of hypenated words
\usepackage{amsmath} % Non-standard math symbols
\usepackage{amsfonts} % Extended fonts for mathematics

\usepackage[hidelinks]{hyperref} % Linking to LaTeX labels and external URLs
\usepackage{newunicodechar}
\usepackage{fourier}
\usepackage{comment}
\usepackage{mathrsfs}
% For nice captions and floating environments, such as for my code snippets
\usepackage{caption}

\usepackage[newfloat,outputdir=build]{minted}
% Credits to Arash Esbati (https://tex.stackexchange.com/a/254177) for the
% listings-related component of minted usage.
\usepackage[most]{tcolorbox}

\usepackage[colorlinks = true,
            linkcolor = blue,
            urlcolor  = blue,
            citecolor = blue,
            anchorcolor = blue]{hyperref}

\newcommand{\MYhref}[3][blue]{\href{#2}{\color{#1}{#3}}}%

\AtBeginEnvironment{minted}{\singlespacing}

%textmarker style from colorbox doc
\tcbset{textmarker/.style={%
        enhanced,
        parbox=false,boxrule=0mm,boxsep=0mm,arc=0mm,
        outer arc=0mm,left=6mm,right=3mm,top=7pt,bottom=7pt,
        toptitle=1mm,bottomtitle=1mm,oversize}}


% define new colorboxes
\newtcolorbox{hintBox}{textmarker,
    borderline west={6pt}{0pt}{yellow},
    colback=yellow!10!white}
\newtcolorbox{importantBox}{textmarker,
    borderline west={6pt}{0pt}{red},
    colback=red!10!white}
\newtcolorbox{noteBox}{textmarker,
    borderline west={6pt}{0pt}{green},
    colback=green!10!white}

% define commands for easy access
\newcommand{\note}[1]{\begin{noteBox} \textbf{Note:} #1 \end{noteBox}}
\newcommand{\important}[1]{\begin{importantBox} \textbf{Important:} #1 \end{importantBox}}

\usemintedstyle{colorful}
\newcommand{\inline}[1]{\mintinline{agda}|#1|}
\newcommand{\inlineCode}[2]{\mintinline{#1}|#2|}


\newunicodechar{λ}{\ensuremath{\mathnormal\lambda}}
\newunicodechar{∙}{\ensuremath{\mathnormal\cdot}}
\newunicodechar{⊔}{\ensuremath{\mathnormal\sqcup}}
\newunicodechar{ℓ}{\ensuremath{\mathnormal\ell}}
\newunicodechar{≈}{\ensuremath{\mathnormal\approx}}
\newunicodechar{←}{\ensuremath{\mathnormal\from}}
\newunicodechar{→}{\ensuremath{\mathnormal\to}}
\newunicodechar{ε}{\ensuremath{\mathnormal\epsilon}}
\newunicodechar{∀}{\ensuremath{\mathnormal\forall}}
\newunicodechar{⟦}{\ensuremath{\mathnormal\llbracket}}
\newunicodechar{⟧}{\ensuremath{\mathnormal\rrbracket}}
\newunicodechar{∘}{\ensuremath{\mathnormal\circ}}
\newunicodechar{∎}{\ensuremath{\mathnormal\blacksquare}}
\newunicodechar{⋆}{\ensuremath{\mathnormal\star}}
\newunicodechar{₂}{\ensuremath{\textsubscript{2}}}
\newunicodechar{₁}{\ensuremath{\textsubscript{1}}}
\newunicodechar{ʳ}{\ensuremath{\textsuperscript{r}}}
\newunicodechar{ˡ}{\ensuremath{\textsuperscript{l}}}
\newunicodechar{⇒}{\ensuremath{\mathnormal\Rightarrow}}
\newunicodechar{⟺}{\ensuremath{\mathnormal\Longleftrightarrow}}
\newunicodechar{≉}{\ensuremath{\mathnormal\not\approx}}	
\newunicodechar{⁻}{\ensuremath{\textsuperscript{-}}}	
\newunicodechar{₃}{\ensuremath{\textsubscript{3}}}
\newunicodechar{‿}{\ensuremath{\mathnormal\smile}}
\newunicodechar{∼}{\ensuremath{\mathnormal\sim}}
\newunicodechar{×}{\ensuremath{\mathnormal\times}}
\newunicodechar{¹}{\ensuremath{\textsuperscript{1}}}
\newunicodechar{ₙ}{\ensuremath{\textsubscript{n}}}
\newunicodechar{≡}{\ensuremath{\mathnormal\equiv}}
\newunicodechar{⟨}{\ensuremath{\mathnormal\langle}}
\newunicodechar{⟩}{\ensuremath{\mathnormal\rangle}}
\newunicodechar{◦}{\ensuremath{\mathnormal\circ}}
\newunicodechar{₀}{\ensuremath{\textsubscript{0}}}
\newunicodechar{∃}{\ensuremath{\mathnormal\exists}}
\newunicodechar{Σ}{\ensuremath{\mathnormal\Sigma}}
\newunicodechar{¬}{\ensuremath{\mathnormal\neg}}

\numberwithin{equation}{section} % Numbers equations based on their section

% ********************************
\begin{document}
\title{Algebraic Structures in Proof Assistant Systems}
\halftitle{Algebraic Structures in Proof Assistant Systems} % 60 Characters Max. Including Spaces

\author{Akshobhya Katte Madhusudana}
\shortauthor{Akshobhya K M} % Used for page header

\dept{Department of Computing and Software}
\field{Computer Science} % What field your thesis is in (e.g. Software Engineering)

\prevdegreeone{B.Eng. (Computer Science and Engineering),\\ Bangalore University, Bangalore, India}
\prevdegreetwo{B.Eng.} % Just your degree's field

\submitdate{April 2023} % Use the month's full spelling e.g. November
\copyrightyear{2023} % Year you are submitting this, usually your graduation 
%year

\doctype{Thesis} % ``Report'' or ``Thesis'' or whatever you need
\degree{Masters of Science} % The degree you get when you submit this
\degreeabbrv{M.Sc.}
\principaladviser{Dr. Jacques Carette} % Your Supervisor
 % LaTeX variables for preface pages/headers
    
\beforepreface % Half title page, title page, declaration page   
  \prefacesection{Abstract}
Algebra is the abstract encapsulation of mathematical intuition. Proof systems
can be described as an inference system that has provable statements or theorems
being their final products. It is important to study the intersection of these
powerful concepts in mathematics and computer science by carefully defining
mathematical concepts in computer language.

In this work, we study the types of algebraic structures in proof systems
especially Agda, Coq, Idris, and Lean 3 to determine the coverage of algebra in
these systems and to set the scope of our research. We contribute to the Agda
standard library, a proof assistant system, so it can be extended to other
relevant fields of algebra. We focus on commonly studied structures such as
quasigroups, loops, semigroups, rings, and Kleene algebra. These structures are
well-studied in universal algebra and have applications in various fields
including computer science, quantum physics, and mathematics. In the effort of
studying several structures with their constructs like morphisms, and direct
products and proving their properties, we analyze five problems that arise and
may not be as relevant in classical mathematics. We define more than 20
algebraic structures and add more than 40 proofs to Agda standard library. % Abstract
  %\thispagestyle{empty}
\null\vfill
\begin{center}
%\textbf{Dedications}
%\linebreak
\textsl{Your Dedication \\ Optional second line}
\end{center}
\vfill
 % Dedication
  \prefacesection{Acknowledgements}

Acknowledgements go here. % Acknowledgements
  \referencepages % Table of Contents, List of Figures, List of Tables
  \academicstatement{academicachievementdeclaration}
\afterpreface
        
  \include{Introduction}                  
        \setcounter{figure}{0}
        \setcounter{equation}{0}
        \setcounter{table}{0}

      \chapter{Universal Algebra: An Overview}
By the early 18th century, mathematicians had discovered how to solve polynomial
equation of up-to degree 4. While trying to find a general solution for
polynomial equation, Lagrange and Abel established permutation groups.
Mathematician Gauss developed modular arithmetic from number theory. These
sources including theory of algebraic structures and geometry became the main
source for \textit{group theory}. It was mathematician Evariste Galois who
coined the term \emph{group} and established group theory. He used group to
determine the solvability of polynomial equations. Group theory was later
discovered to be useful in other fields of mathematics such as modulus theory
and geometry\cite{enwiki:1107380309}. Knowing the usefulness of this tool,
mathematicians abstracted out the axioms of the group into a general tool. Thus
evolved the structure group that we know today. As group theory, the study of
group structures evolved, other abstract structures were invented to solve
problems. This gave rise to a new field in mathematics called \emph{abstract
algebra}. Abstract algebra is the study of algebraic structure and its models or
examples \cite{enwiki:1107380309}. An algebraic structure is a tuple containing
a 'carrier' set, A, a set of operations that act on A, and a set of axioms
involving the operations and A. Some mathematicians were only interested in
studying the structures themselves that is the arbitrary interpretation of the
language and not the models that includes a theory that holds in the structure.
Universal algebra is the study of algebraic structures and its properties. In
the recent years, \emph{universal algebra} has seen an exponential growth in its
study of theories and applications \cite{sankappanavar1981course}. 

Algebraic structures, like monoids, loops, groups and rings have similar
properties. Universal algebra studies these structures by abstracting out the
specific definitions and properties of algebraic structures. Universal algebra
will deal with these algebraic structures as axiomatic theories in equational
first-order logic \cite{YSharoda}.

\section{Relation and function}
In order to understand algebraic structures, it is essential to know some basics
of relations and functions. In this section, we define relations and functions.
We can start with defining a set.
\begin{itemize}
\item 
A set is a well-defined collection of objects. The elements or members of the
set can be mathematical object of any kind such as numbers, symbols, geometrical
shapes, or even other sets. If $x$ is an object in set $S$ then we say x is an
element of S and is denoted as $x\inS$.

\item The \emph{Cartesian product} between two sets $X$ and $Y$,  $X \ \times\
Y$ is defined as a pair ${(x,y) :\ x \ \in\ X$,\ $y\ \in\ Y }$. 

\item
A \emph{binary relation} is a subset of the Cartesian product of two sets that
is a mapping between one set called \textit{domain} to the other set called the
\textit{codomain}. A relation assigns elements of domain to some elements in
the codomain. A binary relation $R$ on the set $X$ to $Y$ is denoted as an
ordered pair $(x,y)$ or $xRy$ and element $x$ in $X$ and $y$ in $Y$.

\item 
When talking about a set, we discuss about how different elements of the set can
be related in some way. For example, in set of integers $Z$, we may say that
some $s,y \in Z$ are related if $x-y$ is divisible by 2. In other way, $x$ and
$y$ are related only if they are both odd or both even. This idea of expressing
same relation in different way can be formalized as \textit{equivalence
relation}. A relation $R$ is equivalence if it satisfies:
\begin{enumerate}
    \item Reflexive: A \emph{reflexive relation} $R$ on set $X$ is a
subset on \(X \times X\)  is defined as \( R : \{(x,x) : x \in X\}\) and can be
denoted as $xRx$

    \item Symmetric: A \emph{symmetric relation} R on set X is a subset of \(X
\times X\) is defined as \(R: \forall x y \in X: xRy ⟺ yRx\)

    \item Transitive: A relation R is said to be \emph{transitive} on set X, is
a subset of \(X \times X\) such that \(∀ x y z \in X \) if \((x,y) \in R\) and
\((y,z) \in R \) then \((x,z) \in R\)
\end{enumerate}
In other words, a relation R is \emph{equivalence} if it is reflexive, symmetric and transitive.

\item
If in a relation, if every element in domain is mapped to only one element in
the codomain, then we call it a \emph{function}. In other words, function is a
map $f$ from set $X$ (domain) to $Y$ (codomain) is a rule that assigns each
element of $X$ to a unique element in $Y$. This can be expressed using the notation:
\[f:\ X \rightarrow\ Y\]
\[x \ \rightarrow\ f(x)\]
For example, on set of natural numbers $N$ to $N$ we can define a function as:
\[f:\ N \rightarrow N\]
\[x \ \rightarrow\ x^{2}\]   
\item Let $X$ and $Y$ be two sets, and $f:X\rightarrowY$ be the function then:
\begin{enumerate}
    \item The function $f$ is \textit{identity} if $X=Y$ and $f(x)\ =\ x,\ \forall x\ \in
    \ Y$. The identity function can be denoted as $f=Id_s$.
    \item A function f is \emph{injective} if f maps distinct elements of domain to
    distinct elements of codomain. $f$ is injective if $f(x)=f(y) \Rightarrow x = y \forallx,y \in X.$
    \item A function is called \emph{surjective} if given $y \in Y$, there
    exists $x\in X$ such that $f(x) = y$.
    \item A function is called \emph{bijective} if it is both injective and surjective.
\end{enumerate}
\item
An \emph{operation} is defined as a function that can take zero or more inputs
and maps it to a well-defined output value. The number of operands is the arity
of the operation.
\end{itemize}

\section{Universe, type and signature}
The naive set theory defines a set as well-defined collection of objects. If a
set is defined using unrestricted comprehensive
principle\cite{enwiki:1125383109}, then it leads to contradiction. This was
first discovered by mathematician Bertrand Russell, and the paradox is called
Russell's paradox. The paradox defines the set of all sets that are not the
member of themselves \cite{russelPara}. This develops to two kinds of
contradiction:
\begin{enumerate}
\item If the set contains itself, then it should not be a member of itself by
definition
\item If the set does not contain itself then it is not a member of itself.
\end{enumerate}

The signature of an algebraic structure can be defined as a collection of
relation and operations with their arity on the set of an algebraic structure. A
structure with \(\Omega\) signature is called as \(\Omega\) algebra.

A Formal definition of algebra is given in \cite{sankappanavar1981course} as:
For $A$ a nonempty set and $n$ a non-negative integer we define $A_0$ =
\{\(\emptyset\)\}, and, for $n > 0$, $A_n$ is the set of n-tuples of elements
from $A$. An n-ary operation (or function) on $A$ is any function $f$ from $A_n$
to $A$; $n$ is the arity (or rank) of $f$. A finitary operation is an n-ary
operation, for some n. The image of $<a_1,a_2,...,a_n>$ under an n-ary operation
$f$ is denoted by $f(a_1,a_2,...,a_n)$. An operation $f$ on $A$ is called a
nullary operation (or constant) if its arity is zero; it is completely
determined by the image $f(\emptyset)$ in $A$ of the only element \(\emptyset\)
in $A_0$, and as such it is convenient to identify it with the element
$f(\emptyset)$. Thus, a nullary operation is thought of as an element of $A$. An
operation $f$ on $A$ is unary, binary, or ternary if its arity is 1,2, or 3,
respectively.

For example, a group G is an algebra with one nullary (1), one unary
(\textsuperscript{-1}) and one binary (∙) operation represented as $(G, ∙,
\textsuperscript{-1}, 1)$ which satisfy the following axioms. 
\begin{enumerate}
\item Associativity - \( ∀ x y z \in G, x ∙ (y ∙ z) ≈ (x ∙ y) ∙ z \)
\item Identity - \(∀ x \in G, x ∙ 1 ≈ 1 ∙ x ≈ x\)
\item Inverse - \( ∀ x \in G, x ∙ x \textsuperscript{-1} ≈  x
\textsuperscript{-1} ∙ x ≈ 1\)
\end{enumerate}
Where ≈ is the equivalence relation.

The type (or language) of the algebra is a set of function symbols. Each member
of this set is assigned a positive number that is the arity of the member. For
example an algebra of type (2,0) denotes an algebra with one binary operation
and one nullary operation. The group structure defined in previous section is of
type (2,1,0). That is ∙ is a binary operation, \textsuperscript{-1} is a unary
operation and 1 is the nullary operation.

\section{Constructions}
Universal algebra provides definitions of constructions related to algebraic
structures. In this section, we will describe some of these constructions. 
\begin{itemize}
    \item The \textit{congruence} relation for an algebraic structure can be
    defined as an equivalence relation that is compatible with the structure
    such that the operations are well-defined on the equivalence class. A more
    formal definition is for an algebra $A$ of type $F$ is given as, congruence
    relation \(\theta\) on $A$ is defined using compatibility property that
    states that for each n-ary function symbol $f \ \in\ F$ and $x_i,\ y_i\ \in\
    A$, If $x_i\ \theta\ y_i$ holds for \(1\leq i \leq n\) then
    $f^{A}(x_1,...,x_n)\ \theta\ f^{A}(y_1,....,y_n)$ holds
    \cite{sankappanavar1981course}.

    For example, consider group structure $(G, ∙, \textsuperscript{-1}, 1)$. A
    congruence relation on $G$ with binary operation $∙$ is an equivalence
    relation $\equiv$ on $G$ such that, \[g_1\equiv g_2\ \text{and}\ h_1 \equiv h_2
    \Rightarrow g_1 ∙ h_1 \equiv g_2 ∙ h_2\]

    \item A \textit{morphism} is a structure preserving map between two
    algebraic structures. It is an abstraction that generalizes the map between
    two structures or mathematical objects in general. If $A$ and $B$ are two
    algebras of same type $F$, then a homomorphism is defined as a mapping
    \(\alpha\) from algebra $A$ to $B$ such that: \[ \alpha
    f^{A}(a_1,a_2,....,a_n)\ =\ f^{B}(\alpha a_1,\alpha a_2,....,\alpha a_n)\]
    For each n-ary $f$ in $F$ and each sequence $a_1,a_2,....,a_n$ from $A$.
    
    As an example, consider $G_1 = {1,-1,i,-i}$, which is a group under
    multiplication, and $G_2$ = group of all integers under addition. A mapping
    $f$ from $G_1$ to $G_2$ such that $f(x) = i^{n} \ \forall n \in G_2$ is a
    homomorphism.
    
    In \cite{sankappanavar1981course}, the author proves that if \(\alpha: A
    \rightarrow B\) and \(\beta: A \rightarrow B\) are homomorphism on algebra
    $A$ to $B$ such that \(\alpha (a) = \beta (a) \) then \(\alpha\) = \(\beta\)

    Some variants of homomorphism are:
    \begin{enumerate}
        \item  Monomorphism: For two algebras $A$ and $B$, if \(\alpha : A
        \rightarrow B \) is a homomorphism from $A$ to $B$, and if \(\alpha\)
        satisfies one-to-one mapping (i.e., \(\alpha\) is injective) then the
        morphism \(\alpha\) is called a \textit{monomorphism}.

        \item Isomorphism: For two algebras $A$ and $B$, if \(\alpha : A \rightarrow B \)  is
        a monomorphism from $A$ to $B$, and if \(\alpha\) is a bijection from
        $A$ to $B$, then \(\alpha\) is called an \textit{isomorphism}.  

        \item Endomorphism: A homomorphism from an algebra $A$ to itself is
        called \textit{endomorphism}. In other words, if $f$ is a homomorphism on $A$
        such that $f:A\rightarrow A$ then, f is endomorphism.

        \item Automorphism: An isomorphism from an algebra $A$ to itself is
        called \textit{automorphism}.

        \item Epimorphism: For two algebras $A$ and $B$, if \(\alpha : A
        \rightarrow B \) is a homomorphism from $A$ to $B$, and if \(\alpha\) is
        surjective then the morphism \(\alpha\) is called a
        \textit{epimorphism}.
    \end{enumerate}

    \item For algebras $A$, $B$, and $C$ the \textit{composition of morphisms} $f:\ A \
    \rightarrow \ B$ and $g:\ B \rightarrow\ C$ is denoted by the function $g\
    \circ\ f\ :\ A\ \rightarrow \C$ and is defined as $(g\ \circ\ f)\ a = \ g(f\
    a) \ \forall\ a\ \in\ A$. In \cite{sankappanavar1981course}, the author
    proves that the composite of two homomorphism (monomorphism/isomorphism) is
    also a homomorphism (monomorphism/isomorphism).

    \item The \textit{direct product} between two algebras $A$ and $B$ is
    defined as \[A\ \cross\ B\ =\ {(a,b)\ |\ a \ \in \ A\ b \ \in\ B}\] If
    $x,y\in A$ and $u,v \in B$, there is a natural binary operation on $A \cross
    B$ such that: $(a,u)\cdot(b,v):=(ab,uv)$. 

\end{itemize}
                  
        \setcounter{figure}{0}
        \setcounter{equation}{0}
        \setcounter{table}{0}

      \chapter{Agda}
Agda is a dependently typed programming language based on unified theory of
dependent types \cite{enwiki:1127496533} and is an extension of Martin-Löf type
theory. Dependent programming allows programmers to define types that depend on
values, to write functions that utilize these types, and to prove the
correctness of the program in the same language. In other words, dependent type
theory allow types to depend on values and expressions. Agda has been used in
various applications such as formal verification, program synthesis, theorem
proving, and automated reasoning. It is also used by researchers and academician
to teach and explore the concepts of functional programming, type theory, and
formal methods.

Agda is also a poroof assistant system. Agda is designed to help programmers to
write and verify correct and efficient programs by allowing them to express
their intentions in a precise and formal way. One of the key features of Agda is
its support for interactive, and constructive programming. Interactive programming allows
the programmer to incrementally develop and refine their code, by testing and
verifying each intermediate step. Constructive programming ensures that every
expression and function in the language has a well-defined meaning and
computation rules, which makes it easier to reason about their behavior and
correctness. This chapter provides a brief overview of programming in Agda in
the context of algebraic structures
. 

\section{Types and functions}
Agda is based on a core language that provides a minimal set of primitives and
types, and is extended with libraries and modules that define more complex data
structures, algorithms, and abstractions. Agda's type system allows for the
definition of new types and operations that are tailored to the specific needs
of a particular application or domain. Agda supports inductive types, simple
types, and parameterized types \cite{10.1007/978-3-642-03359-9_6}. A data type
in Agda can be declared using the keyword \inline{data}. Let us consider an
example of inductive datatype to define natural numbers \inline{Nat}.
\label{code:Nat}
\begin{minted}[breaklines,samepage]{Agda}
data Nat : Set where
  zero : Nat
  suc  : Nat -> Nat
\end{minted}

An inductive datatype is a datatype that is defined in terms of itself. In the
code snippet \ref{code:Nat}, \inline{Nat} is an inductive type defined with base
constant \inline{zero} and an inductive data constructor \inline{suc}.
\inline{zero} and \inline{suc} are constructors, where \inline{suc} has a
parameter (\inline{Nat}) and \inline{zero} has no parameters. In this example,
the smallest element is \inline{zero}. It is important to note that Agda is a
total language, i.e., each program in Agda will terminate, and all possible
patterns will be matched\cite{kidney2020finiteness}. Another way of defining a
type is using the keyword \inline{record}. Record type helps to put values
together, and the values are tuples of values of specified type. A record type
can be defined by referencing other types and creating a synonym. An example of
record type is discussed later in the chapter when we define algebraic
structure.

Those familiar with Haskell will find Agda to be somewhat familiar. For example,
functions have a very similar syntax to those in Haskell. A function in Agda is
defined by declaring the type followed by the clauses \cite{agdaFunction}. 
\begin{minted}[breaklines,samepage]{Agda}
f : (x₁ : A₁) → ... → (xₙ : Aₙ) → B
f p₁ ... pₙ = d
...
f q₁ ... qₙ = e
\end{minted} 
Where \inline{f} is the function identified, \inline{p} and \inline{q} are the
patterns of type \inline{A}. \inline{d} and \inline{e} are expressions. There
are other ways to define a function such as using dot patterns, absurd patterns,
as patterns and case trees \cite{agdaFunction}. In Agda, a function to and from
each type is provided if there is a bijection between two types.

For example, we can define addition on natural numbers as a recursive function:
\label{code:Add}
\begin{minted}[breaklines,samepage]{Agda}
_+_ : Nat -> Nat -> Nat
zero + m = m
suc n + m = suc (n + m)
\end{minted}

In the above example, function \inline{_+_} takes two arguments of type
\inline{Nat} and returns a value that is sum of the two arguments of type
\inline{Nat}. To guarantee that the program always terminate, a recursive call
in must be made on a structurally smaller argument. For the function
\inline{_+_} above, the first argument \inline{n} is smaller in the recursive
call \inline{suc n}. This ensures that the function \inline{_+_} always
terminates.

\section{Structure definition}
Let us first understand how unary and binary operations are defined in Agda standard
library. Below code shows how unary operation \inline{Op₁} and binary operation
\inline{Op₂} are defined:
\begin{minted}[breaklines,samepage]{Agda}
Op₁ : ∀ {ℓ} → Set ℓ → Set ℓ
Op₁ A = A → A
\end{minted}
\begin{minted}[breaklines,samepage]{Agda}
Op₂ : ∀ {ℓ} → Set ℓ → Set ℓ
Op₂ A = A → A → A
\end{minted}

In Agda, not every type belongs to \inline{Set}. Every type belongs somewhere in
the hierarchy \inline{Set₀}, \inline{Set₁}, \inline{Set₂}, and so on.
\inline{Set} abbreviates \inline{Set₀}, and \inline{Set₀ : Set₁}, \inline{Set₁ :
Set₂}, and so  \cite{plfa22.08}. This definition works if we are comparing two
values of some type in \inline{Set}. But, we cannot compare two values that
belong to \inline{Set ℓ} for some arbitrary \inline{ℓ}. To solve this problem,
Agda provides type \inline{Level}. This type helps us to define equality
generalized to an arbitrary level.

An algebraic structure can be defined in Agda using the record keyword, which is
used to define a new data type along with its properties. The structures are
obtained by wrapping the predicates that are expressed as "is-a" relation
\cite{hu2021formalizing}. The following example shows how to define a magma
structure in Agda:

\begin{minted}[breaklines,samepage]{Agda}
record IsMagma (∙ : Op₂ A) : Set (a ⊔ ℓ) where
  field
    isEquivalence : IsEquivalence _≈_
    ∙-cong        : Congruent₂ ∙

  open IsEquivalence isEquivalence public
\end{minted}
In the above example structure \inline{IsMagma} is defined as a record type with
a parameter \inline{Op₂ A}. The properties of the structure \inline{IsMagma} are
declared as the fields of the record, which include equivalence
\inline{isEquivalence} and congruence \inline{∙-cong}. \inline{∙} is a binary
operation on the set \inline{A}. \inline{a ⊔ ℓ} is the least upper bound for the
set. \inline{_≈_} is the binary operation argument for \inline{IsEquivalence}.

If a relation P on set A is equivalent to relation Q on set B, then we say f
preserves p for some map f from set A to B. \inline{Congruent₂ ∙} represents
that the binary operation ∙ preserves equivalence relation.
\inline{IsEquivalence} and \inline{Congruent₂} are predicates defined in
standard library. We open the module \inline{isEquivalence} to bring its
definition into scope. The open statement is made public using the keyword
\inline{public} to be able to re-export the names from another module.

The bundled version of the structures contains the operations of the structures,
sets and axioms. 
\begin{minted}[breaklines,samepage]{Agda}
record Magma c ℓ : Set (suc (c ⊔ ℓ)) where
  infixl 7 _∙_
  infix  4 _≈_
  field
    Carrier : Set c
    _≈_     : Rel Carrier ℓ
    _∙_     : Op₂ Carrier
    isMagma : IsMagma _≈_ _∙_

  open IsMagma isMagma public

  rawMagma : RawMagma _ _
  rawMagma = record { _≈_ = _≈_; _∙_ = _∙_ }

  open RawMagma rawMagma public
    using (_≉_)
\end{minted}
Above is the bundled version of \inline{IsMagma} structure. \inline{RawMagma} is
the raw version of the magma with only the operators and set. infix<l,r> denotes
the fixity and precedence of the operator. The operator with higher fixity binds
more strongly than an operator with a lower numeric value. \inline{using}
keyword is used to limit the imported components. When exporting the modules, we
may need to rename the fields to avoid having ambiguity. Keyword
\inline{renaming} is used to rename the fields.
\label{code:rename}
\begin{minted}[breaklines,samepage]{Agda}
  open IsMagma *-isMagma public
    using ()
    renaming
    ( ∙-congˡ  to *-congˡ
    ; ∙-congʳ  to *-congʳ
    )
\end{minted} 
In the sample code ~\ref{code:rename}, we rename \inline{∙-congˡ}  to \inline{*-congˡ}
and \inline{∙-congʳ}  to \inline{*-congʳ} thus avoiding conflict with same
elements exported by other module.

\section{Equational Proofs in Agda}
In constructive mathematics, knowledge comes with implicit arguments.
Constructive proofs use the existence of a mathematical object is given by
giving a way to create the method. \cite{enwiki:1090644431}. An equational proof
is a sequence of steps that transform one expression into another using a set of
rules. Writing proofs in Agda follows a syntax called dependent types, which
allows us to declare properties of functions and data types that need to be
verified by the compiler. \cite{kidney2020finiteness}.  

In the previous section, we have seen how to define natural number and addition
function on it. Now, we will write an inductive proof using pattern matching
that states that the addition of two natural numbers is commutative.

\begin{minted}[breaklines,samepage]{Agda}
comm : ∀ (m n : Nat) → m + n ≡ n + m
comm zero zero = refl
comm zero (suc n) = cong suc (comm zero n)
comm (suc m) n = cong suc (comm m n)
\end{minted}

In the above example, the proof \inline{comm zero zero} represents commutative
property where both \inline{m} and \inline{n} are \inline{zero}. The
\inline{refl} function is used to prove that two expressions are equal using the
reflexivity of equality. \inline{comm zero suc n} and \inline{suc m + n} are
reduced recursively until the base case is reached. The \inline{cong} function
is used to apply the inductive hypothesis to the successive \inline{suc}
constructors. This is just a simple example of proof, but Agda allows us to
express and verify more complex properties, such as type soundness, termination,
and correctness of algorithms.

In algebraic structure, consider the example to the proposition of the associative property x ∙ (y ∙ z)
= (x ∙ y) ∙ z  for a semigroup i.e., a Magma with associative property (x ∙ (y ∙
z) = (x ∙ y) ∙ z). The proof can be written in Agda as:
\begin{minted}[breaklines,samepage]{Agda}
x∙yz≈xy∙z : ∀ x y z → x ∙ (y ∙ z) ≈ (x ∙ y) ∙ z
x∙yz≈xy∙z x y z = begin 
  x ∙ (y ∙ z) ≈⟨ sym (assoc x y z) ⟩ 
  (x ∙ y) ∙ z ∎
\end{minted}
To make proofs more readable, people have tried to emulate textual proofs, for
example, by creating "begin" and "end" syntax. \inline{begin} indicates the start
of the proof. \inline{begin} is a function that relates two objects.
\begin{minted}{Agda}
begin_ : ∀ {x y} → x IsRelatedTo y → x ∼ y
begin relTo x∼y = x∼y
\end{minted}
\inline{IsRelatedTo} is a type defined to infer arguments even if the underlying equality
evaluates. Standard step to relation is defined as \inline{step-∼}.
\begin{minted}[breaklines,samepage]{Agda}
step-∼ : ∀ x {y z} → y IsRelatedTo z → x ∼ y → x IsRelatedTo z
step-∼ _ (relTo y∼z) x∼y = relTo (trans x∼y y∼z)
\end{minted}
similarly, step using equality is given as
\begin{minted}[breaklines,samepage]{Agda}
step-≈ = Base.step-∼
syntax step-≈ x y≈z x≈y = x ≈⟨ x≈y ⟩ y≈z
\end{minted}
The termination (i.e., QED) of the proof is given using \inline{_∎} that relates object to itself.
\begin{minted}[breaklines,samepage]{Agda}
_∎ : ∀ x → x IsRelatedTo x
x ∎ = relTo refl
\end{minted}
Agda supports quantifiers. Universal quantifier is denoted as \(\forall\) and
existential quantifier is denoted as \(\exists\).

                  
        \setcounter{figure}{0}
        \setcounter{equation}{0}
        \setcounter{table}{0}

  \include{algebrasurvey}                  
        \setcounter{figure}{0}
        \setcounter{equation}{0}
        \setcounter{table}{0}

  \include{quasigrouploop}                  
        \setcounter{figure}{0}
        \setcounter{equation}{0}
        \setcounter{table}{0}

  \include{semigroupring}                  
        \setcounter{figure}{0}
        \setcounter{equation}{0}
        \setcounter{table}{0}

  \include{kleenealgebra}                  
        \setcounter{figure}{0}
        \setcounter{equation}{0}
        \setcounter{table}{0}

  \include{Probleminprogramalgebra}                  
       \setcounter{figure}{0}
       \setcounter{equation}{0}
       \setcounter{table}{0}

  \chapter{Conclusion and Future Work}
The main of this work is to study algebraic structures in proof assistant systems. To define the scope the work we do a survey on coverage of algebraic on four proof assistant systems that are agda, idris, coq and lean 3.The thesis shows how to define a structure with some of its constructs and properties in agda. We divide this into three main chapters based on closeness of structures that is quasigroup and loop, semigroup and ring, and kleene algebra. We then analyze five problems that arises when defining algebraic structures in proof systems and give a brief overview of how it can be tackled with product family algebra. \\

In section ~\ref{contribution} we summarize the contributions of this work and how it refers to the research outline described in Chapter 1. Section ~\ref{future} discuss some extensions or future work of this work. 

\section{Summary of contributions}
\label{contribution}
Universal algebra is a well studied and evolving branch of mathematics. Proof systems are useful in automated reasoning and becoming popular in research and applications more than ever. Chapter 1 provides a overview of quantitative use of algebraic structures in proof assistant systems. We create a 'clickable' table that takes to the definition of structures in the standard libraries of the systems studied (agda, idris, lean and coq). \\

This leads to define the scope of contribution to agda standard library. Chapter 5 is dedicated to study the structures quasigroup, loop and their variations. Chapter 6 provides an overview of semigroup and ring structures with their properties and morphisms. Chapter 7 is dedicated to study of kleene algebra and it's properties in agda. Along with these structures, we define structures unital magma, invertible magma, invertible unital magma, idempotent magma, alternate magma, flexible magma, semimedial magma, medial magma, with their direct products and morphisms.\\

Our approach of defining these structures led us to analyze the problems such as ambiguity in naming, equivalent and identical structures. Chapter 8 discuss how these problems becomes more evident in proof systems that might be ignored in classical 'pen-and-paper' technique. We give an overview of how product family algebra can be used to represent and tackle these problems.

\section{Future work}
\label{future}
Our work can be extended in different ways and agda standard library is evolving with many contributions. The direct products defined in this thesis do not clearly differentiate between direct products and products and co-products. There is currently discussion on agda standard library to overcome this issue but the changes are yet to come. Product family algebra is a powerful tool to solve many problems in ontologies, cryptography and other fields. Only a brief overview of how this tool can be used is discussed in Chapter 8. A more detailed study with implementation is required to concretely say to what extent the discussed problems can be solved. This work will rely on human efforts in building strong libraries in field of abstract algebra. A more robust and reliable generative library will be helpful to reduce human efforts. 



        \setcounter{figure}{0}
        \setcounter{equation}{0}
        \setcounter{table}{0}

\begin{appendix}

\end{appendix}


% The bibliography is set up to allow for multiple bib files
\bibliographystyle{ACM-Reference-Format}
\nocite{*}
\bibliography{references}

\label{NumDocumentPages}

\end{document}
% ********************************
