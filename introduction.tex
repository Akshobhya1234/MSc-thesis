\chapter{Introduction}
Abstract algebra is the study of algebraic structure that came into existence in
the early nineteenth century as complex problems and solutions evolved in other
branches of mathematics such as geometry, number theory, and polynomial
equations. Being a relatively new subject in mathematics, algebraic structures
are used in various fields. For example, \cite{liaqat2021some} use semigroups
\footnotetext{A kind of algebraic structure} to study time dependent partial
differential equations in a technique similar to differential equations on a
function space. Kleene algebra and semigroup structures are used in finite
automata to better model and understand the finite state machines.
\textit{Groups} are one of the oldest structures that are used in number theory,
atomic and molecular theory, and cryptography \cite{enwiki:1133598242}.
\cite{bruck1944some} uses \textit{Quasigroups} and \textit{loops} structures for
the encryption of image data. A magma is a set $S$ with a binary operation $∙$
such that, $\forall x,y \in S \Rightarrow (x ∙ y) \in S$. A magma with
associativity is called a semigroup. A Magma with division operation is called a
quasigroup. Figure \ref{fig_magma} shows the algebra hierarchy from magma to
group. 
 \begin{figure}[ht]
	\centering
	\includegraphics[width=0.7\textwidth]{figures/Sample/Magma_to_group.jpg}
	\caption{Algebraic structure hierarchy \cite{enwiki:1107380309}}
	\label{fig_magma}
 \end{figure}

With the growing help of technology, mathematicians are more indulged in
automated reasoning. Increasing powers of computers and software tools that help
towards automated reasoning become useful in their research. Although the proof
systems that support first-order logic are successful, developing a tool that
supports higher order logic is complex \cite{phillips2010automated} and requires
carefully defining mathematical objects and concepts. Proof assistant systems act
as a bridge between computer intelligence and human effort in developing
mathematical proofs. Agda, Coq, Isabelle, Lean, and Idris are some commonly used
proof assistant systems. Mathematicians use these proof assistants to check
their proof for validity, build proofs and sometimes even generate them via
proof search tools. For the scope of the thesis, we only discuss types of
algebraic structures in proof systems.

\section{Research Outline}
For any software system to be robust, all its dependencies must similarly be
robust. The standard libraries of these systems should support the user with all
necessary functionalities to be able to use the system easily without having to
define all functionalities. To generate robust libraries of knowledge, the
authors Jacques Carette, Russell O'Connor, and Yasmine Sharoda, in their paper
\cite{BuildingDiamond} explore techniques to generate libraries with minimum
human effort. However, while their methods do work in theory, they are difficult
(and expensive) in practice. Although generated libraries can define the
algebraic concepts required, they are not fully reliable and hence not
considered as "standard library" for any proof system. For now, building
standard libraries for proof systems relies on human efforts. This led to the
question of what is the current scope of algebraic structures in proof assistant
systems. A survey was conducted to better understand the coverage of algebra in
four proof systems Agda, Idris, Lean, and Coq. Agda was one such system where
there was better scope to contribute to the standard library. \footnote{I was
exposed to Agda during course work for my Master's degree, further adding bias
to choosing Agda over other systems} 

As part of this thesis, more than twenty-three types of structures have been
defined in the standard library for Agda. Inspired by the ways algebraic
structures are used in research, in this work we explore capturing a select
subset of them in Agda standard library. Following the algebra hierarchy in
Figure ~\ref{fig_magma}, we study magma with division operation that is
quasigroup and loop structures. We also explore various types of loop such as
bol-loop and moufang-loop and their properties. Semigroups are used in various
fields such as probability theory and formal systems. One of the most commonly
studied algebraic structure is Ring. In this thesis we study types of rings such
as near-ring, quasi-ring, and non-associative ring. Along with ring structure,
the most used structure is Kleene algebra. The applications of Kleene algebra
are seen in finite state machines, regular expressions and other branches of
computer science. As part of this thesis, we study Kleene algebra by providing
proofs for its properties that may be used in developing other systems or
applications. By contributing to Agda standard library, we hope that this work
will be used by others. 

Notably, as we explore capturing these structures in Agda, we analyze five
problems that arise:
\begin{enumerate}
\item Ambiguity in naming structures.
\item Equivalent structures that are structurally different.
\item Redundant field during structural inheritance.
\item Identical structures that can be derived in many ways in algebra hierarchy
\item Equivalent structures that are structurally the same.
\end{enumerate}

\section{Thesis Outline}
Chapters 2 and 3 focus on the background information necessary for reading this
work, focusing on reviewing universal algebra and algebraic structures in Agda,
respectively. Chapter 4 justifies the scope of the thesis contribution through a
survey on algebraic coverage in proof systems. The next three chapters 5,6 and 7
are dedicated to discussing the structures in detail. Chapter 5 explores
quasigroup and loop structures that use division operation. Chapter 6 discusses
the properties of semigroup and ring with variations of the ring structure. Chapter
7 explores Kleene algebra, definition, construct and properties in Agda. Chapter
8 describes the various problems we faced during this work, as well as advice on
handling common issues in programming algebras in proof systems. Finally,
Chapter 9 concludes this work with notes on related future works and some
closing thoughts.
