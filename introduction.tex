\chapter{Introduction}
Abstract algebra is the study of algebraic structure that came into existence in
the early nineteenth century as complex problems and solutions evolved in other
branches of mathematics such as geometry, number theory, and polynomial
equations. With the growing help of technology, mathematicians are more indulged
in automated reasoning. Increasing powers of computers and software tools that
help automated reasoning become useful in their research. Although the
proof systems that support first-order logic are successful, developing a tool
that supports higher order logic is complex and requires carefully defining
mathematical objects and concepts \cite{phillips2010automated}. Proof assistant
systems act as a bridge between computer intelligence and human effort in
developing mathematical proofs. Agda, Coq, Isabelle, Lean, and Idris are some
commonly used proof assistant systems. Mathematicians use these proof assistants
to check their proof for validity, build proofs and sometimes even generate them
via proof search tools. For the scope of the thesis, we only discuss types of
algebraic structures in proof systems.

For any software system to be robust, all its dependencies must similarly be
robust. The standard libraries of these systems should support the user with
necessary functionalities to be able to use the system easily without having to
define all functionalities. The paper \cite{BuildingDiamond} explores techniques
to generate libraries with minimum human effort. Although generated libraries
can define algebraic concepts, they are not considered as "standard library"
for any proof system. For now, building standard libraries for proof systems
relies on human efforts. This led to the question of what is the current scope
of algebraic structures in the standard libraries of proof assistant systems. A
survey of the coverage of algebraic structures in the standard libraries of
proof assistant systems can help us understand which algebraic structures are
already supported by various proof assistants, and which structures are still
missing. This information can help researchers identify gaps in existing proof
assistants and guide future development. A survey was conducted to better
understand the coverage of algebra in four proof systems Agda, Idris, Lean, and
Coq. Agda was one such system where there was better scope to contribute to the
standard library.

Agda is used by mathematicians and computer scientists for research purposes.
Contributing certain algebraic structures and theorems to Agda would help
researchers to explore new domains by building upon the existing definitions and
theorems easily. The Agda standard library follows an algebra hierarchy that
starts with magma as the initial structure from which other structures are
defined. A magma is a set $S$ with a binary operation $∙$ such that, $\forall
x,y \in S, (x ∙ y) \in S$. A magma with associativity is called a semigroup. A
magma with division operation is called a quasigroup.

The definitions of constructs like homomorphism and direct product is given to
us by universal algebra. Universal algebra provides a common framework by
abstracting out the specific definitions and properties of algebraic structures.
It helps us to study the commonalities of algebraic structures and define their
constructs. An algebra in universal algebra is defined as an ordered pair
$(S,F)$ where $S$ is a set and $F = (F_i:i\in I)$ is a finitary operations on A
for some indexing set $I$ \cite{sannella2012foundations}. Certain constructs
like morphisms and direct products help us to relate different mathematical
objects and structures in a systematic and rigorous way. Morphisms
allow us to understand how different algebraic structures are related to one
another. Direct product, on the other hand, is a useful tool for combining
structures, such as monoids, groups, or rings, to create new and more
complex structures that retain many of the desirable properties of the original
structures. This allows us to study and understand larger, more complex systems
and their properties.

\section{Research Outline}
To define the scope of our research, that is to study algebraic structures in
proof assistant systems, we capture the current coverage of algebraic structures
in the standard libraries of some commonly used proof assistant systems. As part
of the survey, we consider four libraries: The Agda standard library (v1.7.1),
the mathematical component library (1.12.0) for Coq, Idris 2, and mathematical
library for Lean 3. In the effort of finding the coverage of algebraic
structures in these libraries, we develop a clickable table that directs to the
definition of the structure in the source code of these systems. Through the
survey, we establish our focus for contributing to the Agda standard
library\footnote{I was exposed to Agda during coursework for my Master's degree,
further adding bias to choosing Agda over other systems}.

Inspired by the ways algebraic structures are used in research, in this work we
explore capturing a select subset of them in the Agda standard library. We study
magma with division operation that is quasigroup and loop structures. By
defining them with their morphisms and direct product constructs, we can study
their properties and relationships in a more systematic way. We also explore
various types of loops such as bol-loop and moufang-loop and their properties.
Semigroups are used in various fields such as probability theory and formal
systems. One of the most commonly studied algebraic structure is Ring. In this
thesis, we study types of rings such as near-ring, quasi-ring, and
non-associative ring. I was exposed to Kleene Algebra in discrete mathematics
course. Inspired by the applications of Kleene algebra in finite state machines,
regular expressions, and other branches of computer science, we study Kleene
algebra by providing proof for its properties that may be used in developing
other systems or applications. By contributing to Agda standard library, we hope
that this work will be used by others. 

As we explore capturing these structures in Agda, we encountered several
problems. In this work, we abstract out these problems into five classes:
\begin{enumerate}
\item Ambiguity in naming structures.
\item Equivalent structures that are structurally different.
\item Redundant field during structural inheritance.
\item Identical structures that can be derived in many ways in algebra hierarchy
\item Equivalent structures that are structurally the same.
\end{enumerate}
We analyze each problem and provide plausible solutions except for "Ambiguity in naming structures".

\section{Thesis Outline}
Chapters 2 and 3 focus on the background information necessary for reading this
work, focusing on reviewing universal algebra and algebraic structures in Agda,
respectively. Chapter 4 is a survey on algebraic coverage in proof systems. The
next three chapters 5, 6 and 7 are dedicated to discussing the structures in
detail. Chapter 5 explores quasigroup and loop structures with its variations.
Chapter 6 discusses the properties of semigroup and ring. Chapter 7 explores
Kleene algebra, definition, construct and properties in Agda. Chapter 8
describes the various problems we faced during this work, as well as advice on
handling common issues in programming algebras in proof systems. Finally,
Chapter 9 concludes this work with notes on related future works and some
closing thoughts.
