\chapter{Theory of Semigroup and Ring in Agda}
In early 20th century, mathematician Hilbert proposed the H\textsubscript{10}
problem: does there exist a general approach to verify whether a general
Diophantine equation is solvable\cite{larchey2020hilbert}. Although this problem
was solved by 1970, In 1987 Siekmann and Szabo concluded that the unification
problem of D\textsubscript{A}-rewriting system\cite{DARewriting} cannot be
predicted. In \cite{deng2016characterizations} the authors proposes a type
(2,2,0) algebra that is a \textit{semigroup} that can be used to give a general
construct of D\textsubscript{A}-rewriting system. Semigroup structures are also
used in finite automata systems, probability theory and partial differential
equations are explored in \cite{liaqat2021some}.\\

Similarly, \textit{ring} is an algebraic structure that also have notable
applications such as in number theory \cite{noauthor_undated-eq}, quantum
computing \cite{netto2008influence}, in cryptography \cite{ringcrypt} and many
other fields. Variations of ring structure such as a near-ring, quasi-ring, and
Non-associative ring are being explored to make ring theory (study of ring structures),
more dynamic, concrete and useable. Now, the question arises: how can we encode
these structures in Agda, we will explore this question. The aim of this chapter
is to define these structures and prove some properties in the Agda standard
library that can help build other systems that uses these structures.
\section{Definition}
A magma is an algebraic structure with a set $S$ and a binary operation $∙$ such
that, $\forall x,y \in S \Rightarrow (x ∙ y) \in S$. Following Figure 1.1, we may
observe that we can derive semigroups from magma by adding associative property.
For binary operation \(∙\) on a set S, the associative property is defined as 
\begin{equation}
\forall\ x\ y\ z\ \in\ S :\ x\ ∙\ (y\ ∙\ z)\ =\ (x\ ∙\ y)\ ∙\ z
\end{equation}
A semigroup that satisfies commutative property is called commutative
semigroup. For binary operation \( ∙ \) on a set S, commutative property is
defined as 
\begin{equation}
\forall\ x\ y\ \in\ S :\ x\ ∙\ y\ =\ y\ ∙\ x\
\end{equation}
Conversely, in Agda, we can describe associativity and commutativity as follows:
\begin{minted}[breaklines,samepage]{Agda}
Associative : Op₂ A → Set _
Associative _∙_ = ∀ x y z → ((x ∙ y) ∙ z) ≈ (x ∙ (y ∙ z))

Commutative : Op₂ A → Set _
Commutative _∙_ = ∀ x y → (x ∙ y) ≈ (y ∙ x)
\end{minted}

With this declaration of associativity and commutativity, we may further
restrict the operations used to build a magma to one that is also associative to
make it a semigroup. This we obtain the code below, semigroup that is
structurally derived from magma.\footnote{Semigroup and commutative semigroup
structure definitions with direct product and morphism constructs were
previously defined in Agda standard library and hence will not be discussed in
details in this chapter.}

\begin{minted}[breaklines,samepage]{Agda}
record IsSemigroup (∙ : Op₂ A) : Set (a ⊔ ℓ) where
field
  isMagma : IsMagma ∙
  assoc   : Associative ∙

open IsMagma isMagma public
\end{minted}
In the above definition \inline{IsSemigroup} is a record type with two fields
\inline{isMagma} and \inline{assoc}. $∙$ is a parameter of type Op₂ on set $A$
denotes the binary operation for the semiring. \inline{a ⊔ ℓ} is the least upper
bound for the set. Similarly, commutative semigroup can be derived from
semigroup as:
\begin{minted}[breaklines,samepage]{Agda}
record IsCommutativeSemigroup (∙ : Op₂ A) : Set (a ⊔ ℓ) where
field
  isSemigroup : IsSemigroup ∙
  comm        : Commutative ∙

open IsSemigroup isSemigroup public
\end{minted}

Continuing on, we may encode various ring structures as follows: Non-associative
ring is an algebraic structure with two binary operations addition (+) and
multiplication (*). Addition is an Abelian group that is a group with
commutative property and multiplication is unital magma that is a magma with
identity. A group is a monoid with inverse property and a monoid is a semigroup
with an identity element. A magma is called unital if it has an identity
element. In non-associative ring, multiplication distributes over addition and
it has an annihilating zero. Formally, nonAssociativeRing $(R,+,*,⁻¹,0,1)$
should satisfy the following axioms:
\begin{itemize}
  \item $(R,+,⁻¹,0)$ is an abelianGroup:
   \begin{itemize}
    \item Associativity: $\forall x,y,z \in R: x + (y + z) = (x + y) + z$
    \item Identity: $\forall x \in R: (x + 0) = x = (0 + x)$
    \item Inverse: $\forall x \in R: (x + x⁻¹) = 0 = (x⁻¹ + x)$
  \end{itemize}
  \item $(R,*,1)$ is a unital magma
  \begin{itemize}
    \item Identity: $\forall x,y \in R: (x * 1) = x = (1 * x)$
  \end{itemize}
  \item Multiplication distributes over addition: \(\forall x , y , z \in R: (x * (y + z)) = (x * y) + (x
  * z)\) and \( (x + y) * z = (x * z) + (y * z) \)
  \item Annihilating zero: \(\forall x \in R: (x * 0) = 0 = (0 * x)\)
\end{itemize}
\begin{minted}[breaklines,samepage]{Agda}
record IsNonAssociativeRing (+ * : Op₂ A) (-_ : Op₁ A) (0# 1# : A) : Set (a ⊔ ℓ) where
 field
   +-isAbelianGroup : IsAbelianGroup + 0# -_
   *-cong           : Congruent₂ *
   *-identity       : Identity 1# *
   distrib          : * DistributesOver +
   zero             : Zero 0# *

 open IsAbelianGroup +-isAbelianGroup public
\end{minted}

A quasiring is a type (2,2,0,0) algebraic structure for which both addition and
multiplication is a monoid and multiplication distributes over addition, and has
an annihilating zero. A quasiring $(Q,+,*,0,1)$ should satisfy the following axioms:
\begin{itemize}
  \item $(Q,+,0)$ is a monoid:
  \begin{itemize}
    \item Associativity: $\forall x,y,z \in Q, x + (y + z) = (x + y) + z$
    \item Identity: $\forall x \in Q, (x + 0) = x = (0 + x)$
  \end{itemize}
  \item $Q,*,1$ is a monoid:
  \begin{itemize}
    \item Associativity: $ \forall x,y,z \in Q: x * (y*z)  = (x*y)*z$
    \item Identity: $\forall x \in Q, (x * 1) = x = (1 * x)$
  \end{itemize}
  \item Multiplication distributes over addition: \(\forall x , y , z \in Q, (x * (y + z)) = (x * y) + (x
  * z)\) and \( (x + y) * z = (x * z) + (y * z) \)
  \item Annihilating zero: \(\forall x \in Q, (x * 0) = 0 = (0 * x)\)
\end{itemize}

\begin{minted}[breaklines,samepage]{Agda}
record IsQuasiring (+ * : Op₂ A) (0# 1# : A) : Set (a ⊔ ℓ) where
  field
    +-isMonoid    : IsMonoid + 0#
    *-cong        : Congruent₂ *
    *-assoc       : Associative *
    *-identity    : Identity 1# *
    distrib       : * DistributesOver +
    zero          : Zero 0# *

  open IsMonoid +-isMonoid public
\end{minted}
Note: We don't define quasiring with \inline{*-isMonoid} to remove the redundant
equivalence relation. This is discussed in Chapter 8.

A quasiring with additive inverse is called a nearring. This implies that for
the structure nearring, addition is a group, multiplication is a monoid,
multiplication distributes over addition, and has an annihilating zero.

\begin{minted}[breaklines,samepage]{Agda}
record IsNearring (+ * : Op₂ A) (0# 1# : A) (_⁻¹ : Op₁ A) : Set (a ⊔ ℓ) where
  field
    isQuasiring : IsQuasiring + * 0# 1#
    +-inverse   : Inverse 0# _⁻¹ +
    ⁻¹-cong     : Congruent₁ _⁻¹

  open IsQuasiring isQuasiring public
\end{minted}

Ring without one or rig or ring without unit is an algebraic structure with two
binary operations with a unary and a nullary operations. The binary operation
addition (+) is an Abelian group and the binary operation multiplication (*) is
a semigroup. For RingWithoutOne, multiplication distributes over addition and
has an annihilating zero. A ringWithoutOne $(R,+,*,⁻¹,0)$ should satisfy the
following axiom:
\begin{itemize}
  \item $(R,+,⁻¹,0)$ is an abelianGroup:
   \begin{itemize}
    \item Associativity: $\forall x,y,z \in R, x + (y + z) = (x + y) + z$
    \item Identity: $\forall x \in R, (x + 0) = x = (0 + x)$
    \item Inverse: $\forall x \in R, (x + x⁻¹) = 0 = (x⁻¹ + x)$
  \end{itemize}
  \item $(R,*)$ is a semigroup
  \begin{itemize}
    \item Associativity: $ \forall x,y,z \in R, x * (y*z)  = (x*y)*z$
  \end{itemize}
  \item Multiplication distributes over addition: \(\forall x , y , z \in R, (x * (y + z)) = (x * y) + (x
  * z)\) and \( (x + y) * z = (x * z) + (y * z) \)
  \item Annihilating zero: \(\forall x \in R, (x * 0) = 0 = (0 * x)\)
\end{itemize}

\begin{minted}[breaklines,samepage]{Agda}
record IsRingWithoutOne (+ * : Op₂ A) (-_ : Op₁ A) (0# : A) : Set (a ⊔ ℓ) where
  field
    +-isAbelianGroup : IsAbelianGroup + 0# -_
    *-cong           : Congruent₂ *
    *-assoc          : Associative *
    distrib          : * DistributesOver +
    zero             : Zero 0# *

  open IsAbelianGroup +-isAbelianGroup public
\end{minted}
\section{Morphism} 
A structure preserving map between two structures is called \textit{morphism}.
In this section morphism of RingWithoutOne structure is discussed. Morphisms of
quasiring, nearring can be found in Agda standard library. The homomorphism for
ringWithoutOne structure can be defined using group homomorphism. For two group
structures $(G_1,+_1,)$ and $(G_2,+_2)$, homomorphism $f:G_1 \rightarrow G_2$ is
a structure preserving map such that:
\begin{itemize}
  \item $f$ preserves the binary operation: $f(x +_1 y) = f(x) +_2 f(y)$
  \item $f$ preserves the inverse operation: $f(x⁻¹) = f(x)⁻¹$
  \item $f$ preserves the identity: $f(e_1) = e_2$ where $e_1$ is the identity
  in $G_1$ and $e_2$ is the identity in $G_2$
\end{itemize}
Homomorphism for ringWithoutOne is extended from group homomorphism such that
got two ringWithoutOne structures $(R_1,+_1,*_1)$ and $(R_2,+_2,*_2)$, the
homomorphism $f: R_1 \rightarrow R_2$ is a group homomorphism and preserves the
multiplication operation. That is $f$ is a group homomorphism and \(f(x *_1 y) =
f(x) *_2 f(y)\).
\begin{minted}[breaklines,samepage]{Agda}
record IsRingWithoutOneHomomorphism (⟦_⟧ : A → B) : Set (a ⊔ ℓ₁ ⊔ ℓ₂) where
  field
    +-isGroupHomomorphism : +.IsGroupHomomorphism ⟦_⟧
    *-homo : Homomorphic₂ ⟦_⟧ _*₁_ _*₂_

  open +.IsGroupHomomorphism +-isGroupHomomorphism public
    renaming (homo to +-homo; ε-homo to 0#-homo;
    isMagmaHomomorphism to +-isMagmaHomomorphism)
\end{minted} 
In the above definition of ringWithoutOne homomorphism
\inline{IsRingWithoutOneHomomorphism} is defined as a record type with two
fields \inline{+-isGroupHomomorphism} and \inline{*-homo}. \inline{Homomorphic₂}
is used to define the homomorphism for \inline{*₁} and \inline{*₂}. A
Homomorphism that is injective is called monomorphism and can ve defined as:
\begin{minted}[breaklines,samepage]{Agda}
record IsRingWithoutOneMonomorphism (⟦_⟧ : A → B) : Set (a ⊔ ℓ₁ ⊔ ℓ₂) where
  field
    isRingWithoutOneHomomorphism : IsRingWithoutOneHomomorphism ⟦_⟧
    injective                    : Injective ⟦_⟧

  open IsRingWithoutOneHomomorphism isRingWithoutOneHomomorphism public
\end{minted}
A monomorphism that is bijective is called an isomorphism. Isomorphism of
ringWithoutOne structure can be defined as:
\begin{minted}[breaklines,samepage]{Agda}
record IsRingWithoutOneIsoMorphism (⟦_⟧ : A → B) : Set (a ⊔ b ⊔ ℓ₁ ⊔ ℓ₂) where
  field
    isRingWithoutOneMonomorphism : IsRingWithoutOneMonomorphism ⟦_⟧
    surjective                   : Surjective ⟦_⟧

  open IsRingWithoutOneMonomorphism isRingWithoutOneMonomorphism public
\end{minted}
\section{Morphism composition}
If $f$ is a morphism such that $f\ :\ a \ \rightarrow \ b$ and $g$ is a morphism
such that $g\ :\ b\ \rightarrow \ c$, then composition of morphism can be
defined as $g \ ∘\ f\ :\ a \ \rightarrow \ c$.
\begin{minted}[breaklines,samepage]{Agda}
  isRingWithoutOneHomomorphism
    : IsRingWithoutOneHomomorphism R₁ R₂ f
    → IsRingWithoutOneHomomorphism R₂ R₃ g
    → IsRingWithoutOneHomomorphism R₁ R₃ (g ∘ f)
  isRingWithoutOneHomomorphism f-homo g-homo = record
    { +-isGroupHomomorphism = isGroupHomomorphism ≈₃-trans
		 F.+-isGroupHomomorphism G.+-isGroupHomomorphism
    ; *-homo                 = λ x y → ≈₃-trans 
		(G.⟦⟧-cong (F.*-homo x y)) (G.*-homo (f x) (f y))
    } where module F = IsRingWithoutOneHomomorphism f-homo;
		 module G = IsRingWithoutOneHomomorphism g-homo
\end{minted}
In the above ringWithoutOne homomorphism composition, \inline{f} is a
homomorphism from ringWithoutOne structures $R₁$ to $R₂$, \inline{g} is a
homomorphism from ringWithoutOne structures $R₂$ to $R₃$.
\inline{isGroupHomomorphism} field givese the composition of group homomorphism.
We can define the composition for binary operations homomorphism (*) using
transitive relation \inline{≈₃-trans} from $R₁$ to $R₃$ such that \[g (f ((R₁ * x)
y)) ≈ (g ((R₂ * f x) (f y)) \text{ and } g ((R₂ * f x) (f y))) ≈ ((R₃ * g (f x))
(g (f y)))\]
\[\Rightarrow g (f ((R₁ * x) y)) ≈ ((R₃ * g (f x)) (g (f y)))\]
\section{Direct Product}
The \textit{direct product} of ring like structures in Agda.
\begin{minted}[breaklines,samepage]{Agda}
ringWithoutOne : RingWithoutOne a ℓ₁ → 
		RingWithoutOne b ℓ₂ → RingWithoutOne (a ⊔ b) (ℓ₁ ⊔ ℓ₂)
ringWithoutOne R S = record
  { isRingWithoutOne = record
      { +-isAbelianGroup = AbelianGroup.isAbelianGroup
		 ((abelianGroup R.+-abelianGroup S.+-abelianGroup))
      ; *-cong           = Semigroup.∙-cong
		 (semigroup R.*-semigroup S.*-semigroup)
      ; *-assoc   = Semigroup.assoc (semigroup R.*-semigroup S.*-semigroup)
      ; distrib     = (λ x y z → 
		(R.distribˡ , S.distribˡ) <*> x <*> y <*> z)
                            , (λ x y z → 
		(R.distribʳ , S.distribʳ) <*> x <*> y <*> z)
      ; zero     = uncurry (λ x y → R.zeroˡ x , S.zeroˡ y)
                            , uncurry (λ x y → R.zeroʳ x , S.zeroʳ y)
      }

  } where module R = RingWithoutOne R; module S = RingWithoutOne S
\end{minted}

\begin{minted}[breaklines,samepage]{Agda}
nonAssociativeRing : NonAssociativeRing a ℓ₁ →
		 NonAssociativeRing b ℓ₂ → NonAssociativeRing (a ⊔ b) (ℓ₁ ⊔ ℓ₂)
nonAssociativeRing R S = record
  { isNonAssociativeRing = record
      { +-isAbelianGroup = AbelianGroup.isAbelianGroup 
		((abelianGroup R.+-abelianGroup S.+-abelianGroup))
      ; *-cong           = UnitalMagma.∙-cong 
		(unitalMagma R.*-unitalMagma S.*-unitalMagma)
      ; *-identity       = UnitalMagma.identity 
		(unitalMagma R.*-unitalMagma S.*-unitalMagma)
      ; distrib          = (λ x y z → 
		(R.distribˡ , S.distribˡ) <*> x <*> y <*> z)
                            , (λ x y z → 
		(R.distribʳ , S.distribʳ) <*> x <*> y <*> z)
      ; zero             = uncurry (λ x y → R.zeroˡ x , S.zeroˡ y)
                            , uncurry (λ x y → R.zeroʳ x , S.zeroʳ y)
      }

  } where module R = NonAssociativeRing R; module S = NonAssociativeRing S
\end{minted}

\begin{minted}[breaklines,samepage]{Agda}
quasiring : Quasiring a ℓ₁ →
		 Quasiring b ℓ₂ → Quasiring (a ⊔ b) (ℓ₁ ⊔ ℓ₂)
quasiring R S = record
  { isQuasiring = record
      { +-isMonoid = Monoid.isMonoid
		 ((monoid R.+-monoid S.+-monoid))
      ; *-cong           = Monoid.∙-cong
		 (monoid R.*-monoid S.*-monoid)
      ; *-assoc          = Monoid.assoc
		 (monoid R.*-monoid S.*-monoid)
      ; *-identity       = Monoid.identity
		 ((monoid R.*-monoid S.*-monoid))
      ; distrib          = (λ x y z →
		 (R.distribˡ , S.distribˡ) <*> x <*> y <*> z)
                            , (λ x y z →
		 (R.distribʳ , S.distribʳ) <*> x <*> y <*> z)
      ; zero             = uncurry (λ x y → R.zeroˡ x , S.zeroˡ y)
                            , uncurry (λ x y → R.zeroʳ x , S.zeroʳ y)
      }

  } where module R = Quasiring R; module S = Quasiring S
\end{minted}

\begin{minted}[breaklines,samepage]{Agda}
nearring : Nearring a ℓ₁ → 
		Nearring b ℓ₂ → Nearring (a ⊔ b) (ℓ₁ ⊔ ℓ₂)
nearring R S =  record
  { isNearring = record
      { isQuasiring = Quasiring.isQuasiring
		 (quasiring R.quasiring S.quasiring)
      ; +-inverse   = (λ x → (R.+-inverseˡ , S.+-inverseˡ) <*> x)
                      , (λ x → (R.+-inverseʳ , S.+-inverseʳ) <*> x)
      ; ⁻¹-cong     = map R.⁻¹-cong S.⁻¹-cong
      }
  } where module R = Nearring R; module S = Nearring S
\end{minted}
\section{Properties}
With these definitions, we can prove some frequently used properties and theories
about the structures.\footnote{This section provides proof for properties that
was contributed by the author and other properties can be found in Agda standard
library.}
\subsection{Properties of Semigroup}
Let $(S, ∙)$ be a semigroup then
\begin{enumerate}
\item S is alternative. The Semigroup S left alternative if \((x\ ∙\ x)\ ∙\ y\ =\ x\ ∙\ (x\ ∙\ y)\ \) and right alternative is
\(x\ ∙\ (y\ ∙\ y)\ =\ (x\ ∙\ y)\ ∙\ y\). Semigroup is
said to be alternative if it is both left and right alternative. 
\item S is flexible. The Semigroup S is flexible if \(x\ ∙\ (y\
∙\ x)\ =\ (x\ ∙\ y)\ ∙\ x\).
\item S has Jordan identity.  Jordan identity for binary operation ∙ can be
defined on set S as \((x\ ∙\ y)\ ∙\ (x\ ∙\ x)\ =\
x\ ∙\ (y\ ∙\ (x\ ∙\ x)). \)
\end{enumerate}
Proof:
\begin{enumerate}
\item
\begin{minted}[breaklines,samepage]{Agda}
alternativeˡ : LeftAlternative _∙_
alternativeˡ x y = assoc x x y

alternativeʳ : RightAlternative _∙_
alternativeʳ x y = sym (assoc x y y)

alternative : Alternative _∙_
alternative = alternativeˡ , alternativeʳ
\end{minted}
 \item
\begin{minted}[breaklines,samepage]{Agda}
flexible : Flexible _∙_
flexible x y = assoc x y x
\end{minted}
\item
\begin{minted}[breaklines,samepage]{Agda}
xy∙xx≈x∙yxx : ∀ x y → (x ∙ y) ∙ (x ∙ x) ≈ x ∙ (y ∙ (x ∙ x))
xy∙xx≈x∙yxx x y = assoc x y ((x ∙ x))
\end{minted}
\end{enumerate}
\subsection{Properties of Commutative Semigroup}
Let $(S, ∙)$ be a commutative semigroup then
\begin{enumerate}
\item S is semimedial. The semigroup S is left semimedial if  \(
(x\ ∙\ x)\ ∙\ (y\ ∙\ z)\ =\ (x\ ∙\ y)\ ∙\ (x\ ∙\ z) \) and right
semimedial if \( (y\ ∙\ z)\ ∙\ (x\ ∙\ x)\ =\ (y\ ∙\ x)\ ∙\ (z\ ∙\ x) \).
A structure is semimedial if it is both left and right semimedial. 
\item S is middle semimedial. The semigroup S is middle semimedial if
\((x\ ∙\ y)\ ∙\ (z\ ∙\ x)\ =\ (x\ ∙\ z)\ ∙\ (y\ ∙\
x)\)
\end{enumerate}
Proof:
\begin{enumerate}
\item
\begin{minted}[breaklines,samepage]{Agda}
semimedialˡ : LeftSemimedial _∙_
semimedialˡ x y z = begin
(x ∙ x) ∙ (y ∙ z) ≈⟨ assoc x x (y ∙ z) ⟩
x ∙ (x ∙ (y ∙ z)) ≈⟨ ∙-congˡ (sym (assoc x y z)) ⟩
x ∙ ((x ∙ y) ∙ z) ≈⟨ ∙-congˡ (∙-congʳ (comm x y)) ⟩
x ∙ ((y ∙ x) ∙ z) ≈⟨ ∙-congˡ (assoc y x z) ⟩
x ∙ (y ∙ (x ∙ z)) ≈⟨ sym (assoc x y ((x ∙ z))) ⟩
(x ∙ y) ∙ (x ∙ z) ∎

semimedialʳ : RightSemimedial _∙_
semimedialʳ x y z = begin
(y ∙ z) ∙ (x ∙ x) ≈⟨ assoc y z (x ∙ x) ⟩
y ∙ (z ∙ (x ∙ x)) ≈⟨ ∙-congˡ (sym (assoc z x x)) ⟩
y ∙ ((z ∙ x) ∙ x) ≈⟨ ∙-congˡ (∙-congʳ (comm z x)) ⟩
y ∙ ((x ∙ z) ∙ x) ≈⟨ ∙-congˡ (assoc x z x) ⟩
y ∙ (x ∙ (z ∙ x)) ≈⟨ sym (assoc y x ((z ∙ x))) ⟩
(y ∙ x) ∙ (z ∙ x) ∎

semimedial : Semimedial _∙_
semimedial = semimedialˡ , semimedialʳ
\end{minted}
\item
\begin{minted}[breaklines,samepage]{Agda}
middleSemimedial : ∀ x y z → (x ∙ y) ∙ (z ∙ x) ≈ (x ∙ z) ∙ (y ∙ x)
middleSemimedial x y z = begin
  (x ∙ y) ∙ (z ∙ x) ≈⟨ assoc x y ((z ∙ x)) ⟩
  x ∙ (y ∙ (z ∙ x)) ≈⟨ ∙-congˡ (sym (assoc y z x)) ⟩
  x ∙ ((y ∙ z) ∙ x) ≈⟨ ∙-congˡ (∙-congʳ (comm y z)) ⟩
  x ∙ ((z ∙ y) ∙ x) ≈⟨ ∙-congˡ ( assoc z y x) ⟩
  x ∙ (z ∙ (y ∙ x)) ≈⟨ sym (assoc x z ((y ∙ x))) ⟩
  (x ∙ z) ∙ (y ∙ x) ∎
\end{minted}
\end{enumerate}
\subsection{Properties of Ring without one}
Let $(R, +, *, -, 0)$ be ring without one structure then:
\begin{enumerate}
\item \(- (x\ *\ y)\ =\ - x\ *\ y\)
\item \(- (x\ *\ y)\ =\ x\ *\ - y\)
\end{enumerate}
Proof:
\begin{enumerate}
\item
\begin{minted}[breaklines,samepage]{Agda}
-‿distribˡ-* : ∀ x y → - (x * y) ≈ - x * y
-‿distribˡ-* x y = sym $ begin
  - x * y                        
	≈⟨ sym $ +-identityʳ (- x * y) ⟩
  - x * y + 0#                   
	≈⟨ +-congˡ $ sym ( -‿inverseʳ (x * y) ) ⟩
  - x * y + (x * y + - (x * y))  
	≈⟨ sym $ +-assoc (- x * y) (x * y) (- (x * y)) ⟩
  - x * y + x * y + - (x * y)    
	≈⟨ +-congʳ $ sym ( distribʳ y (- x) x ) ⟩
  (- x + x) * y + - (x * y)      
	≈⟨ +-congʳ $ *-congʳ $ -‿inverseˡ x ⟩
  0# * y + - (x * y)             
	≈⟨ +-congʳ $ zeroˡ y ⟩
  0# + - (x * y)                 
	≈⟨ +-identityˡ (- (x * y)) ⟩
  - (x * y)                      
	∎
\end{minted}
\item
\begin{minted}[breaklines,samepage]{Agda}
-‿distribʳ-* : ∀ x y → - (x * y) ≈ x * - y
-‿distribʳ-* x y = sym $ begin
  x * - y                        
	≈⟨ sym $ +-identityˡ (x * (- y)) ⟩
  0# + x * - y                   
	≈⟨ +-congʳ $ sym ( -‿inverseˡ (x * y) ) ⟩
  - (x * y) + x * y + x * - y    
	≈⟨ +-assoc (- (x * y)) (x * y) (x * (- y)) ⟩
  - (x * y) + (x * y + x * - y)  
	≈⟨ +-congˡ $ sym ( distribˡ x y ( - y) )  ⟩
  - (x * y) + x * (y + - y)      
	≈⟨ +-congˡ $ *-congˡ $ -‿inverseʳ y ⟩
  - (x * y) + x * 0#             
	≈⟨ +-congˡ $ zeroʳ x ⟩
  - (x * y) + 0#                 
	≈⟨ +-identityʳ (- (x * y)) ⟩
  - (x * y)                      
	∎
\end{minted}
\end{enumerate}
\subsection{Properties of Ring}
Let $(R, +, *, -, 0, 1)$ be a ring structure then
\begin{enumerate}
\item \(- 1\ *\ x\ =\ -x\)
\item \(\text{if}\ x\ +\ x\ =\ 0\ \text{then}\ x\ =\ 0\)
\item \(x\ *\ (y\ -\ z)\ =\ x\ *\ y\ -\ x\ *\ z\)
\item \((y\ -\ z)\ *\ x\ =\ (y\ *\ x)\ -\ (z\ *\ x)\)
\end{enumerate}
Proof:
\begin{enumerate}
\item
\begin{minted}[breaklines,samepage]{Agda}
-1*x≈-x : ∀ x → - 1# * x ≈ - x
-1*x≈-x x = begin
  - 1# * x    ≈⟨ sym (-‿distribˡ-* 1# x ) ⟩
  - (1# * x)  ≈⟨ -‿cong ( *-identityˡ x ) ⟩
  - x         ∎
\end{minted}
\item
\begin{minted}[breaklines,samepage]{Agda}
x+x≈x⇒x≈0 : ∀ x → x + x ≈ x → x ≈ 0#
x+x≈x⇒x≈0 x eq = begin
  x ≈⟨ sym(+-identityʳ x) ⟩
  x + 0# ≈⟨ +-congˡ (sym (-‿inverseʳ x)) ⟩
  x + (x - x) ≈⟨ sym (+-assoc x x (- x)) ⟩
  x + x - x ≈⟨ +-congʳ(eq) ⟩
  x - x ≈⟨ -‿inverseʳ x ⟩
  0# ∎
\end{minted}
\item
\begin{minted}[breaklines,samepage]{Agda}
x[y-z]≈xy-xz : ∀ x y z → x * (y - z) ≈ x * y - x * z
x[y-z]≈xy-xz x y z = begin
  x * (y - z)      ≈⟨ distribˡ x y (- z) ⟩
  x * y + x * - z  ≈⟨ +-congˡ (sym (-‿distribʳ-* x z)) ⟩
  x * y - x * z    ∎
\end{minted}
\item
\begin{minted}[breaklines,samepage]{Agda}
[y-z]x≈yx-zx : ∀ x y z → (y - z) * x ≈ (y * x) - (z * x)
[y-z]x≈yx-zx x y z = begin
  (y - z) * x      ≈⟨ distribʳ x y (- z) ⟩
  y * x + - z * x  ≈⟨ +-congˡ (sym (-‿distribˡ-* z x)) ⟩
  y * x - z * x    ∎
\end{minted}
\end{enumerate}