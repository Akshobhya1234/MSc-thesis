\chapter{Theory of Semigroup and Ring in Agda}
In early 20th century, mathematician Hilbert proposed the H\textsubscript{10}
problem that argues if there exists a useful approach to verify whether a
general Diophantine equation is solvable. \cite{larchey2020hilbert}. Although
this problem was solved by 1970, In 1987 Siekmann and Szabo concluded that the
unification problem of D\textsubscript{A}-rewriting system cannot be predicted.
In \cite{deng2016characterizations} the author proposes a type (2,2,0) algebra
that is a semigroup that is a general construct of D\textsubscript{A}-rewriting
system. Other applications of semigroup in finite automata systems, probability
theory and partial differential equations are explored in
\cite{liaqat2021some}.\\

Ring is an algebraic structure has its applications in various branch if
studies. Ring structures are studied in number theory
\cite{noauthor_undated-eq}, quantum computing \cite{netto2008influence}, in
cryptography \cite{ringcrypt} and many other fields. More variations of rings
such as nearring, quasiring, ideal ring are being explored to make ring theory
more dynamic, concrete and useable. The aim of this chapter is to define these
structures and prove some of the most commonly used properties in the standard
library that can help build other systems that uses these structures.
\section{Definition}
A set that has a binary operation is called Magma. Magma with associative
property is called a semigroup. For binary operation \(∙\) on a set S, the
associative property is defined as 
\begin{equation}
\forall x y z \in S : x ∙ (y ∙ z) = (x ∙ y) ∙ z.
\end{equation}
 A semigroup that satisfies commutative property is called commutative
 semigroup. For binary operation \( ∙ \) on a set S, commutative property is
 defined as 
\begin{equation}
\forall x y \in S : x ∙ y = y ∙ x
\end{equation}
The definition of associative and commutative property in Agda.
\begin{minted}[breaklines,samepage]{Agda}
Associative : Op₂ A → Set _
Associative _∙_ = ∀ x y z → ((x ∙ y) ∙ z) ≈ (x ∙ (y ∙ z))

Commutative : Op₂ A → Set _
Commutative _∙_ = ∀ x y → (x ∙ y) ≈ (y ∙ x)
\end{minted}

The Semigroup structure can be structurally derived from Magma in Agda as 

\begin{minted}[breaklines,samepage]{Agda}
record IsSemigroup (∙ : Op₂ A) : Set (a ⊔ ℓ) where
field
  isMagma : IsMagma ∙
  assoc   : Associative ∙

open IsMagma isMagma public
\end{minted}
Similarly, commutative semigroup can be derived from semigroup as
\begin{minted}[breaklines,samepage]{Agda}
record IsCommutativeSemigroup (∙ : Op₂ A) : Set (a ⊔ ℓ) where
field
  isSemigroup : IsSemigroup ∙
  comm        : Commutative ∙

open IsSemigroup isSemigroup public
\end{minted}
Semigroup and commutative semigroup structure definitions with direct product
and morphism constructs were previously defined in agda standard library and
hence will not be discussed in details in this chapter.\\
Non-associative ring is an algebraic structure with two binary operations
addition and multiplication. Addition is an abelian group and multiplication is
unital magma and multiplication distributes over addition. A unital magma is a
magma with identity element. 
\begin{minted}[breaklines,samepage]{Agda}
record IsNonAssociativeRing (+ * : Op₂ A) (-_ : Op₁ A) (0# 1# : A) : Set (a ⊔ ℓ) where
field
  +-isAbelianGroup : IsAbelianGroup + 0# -_
  *-cong           : Congruent₂ *
  *-identity       : Identity 1# *
  distrib          : * DistributesOver +
  zero             : Zero 0# *
\end{minted}

A quasiring is a type (2,2) algebraic structure for which both addition and multiplication is a monoid and multiplication distributes over addition.

\begin{minted}[breaklines,samepage]{Agda}
record IsQuasiring (+ * : Op₂ A) (0# 1# : A) : Set (a ⊔ ℓ) where
  field
    +-isMonoid    : IsMonoid + 0#
    *-cong        : Congruent₂ *
    *-assoc       : Associative *
    *-identity    : Identity 1# *
    distrib       : * DistributesOver +
    zero          : Zero 0# *

  open IsMonoid +-isMonoid public
\end{minted}

A quasiring with additive inverse is called a nearring. This implies that for
nearring addition is a group and multiplication is a monoid and multiplication
distributes over addition.

\begin{minted}[breaklines,samepage]{Agda}
record IsNearring (+ * : Op₂ A) (0# 1# : A) (_⁻¹ : Op₁ A) : Set (a ⊔ ℓ) where
  field
    isQuasiring : IsQuasiring + * 0# 1#
    +-inverse   : Inverse 0# _⁻¹ +
    ⁻¹-cong     : Congruent₁ _⁻¹

  open IsQuasiring isQuasiring public
\end{minted}

Ring without one or rig or ring without unit is an algebraic structure with two binary operations such that addition is an abelian group and multiplication is a semigroup and multiplication distributes over addition Ring is rig with identity.
\begin{minted}[breaklines,samepage]{Agda}
record IsRingWithoutOne (+ * : Op₂ A) (-_ : Op₁ A) (0# : A) : Set (a ⊔ ℓ) where
  field
    +-isAbelianGroup : IsAbelianGroup + 0# -_
    *-cong           : Congruent₂ *
    *-assoc          : Associative *
    distrib          : * DistributesOver +
    zero             : Zero 0# *

  open IsAbelianGroup +-isAbelianGroup public
\end{minted}
\section{Morphism} 
A structure preserving map between two structures is called morphism. In this
section morphism of ring without one is given. Morphisms of other structures
that are structurally different from ring without one discussed in this section
can be found in agda standard library
\begin{minted}[breaklines,samepage]{Agda}
record IsRingWithoutOneHomomorphism (⟦_⟧ : A → B) : Set (a ⊔ ℓ₁ ⊔ ℓ₂) where
  field
    +-isGroupHomomorphism : +.IsGroupHomomorphism ⟦_⟧
    *-homo : Homomorphic₂ ⟦_⟧ _*₁_ _*₂_

  open +.IsGroupHomomorphism +-isGroupHomomorphism public
    renaming (homo to +-homo; ε-homo to 0#-homo;
    isMagmaHomomorphism to +-isMagmaHomomorphism)
\end{minted} 
A Homomorphism that is injective is called monomorphism 
\begin{minted}[breaklines,samepage]{Agda}
record IsRingWithoutOneMonomorphism (⟦_⟧ : A → B) : Set (a ⊔ ℓ₁ ⊔ ℓ₂) where
  field
    isRingWithoutOneHomomorphism : IsRingWithoutOneHomomorphism ⟦_⟧
    injective                    : Injective ⟦_⟧

  open IsRingWithoutOneHomomorphism isRingWithoutOneHomomorphism public
\end{minted}
A monomorphism that is bijective is called an isomorphism
\begin{minted}[breaklines,samepage]{Agda}
record IsRingWithoutOneMonomorphism (⟦_⟧ : A → B) : Set (a ⊔ ℓ₁ ⊔ ℓ₂) where
  field
    isRingWithoutOneHomomorphism : IsRingWithoutOneHomomorphism ⟦_⟧
    injective                    : Injective ⟦_⟧

  open IsRingWithoutOneHomomorphism isRingWithoutOneHomomorphism public
\end{minted}
\section{Morphism composition}
If f is a morphism such that f : a \(\rightarrow\) b and g is a morphism on same
structure such that g : b \(\rightarrow\) c then composition of morphism can be
defined as g ∘ f : a \(\rightarrow\) c.
\begin{minted}[breaklines,samepage]{Agda}
  isRingWithoutOneHomomorphism
    : IsRingWithoutOneHomomorphism R₁ R₂ f
    → IsRingWithoutOneHomomorphism R₂ R₃ g
    → IsRingWithoutOneHomomorphism R₁ R₃ (g ∘ f)
  isRingWithoutOneHomomorphism f-homo g-homo = record
    { +-isGroupHomomorphism = isGroupHomomorphism ≈₃-trans
		 F.+-isGroupHomomorphism G.+-isGroupHomomorphism
    ; *-homo                 = λ x y → ≈₃-trans 
		(G.⟦⟧-cong (F.*-homo x y)) (G.*-homo (f x) (f y))
    } where module F = IsRingWithoutOneHomomorphism f-homo;
		 module G = IsRingWithoutOneHomomorphism g-homo
\end{minted}

\section{Direct Product}
The direct product M \(\times\) N of two ring without one structures M and N is
defined as a pair (m,n) where m \(\in\) M and n \(\in\) N.
\begin{minted}[breaklines,samepage]{Agda}
ringWithoutOne : RingWithoutOne a ℓ₁ → 
		RingWithoutOne b ℓ₂ → RingWithoutOne (a ⊔ b) (ℓ₁ ⊔ ℓ₂)
ringWithoutOne R S = record
  { isRingWithoutOne = record
      { +-isAbelianGroup = AbelianGroup.isAbelianGroup
		 ((abelianGroup R.+-abelianGroup S.+-abelianGroup))
      ; *-cong           = Semigroup.∙-cong
		 (semigroup R.*-semigroup S.*-semigroup)
      ; *-assoc   = Semigroup.assoc (semigroup R.*-semigroup S.*-semigroup)
      ; distrib     = (λ x y z → 
		(R.distribˡ , S.distribˡ) <*> x <*> y <*> z)
                            , (λ x y z → 
		(R.distribʳ , S.distribʳ) <*> x <*> y <*> z)
      ; zero     = uncurry (λ x y → R.zeroˡ x , S.zeroˡ y)
                            , uncurry (λ x y → R.zeroʳ x , S.zeroʳ y)
      }

  } where module R = RingWithoutOne R; module S = RingWithoutOne S
\end{minted}
The direct product M \(\times\) N of two non associative ring a structures M and
N is defined as a pair (m,n) where m \(\in\) M and n \(\in\) N.
\begin{minted}[breaklines,samepage]{Agda}
nonAssociativeRing : NonAssociativeRing a ℓ₁ →
		 NonAssociativeRing b ℓ₂ → NonAssociativeRing (a ⊔ b) (ℓ₁ ⊔ ℓ₂)
nonAssociativeRing R S = record
  { isNonAssociativeRing = record
      { +-isAbelianGroup = AbelianGroup.isAbelianGroup 
		((abelianGroup R.+-abelianGroup S.+-abelianGroup))
      ; *-cong           = UnitalMagma.∙-cong 
		(unitalMagma R.*-unitalMagma S.*-unitalMagma)
      ; *-identity       = UnitalMagma.identity 
		(unitalMagma R.*-unitalMagma S.*-unitalMagma)
      ; distrib          = (λ x y z → 
		(R.distribˡ , S.distribˡ) <*> x <*> y <*> z)
                            , (λ x y z → 
		(R.distribʳ , S.distribʳ) <*> x <*> y <*> z)
      ; zero             = uncurry (λ x y → R.zeroˡ x , S.zeroˡ y)
                            , uncurry (λ x y → R.zeroʳ x , S.zeroʳ y)
      }

  } where module R = NonAssociativeRing R; module S = NonAssociativeRing S
\end{minted}
The direct product M \(\times\) N of two quasiring a structures M and N is
defined as a pair (m,n) where m \(\in\) M and n \(\in\) N.
\begin{minted}[breaklines,samepage]{Agda}
quasiring : Quasiring a ℓ₁ →
		 Quasiring b ℓ₂ → Quasiring (a ⊔ b) (ℓ₁ ⊔ ℓ₂)
quasiring R S = record
  { isQuasiring = record
      { +-isMonoid = Monoid.isMonoid
		 ((monoid R.+-monoid S.+-monoid))
      ; *-cong           = Monoid.∙-cong
		 (monoid R.*-monoid S.*-monoid)
      ; *-assoc          = Monoid.assoc
		 (monoid R.*-monoid S.*-monoid)
      ; *-identity       = Monoid.identity
		 ((monoid R.*-monoid S.*-monoid))
      ; distrib          = (λ x y z →
		 (R.distribˡ , S.distribˡ) <*> x <*> y <*> z)
                            , (λ x y z →
		 (R.distribʳ , S.distribʳ) <*> x <*> y <*> z)
      ; zero             = uncurry (λ x y → R.zeroˡ x , S.zeroˡ y)
                            , uncurry (λ x y → R.zeroʳ x , S.zeroʳ y)
      }

  } where module R = Quasiring R; module S = Quasiring S
\end{minted}
The direct product M \(\times\) N of two nearring a structures M and N is
defined as a pair (m,n) where m \(\in\) M and n \(\in\) N.
\begin{minted}[breaklines,samepage]{Agda}
nearring : Nearring a ℓ₁ → 
		Nearring b ℓ₂ → Nearring (a ⊔ b) (ℓ₁ ⊔ ℓ₂)
nearring R S =  record
  { isNearring = record
      { isQuasiring = Quasiring.isQuasiring
		 (quasiring R.quasiring S.quasiring)
      ; +-inverse   = (λ x → (R.+-inverseˡ , S.+-inverseˡ) <*> x)
                      , (λ x → (R.+-inverseʳ , S.+-inverseʳ) <*> x)
      ; ⁻¹-cong     = map R.⁻¹-cong S.⁻¹-cong
      }
  } where module R = Nearring R; module S = Nearring S
\end{minted}
\section{Properties}
This section only provides proof for properties that was contributed by the
author and other properties can be found in agda standard library.
\subsection{Properties of Semigroup}
Let (S, ∙) be a semigroup then
\begin{enumerate}
\item S is alternative. The Semigroup S left alternative if \( \forall x,y \in S : (x ∙ x) ∙ y = x ∙ (x ∙ y) \) and right alternative is \(\forall x,y \in S : x ∙ (y ∙ y) = (x ∙ y) ∙ y\). Semigroup is said to be alternative if it is both left and right alternative. 
\item S is flexible. The Semigroup S is flexible if \(\forall x y \in S: x ∙ (y ∙ x) = (x ∙ y) ∙ x\)
\item S has Jordan identity.  Jordan identity for binary operation ∙ can be defined on set S as \( \forall x y z \in S: (x ∙ y) ∙ (x ∙ x) = x ∙ (y ∙ (x ∙ x)) \)
\end{enumerate}
Proof:
\begin{enumerate}
\item
\begin{minted}[breaklines,samepage]{Agda}
alternativeˡ : LeftAlternative _∙_
alternativeˡ x y = assoc x x y

alternativeʳ : RightAlternative _∙_
alternativeʳ x y = sym (assoc x y y)

alternative : Alternative _∙_
alternative = alternativeˡ , alternativeʳ
\end{minted}
 \item
 \begin{minted}[breaklines,samepage]{Agda}
flexible : Flexible _∙_
flexible x y = assoc x y x
\end{minted}
\item
\begin{minted}[breaklines,samepage]{Agda}
xy∙xx≈x∙yxx : ∀ x y → (x ∙ y) ∙ (x ∙ x) ≈ x ∙ (y ∙ (x ∙ x))
xy∙xx≈x∙yxx x y = assoc x y ((x ∙ x))
\end{minted}
\end{enumerate}
\subsection{Properties of Commutative Semigroup}
Let (S, ∙) be a commutative semigroup then
\begin{enumerate}
\item S is semimedial. The commutative semigroup S is left semimedial if  \(
\forall x y z \in S: (x ∙ x) ∙ (y ∙ z) = (x ∙ y) ∙ (x ∙ z) \) and right
semimedial if \( \forall x y z \in S: (y ∙ z) ∙ (x ∙ x) = (y ∙ x) ∙ (z ∙ x) \).
A structure is semimedial if it is both left and right semimedial. 
\item S is middle semimedia.  The commutative semigroup S is middle semimedial
if  \(\forall x y z  \in S: (x ∙ y) ∙ (z ∙ x) = (x ∙ z) ∙ (y ∙ x)\)
\end{enumerate}
Proof:
\begin{enumerate}
\item
\begin{minted}[breaklines,samepage]{Agda}
semimedialˡ : LeftSemimedial _∙_
semimedialˡ x y z = begin
(x ∙ x) ∙ (y ∙ z) ≈⟨ assoc x x (y ∙ z) ⟩
x ∙ (x ∙ (y ∙ z)) ≈⟨ ∙-congˡ (sym (assoc x y z)) ⟩
x ∙ ((x ∙ y) ∙ z) ≈⟨ ∙-congˡ (∙-congʳ (comm x y)) ⟩
x ∙ ((y ∙ x) ∙ z) ≈⟨ ∙-congˡ (assoc y x z) ⟩
x ∙ (y ∙ (x ∙ z)) ≈⟨ sym (assoc x y ((x ∙ z))) ⟩
(x ∙ y) ∙ (x ∙ z) ∎

semimedialʳ : RightSemimedial _∙_
semimedialʳ x y z = begin
(y ∙ z) ∙ (x ∙ x) ≈⟨ assoc y z (x ∙ x) ⟩
y ∙ (z ∙ (x ∙ x)) ≈⟨ ∙-congˡ (sym (assoc z x x)) ⟩
y ∙ ((z ∙ x) ∙ x) ≈⟨ ∙-congˡ (∙-congʳ (comm z x)) ⟩
y ∙ ((x ∙ z) ∙ x) ≈⟨ ∙-congˡ (assoc x z x) ⟩
y ∙ (x ∙ (z ∙ x)) ≈⟨ sym (assoc y x ((z ∙ x))) ⟩
(y ∙ x) ∙ (z ∙ x) ∎

semimedial : Semimedial _∙_
semimedial = semimedialˡ , semimedialʳ
\end{minted}
\item
\begin{minted}[breaklines,samepage]{Agda}
middleSemimedial : ∀ x y z → (x ∙ y) ∙ (z ∙ x) ≈ (x ∙ z) ∙ (y ∙ x)
middleSemimedial x y z = begin
  (x ∙ y) ∙ (z ∙ x) ≈⟨ assoc x y ((z ∙ x)) ⟩
  x ∙ (y ∙ (z ∙ x)) ≈⟨ ∙-congˡ (sym (assoc y z x)) ⟩
  x ∙ ((y ∙ z) ∙ x) ≈⟨ ∙-congˡ (∙-congʳ (comm y z)) ⟩
  x ∙ ((z ∙ y) ∙ x) ≈⟨ ∙-congˡ ( assoc z y x) ⟩
  x ∙ (z ∙ (y ∙ x)) ≈⟨ sym (assoc x z ((y ∙ x))) ⟩
  (x ∙ z) ∙ (y ∙ x) ∎
\end{minted}
\end{enumerate}
\subsection{Properties of Ring without one}
Let (R, +, *, -, 0) be ring without one structure then:
\begin{enumerate}
\item \(\forall x, y \in R\): - (x * y) = - x * y\\
\item \(\forall x, y \in R\): - (x * y) = x * - y\\
\end{enumerate}
proof:
\begin{enumerate}
\item
\begin{minted}[breaklines,samepage]{Agda}
-‿distribˡ-* : ∀ x y → - (x * y) ≈ - x * y
-‿distribˡ-* x y = sym $ begin
  - x * y                        
	≈⟨ sym $ +-identityʳ (- x * y) ⟩
  - x * y + 0#                   
	≈⟨ +-congˡ $ sym ( -‿inverseʳ (x * y) ) ⟩
  - x * y + (x * y + - (x * y))  
	≈⟨ sym $ +-assoc (- x * y) (x * y) (- (x * y)) ⟩
  - x * y + x * y + - (x * y)    
	≈⟨ +-congʳ $ sym ( distribʳ y (- x) x ) ⟩
  (- x + x) * y + - (x * y)      
	≈⟨ +-congʳ $ *-congʳ $ -‿inverseˡ x ⟩
  0# * y + - (x * y)             
	≈⟨ +-congʳ $ zeroˡ y ⟩
  0# + - (x * y)                 
	≈⟨ +-identityˡ (- (x * y)) ⟩
  - (x * y)                      
	∎
\end{minted}
\item
\begin{minted}[breaklines,samepage]{Agda}
-‿distribʳ-* : ∀ x y → - (x * y) ≈ x * - y
-‿distribʳ-* x y = sym $ begin
  x * - y                        
	≈⟨ sym $ +-identityˡ (x * (- y)) ⟩
  0# + x * - y                   
	≈⟨ +-congʳ $ sym ( -‿inverseˡ (x * y) ) ⟩
  - (x * y) + x * y + x * - y    
	≈⟨ +-assoc (- (x * y)) (x * y) (x * (- y)) ⟩
  - (x * y) + (x * y + x * - y)  
	≈⟨ +-congˡ $ sym ( distribˡ x y ( - y) )  ⟩
  - (x * y) + x * (y + - y)      
	≈⟨ +-congˡ $ *-congˡ $ -‿inverseʳ y ⟩
  - (x * y) + x * 0#             
	≈⟨ +-congˡ $ zeroʳ x ⟩
  - (x * y) + 0#                 
	≈⟨ +-identityʳ (- (x * y)) ⟩
  - (x * y)                      
	∎
\end{minted}
\end{enumerate}
\subsection{Properties of Ring}
Let (R, +, *, -, 0, 1) be a ring structure then
\begin{enumerate}
\item \(\forall x \in R\): - 1 * x = -x
\item \(\forall x \in R\): if x + x = 0 then x = 0
\item \(\forall x,y,z \in R\): x * (y - z) = x * y - x * z
\item \(\forall x,y,z \in R\): (y - z) * x = (y * x) - (z * x)
\end{enumerate}
Proof:
\begin{enumerate}
\item
\begin{minted}[breaklines,samepage]{Agda}
-1*x≈-x : ∀ x → - 1# * x ≈ - x
-1*x≈-x x = begin
  - 1# * x    ≈⟨ sym (-‿distribˡ-* 1# x ) ⟩
  - (1# * x)  ≈⟨ -‿cong ( *-identityˡ x ) ⟩
  - x         ∎
\end{minted}
\item
\begin{minted}[breaklines,samepage]{Agda}
x+x≈x⇒x≈0 : ∀ x → x + x ≈ x → x ≈ 0#
x+x≈x⇒x≈0 x eq = begin
  x ≈⟨ sym(+-identityʳ x) ⟩
  x + 0# ≈⟨ +-congˡ (sym (-‿inverseʳ x)) ⟩
  x + (x - x) ≈⟨ sym (+-assoc x x (- x)) ⟩
  x + x - x ≈⟨ +-congʳ(eq) ⟩
  x - x ≈⟨ -‿inverseʳ x ⟩
  0# ∎
\end{minted}
\item
\begin{minted}[breaklines,samepage]{Agda}
x[y-z]≈xy-xz : ∀ x y z → x * (y - z) ≈ x * y - x * z
x[y-z]≈xy-xz x y z = begin
  x * (y - z)      ≈⟨ distribˡ x y (- z) ⟩
  x * y + x * - z  ≈⟨ +-congˡ (sym (-‿distribʳ-* x z)) ⟩
  x * y - x * z    ∎
\end{minted}
\item
\begin{minted}[breaklines,samepage]{Agda}
[y-z]x≈yx-zx : ∀ x y z → (y - z) * x ≈ (y * x) - (z * x)
[y-z]x≈yx-zx x y z = begin
  (y - z) * x      ≈⟨ distribʳ x y (- z) ⟩
  y * x + - z * x  ≈⟨ +-congˡ (sym (-‿distribˡ-* z x)) ⟩
  y * x - z * x    ∎
\end{minted}
\end{enumerate}