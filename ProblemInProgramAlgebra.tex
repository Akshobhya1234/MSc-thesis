\chapter{Problem in Programming Algebra}
Algebraic structures show variations in syntax and semantics depending on the
system or language in which they are defined. Each system discussed in chapter 3
have their own style of defining structures in the standard libraries. For
example, in Coq, ring is defined without multiplicative identity. However, in
Agda, ring has multiplicative identity and rng is defined as ringWithoutOne that
has no multiplicative identity. This ambiguity in naming is also seen in
literature. Another example is same structure having multiple definitions like
Quasigroups. Quasigroups can be defined as a type(2) algebra with Latin square
property or as a type(2,2,2) with left and right division operators. Both the
definitions are equivalent, but they are structurally different. This chapter
identifies and classifies five important problems that arises when defining
types algebraic structures in proof assistant systems. 

\section{Ambiguity in naming}
Ambiguity arises when something can be interpreted in more than one way. The
example of quasigroup having more than one definition can give rise to a
scenario of making an incorrect interpretation of the algebraic structure when
it is not clearly stated. In abstract algebra and algebraic structure these
scenarios can be more common and this can be attributed to lack of naming
convention that is followed in naming algebraic structures and it's properties.
For example, consider algebraic structures ring and rng. Some mathematicians
define ring as an algebraic structure that is an abelian group under addition
and a monoid under multiplication. This definition is also named explicitly as
ring with unit or ring with identity. Rng is defined as an algebraic structure
that is an abelian group under addition and a semigroup under multiplication.
The same structure is also defined as ring without identity. However, these
definitions are often interchanged i.e., some mathematicians define ring as ring
without identity that is multiplication has no identity or is a semigroup. This
ambiguity may be attributed to the language of origin of the algebraic
structures. In this case rng is used in French whereas ring in English. These
confusions can be seen in literature and in online blogs where it is difficult
to imply the definition of intent when they are not explicitly defined. 

In Agda, a ring structure is defined as an algebraic structure with two binary
operations $+$ and $*$, one unary operator $^{-1}$, and two elements $0$ and $1$
on setoid $A$. $(A,+,^{-1},0)$ is an abelian group and $(A,*,1)$ is a monoid.
The binary operation $*$ distributes over $+$, that is multiplication
distributes over addition, and it has annihilating zero.

\begin{minted}[breaklines,samepage]{Agda}
record IsRing (+ * : Op₂ A) (-_ : Op₁ A) (0# 1# : A) : Set (a ⊔ ℓ) where
field
  +-isAbelianGroup : IsAbelianGroup + 0# -_
  *-cong           : Congruent₂ *
  *-assoc          : Associative *
  *-identity       : Identity 1# *
  distrib          : * DistributesOver +
  zero             : Zero 0# *

open IsAbelianGroup +-isAbelianGroup public
\end{minted} 
\inline{Rng} is defined as ringWihthoutOne that is a ring structure without
multiplicative identity.
\begin{minted}[breaklines,samepage]{Agda}
record IsRingWithoutOne (+ * : Op₂ A) (-_ : Op₁ A) (0# : A) : Set (a ⊔ ℓ) where
field
  +-isAbelianGroup : IsAbelianGroup + 0# -_
  *-cong           : Congruent₂ *
  *-assoc          : Associative *
  distrib          : * DistributesOver +
  zero             : Zero 0# *

open IsAbelianGroup +-isAbelianGroup public
\end{minted}

Another example of ambiguity arises when defining structure nearring. Nearring
is defined as a structure for which addition is a group and multiplication is a
monoid. But some mathematicians use the definition where multiplication is a
semigroup. The same confusion also arises in defining semiring and rig
structures. \cite{enwiki:1133737666} states that the term rig originated as a
joke that it is similar to rng that is missing the alphabet n and i to represent
the identity does not exist for these structures. In Agda, the algebraic
structure rig is defined as semiring without one where one is represents the
multiplicative identity.

For axioms of structures, the names are usually invented when defining the
structure. As an example when defining Kleene Algebra in Agda,
\inline{starExpansive} and \inline{starDestructive} names were invented
(inspired from what is used in literature). Due to lack of standardized names,
many names can be coined for the same axiom.

\begin{minted}[breaklines,samepage]{Agda}
record IsKleeneAlgebra (+ * : Op₂ A) (⋆ : Op₁ A) (0# 1# : A) : Set (a ⊔ ℓ) where
field
  isIdempotentSemiring  : IsIdempotentSemiring + * 0# 1#
  starExpansive         : StarExpansive 1# + * ⋆
  starDestructive       : StarDestructive + * ⋆

open IsIdempotentSemiring isIdempotentSemiring public
\end{minted}

\section{Equivalent but structurally different}
Quasigroup structure is an example that can be defined in two ways that are
equivalent but structurally different. A type (2) Quasigroup can be defined as a
set Q and binary operation ∙ that is a magma and satisfies Latin square
property. Quasigroup of type (2,2,2) is a structure with three binary
operations, a magma with division operation. Latin square property states that
for each a, b in set Q there exists unique elements x, y in Q such that the
following property is satisfied:
\cite{quasigroupWiki}
\begin{center}
\[a ∙ x = b\]
\[y ∙ a = b\]
\end{center}
Another definition of quasigroup is given as type a (2,2,2) algebra in which for a
set Q and binary operations ∙, \textbackslash{}, / quasigroup should satisfy
the below identities that implies left division and right division. 
\[y = x ∙ (x \backslash y)\]
\[y = x \backslash (x ∙ y)\]
\[y = (y / x) ∙ x\]
\[y = (y ∙ x) / x\] 
In Agda standard library, the quasigroup is defined as a
type (2,2,2) algebra (shown below).

\begin{minted}[breaklines,samepage]{Agda}
record IsQuasigroup (∙ \\ // : Op₂ A) : Set (a ⊔ ℓ) where
field
  isMagma       : IsMagma ∙
  \\-cong       : Congruent₂ \\
  //-cong       : Congruent₂ //
  leftDivides   : LeftDivides ∙ \\
  rightDivides  : RightDivides ∙ //

open IsMagma isMagma public
\end{minted}

A quasigroup that is a type (2) algebra and a quasigroup that is a type (2,2,2)
algebra are equivalent but are structurally different \cite{flinn2021algebraic}. In the
algebra hierarchy, a Loop is an algebraic structure that is a quasigroup with
identity. It can be observed the same problem persists through the hierarchy. If
a loop is defined with a quasigroup that is a type (2,2,2) algebra then, a
loop structure of type (2) will be forced to be defined with suboptimal name.
One possible solution to this problem is to define the structures in different
modules and import restrict them when using. This problem of not being able to
overload names for structures also affects when defining types of quasigroup or
loops such as bol loop and moufang loop.

Since quasigroup is defined in terms of division operation, loop is also defined
as a type (2,2,2) algebra in Agda. The definition of loop structure in Agda is
as follows:

\begin{minted}[breaklines,samepage]{Agda}
record IsLoop (∙ \\ // : Op₂ A) (ε : A) : Set (a ⊔ ℓ) where
field
  isQuasigroup : IsQuasigroup ∙ \\ //
  identity     : Identity ε ∙

open IsQuasigroup isQuasigroup public
\end{minted}

\section{Redundant field in structural inheritance}
Redundancy arises when there is duplication of the same field. In programming
redundant of code is considered a bad practice and is usually avoided by
modularizing and creating functions that perform similar tasks. In algebraic
structures, redundant fields can be introduced in structures that are defined in
terms of two or more structures. For example semiring can be defined as
commutative monoid under addition and a monoid under multiplication. In Agda,
both monoid and commutative monoid have an instance of equivalence relation.
Hence, if semiring is defined in terms of commutative monoid and monoid then
this definition of the semiring will have a redundant equivalence field. This
redundancy can also be seen in other structures like ring, lattice, module, and
other algebraic structures. To remove this redundant field in Agda the structure
except the first is opened and expressed in terms of independent axioms that
they satisfy. For example, semiring without identity or rig structure in Agda is
defined as:
\begin{minted}[breaklines,samepage]{Agda}
record IsSemiringWithoutOne (+ * : Op₂ A) (0# : A) : Set (a ⊔ ℓ) where
field
  +-isCommutativeMonoid : IsCommutativeMonoid + 0#
  *-cong                : Congruent₂ *
  *-assoc               : Associative *
  distrib               : * DistributesOver +
  zero                  : Zero 0# *

open IsCommutativeMonoid +-isCommutativeMonoid public
\end{minted}
From the above definition, we can observe that the operation $*$ is a semigroup
is expressed with axioms congruent and associative. But, there is no field to
say that $*$ is a semigroup. To overcome this problem an instance is created in
the definition as follows along with near semiring structure.
\begin{minted}[breaklines,samepage]{Agda}
*-isMagma : IsMagma *
*-isMagma = record
  { isEquivalence = isEquivalence
  ; ∙-cong        = *-cong
  }

*-isSemigroup : IsSemigroup *
*-isSemigroup = record
  { isMagma = *-isMagma
  ; assoc   = *-assoc
  }

isNearSemiring : IsNearSemiring + * 0#
isNearSemiring = record
  { +-isMonoid    = +-isMonoid
  ; *-cong        = *-cong
  ; *-assoc       = *-assoc
  ; distribʳ      = proj₂ distrib
  ; zeroˡ         = zeroˡ
  }
\end{minted}
The above technique will effectively remove the redundant equivalence relation.
However, it fails to express the structure in terms of two or more structures
that is commonly used in literature and in other systems. Agda 2.0 removed
redundancy by unfolding the structure. This solution should ensure that the
structure clearly exports the unfolded structure whose properties can be
imported when required.

\section{Identical structures}
In abstract algebra when formalizing algebraic structures from the hierarchy,
same algebraic structure can be derived from two or more structures. One such
example is Nearring. Nearring is an algebraic structure with two binary
operations addition and multiplication. Nearring is a group under addition and
is a monoid under multiplication and multiplication right distributes over
addition. In this case nearring is defined using two algebraic structures group
and monoid. Other definition of nearring can be derived using the structure
quasiring. Quasiring is an algebraic structure in which addition is a monoid,
multiplication is a monoid and multiplication distributes over addition. Using
this definition of quasiring, nearring can be defined as a quasiring which has
an additive inverse. In Agda nearring is defined in terms of quasiring with
additive inverse 

\begin{minted}[breaklines,samepage]{Agda}
record IsNearring (+ * : Op₂ A) (0# 1# : A) (_⁻¹ : Op₁ A) : Set (a ⊔ ℓ) where
field
  isQuasiring : IsQuasiring + * 0# 1#
  +-inverse   : Inverse 0# _⁻¹ +
  ⁻¹-cong     : Congruent₁ _⁻¹

open IsQuasiring isQuasiring public
\end{minted}

Note that in some literature, nearring is defined in which multiplication is a
semigroup that is without identity. This can be attributed to the problem with
ambiguity. It can be analyzed that having two different definitions for same
structure is not a good practice. If nearring is defined using quasiring then
it should also give an instance of additive group without having it to construct
it when using the above formalization. This solution might solve the problem at
first but in practice this becomes tedious and can go to a point at which this
can be impractical especially when formalizing structures at higher level in the
algebra hierarchy.

\section{Equivalent structures}
Consider the example of idempotent commutative monoid and bounded semilattice.
It can be observed that both are essentially the same structure. It is redundant
to define two different structures from different hierarchy. Instead, in Agda,
aliasing may be used to say that the bounded semilattice is same as idempotent
commutative monoid. Idempotent commutative monoid is defined and an aliasing for
bounded semilattice is given.
\begin{minted}[breaklines,samepage]{Agda}
  record IsIdempotentCommutativeMonoid (∙ : Op₂ A)
  (ε : A) : Set (a ⊔ ℓ) where
field
isCommutativeMonoid : IsCommutativeMonoid ∙ ε
idem                : Idempotent ∙

open IsCommutativeMonoid isCommutativeMonoid public

IsBoundedSemilattice = IsIdempotentCommutativeMonoid
module IsBoundedSemilattice {∙ ε} (L : IsBoundedSemilattice ∙ ε) where

  open IsIdempotentCommutativeMonoid L public
\end{minted}

Note that some mathematicians argue that bounded semilattice and idempotent
commutative monoid are not the same structures but are isomorphic to each other.
We do not consider this argument in the scope of this thesis.
