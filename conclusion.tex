\chapter{Conclusion and Future Work}
The main of this work is to study algebraic structures in proof assistant
systems. To define the scope the work we do a survey on coverage of algebraic on
four proof assistant systems that are Agda, Idris, Coq and Lean. The thesis
shows how to define a structure with some of its constructs and properties in
Agda. We divide this into three main chapters based on closeness of structures
that is quasigroup and loop, semigroup and ring, and Kleene algebra. We then
analyze five problems that arises when defining algebraic structures in proof
systems and give a brief overview of how it can be tackled with product family
algebra. \\

In section ~\ref{contribution} we summarize the contributions of this work and
how it refers to the research outline described in Chapter 1. Section
~\ref{future} discuss some extensions or future work of this work. 

\section{Summary of contributions}
\label{contribution}
Universal algebra is a well studied and evolving branch of mathematics. Proof
systems are useful in automated reasoning and becoming popular in research and
applications more than ever. Chapter 1 provides a overview of quantitative use
of algebraic structures in proof assistant systems. We create a 'clickable'
table that takes to the definition of structures in the standard libraries of
the systems studied (Agda, Idris, Lean and Coq). \\

This leads to define the scope of contribution to Agda standard library. Chapter
5 is dedicated to study the structures quasigroup, loop and their variations.
Chapter 6 provides an overview of semigroup and ring structures with their
properties and morphisms. Chapter 7 is dedicated to study of Kleene algebra and
it's properties in Agda. Along with these structures, we define structures
unital magma, invertible magma, invertible unital magma, idempotent magma,
alternate magma, flexible magma, semimedial magma, medial magma, with their
direct products and morphisms.\\

Our approach of defining these structures led us to analyze the problems such as
ambiguity in naming, equivalent and identical structures. Chapter 8 discuss how
these problems becomes more evident in proof systems that might be ignored in
classical 'pen-and-paper' technique. We give an overview of how product family
algebra can be used to represent and tackle these problems.

\section{Future work}
\label{future}
Our work can be extended in different ways and Agda standard library is evolving
with many contributions. The direct products defined in this thesis do not
clearly differentiate between direct products, products and co-products. There
is currently discussion on Agda standard library to overcome this issue but the
changes are yet to come. Product family algebra is a powerful tool to solve many
problems in ontologies, cryptography and other fields. Only a brief overview of
how this tool can be used is discussed in Chapter 8. A more detailed study with
implementation is required to concretely say to what extent the discussed
problems can be solved. This work will rely on human efforts in building strong
libraries in field of abstract algebra. A more robust and reliable generative
library will be helpful to reduce human efforts. 


