\chapter{Agda}
Agda is a dependently typed programming language. Agda is pure i.e., the
functions have no effect and does not use state and uses lazy evaluation i.e.,
the expressions are not evaluated until they are needed. Therefore, in Agda, the
order of evaluation is hard to predict. Since Agda is lazy, both strict and lazy
evaluation will give the same output \cite{kidney2020finiteness}. Strict
evaluation is to evaluate all arguments to the function before evaluating the
function. Therefore, a function in agda gives output in a finite amount of time
for a valid input. Agda enforces all functions to be total and does not allow it
to be undefined. \\
The process of validating and imposing constraints on values is called type
checking. Agda is based on unified theory of dependent types
\cite{enwiki:1127496533} hence the program written in Agda is in line with the
Martin-Löf Type Theory that the Curry Howard Correspondence hold
\cite{kidney2020finiteness}. Agda can be compiled to haskell or javascript but
it is only type checked, so it can be used as a proof assistant. This chapter
provides a brief overview of programming in Agda in the context of algebraic
structures. 

\section{Types and functions}
Agda provides a logical framework that gives the type \inline{Set} and dependent
functions \inline{(x : A) -> B}. Agda supports inductive types, simple types,
and parameterized types \cite{10.1007/978-3-642-03359-9_6}. These data types are
declared using the keyword \inline{data}.
\label{code:Nat}
\begin{minted}{Agda}
data Nat : Set where
  zero : Nat
  suc  : Nat -> Nat
\end{minted}

In the code snippet \ref{code:Nat}, \inline{Nat} is an inductive defined with
base constant \inline{zero} and an inductive data constructor \inline{suc}.
Inductive data type have list of values, which might have parameters themselves.
Those values are called "constructors", and again they can have parameters
themselves. Here, both \inline{zero} and \inline{suc} are constructors, where
\inline{suc} has a parameter (\inline{Nat}) but not \inline{zero}. In this
example, the smallest element \inline{zero} is closed under the function
\inline{suc}. Since the properties of this function can be proved inductively,
the type is called inductive type. Data types are called inductive types when
they form inductive sets, and inductive sets are partially ordered sets
\cite{enwiki:1142884446} where the inductive hypothesis
holds\cite{enwiki:1127496533}. \\
Another way of defining a type is using the keyword \inline{record}. A record
type can be defined by referencing other types and creating a synonym. A record
type is an inductive data type with single constructor and named components for
the parameters or arguments. An example of record type is discussed later in the
chapter when we define algebraic structure.\\
Since types are values in Agda, there is no real way to distinguish between
them. If \inline{bool} is a simple type\footnote{A simple type represents a
single value} i.e., type\textsubscript{0}, then what is the type of
type\textsubscript{0}?. Agda uses universe polymorphism to resolve this issue.
That is the type of \inline{true} is Bool and its type is type\textsubscript{0}.
The type of type\textsubscript{0} is type\textsubscript{1} and so on.
\cite{kidney2020finiteness}.\\
Similarly, in Agda not every type is a set and the set type can be defined using
keyword Set\textsubscript{1}. A type whose elements are types are called
universe \cite{universeagda}. This primitive (or simple) type is useful to
define and prove theorems about functions that operate on the set.\\

Those familiar with Haskell will find Agda to be somewhat familiar. For example,
functions have a very similar syntax to those in Haskell. A function in Agda is
defined by declaring the type followed by the clauses \cite{agdaFunction}. 
\begin{minted}{Agda}
f : (x₁ : A₁) → ... → (xₙ : Aₙ) → B
f p₁ … pₙ = d
...
f q₁ … qₙ = e
\end{minted} 
Where \inline{f} is the function identified, \inline{p} and \inline{q} are the
patterns of type \inline{A}. \inline{d} and \inline{e} are expressions. There
are other ways to define a function such as using dot patterns, absurd patterns,
as patterns and case trees \cite{agdaFunction}.\\

Functions are operations with a prefix syntax. The infix notations for the
functions/operations can be defined by using underscores in the function to
denote the parameters or arguments. For example, we can define \inline{and} as:
\begin{minted}{Agda}
_and_ : Bool → Bool → Bool
true and x = x
false and _ = false
\end{minted}
Agda standard library defines unary \inline{Op₁} and binary
\inline{Op₂} operations as:
\begin{minted}{Agda}
Op₁ : ∀ {ℓ} → Set ℓ → Set ℓ
Op₁ A = A → A
\end{minted}
\begin{minted}{Agda}
Op₂ : ∀ {ℓ} → Set ℓ → Set ℓ
Op₂ A = A → A → A
\end{minted}

\section{Structure definition}
Algebraic structures are defined as record types in Agda. Records types are used
to group values together, and they provide named fields to generalize dependent
product types. The structures are obtained by wrapping the predicates that are
expressed as "is-a relation~\citep{hu2021formalizing}.
\begin{minted}{Agda}
record IsMagma (∙ : Op₂ A) : Set (a ⊔ ℓ) where
  field
    isEquivalence : IsEquivalence _≈_
    ∙-cong        : Congruent₂ ∙

  open IsEquivalence isEquivalence public
\end{minted}
In the above example structure \inline{IsMagma} is defined as a record type with
fields \inline{isEquivalence} and \inline{∙-cong}. \inline{∙} is a binary
operation on the set \inline{A}. \inline{a ⊔ ℓ} is the least upper bound for the
set. \textunderscore  ≈ \textunderscore is the binary operation argument for
\inline{IsEquivalence}. If a relation P on set A is equivalent to relation Q on
set B, then we say f preserves p for some map f from set A to B.
\inline{Congruent₂ ∙} represents that the binary operation ∙ preserves
equivalence relation. \inline{IsEquivalence} and \inline{Congruent₂} are
predicates defined in standard library.\\
We open the module \inline{isEquivalence} to be able to use it when defining other
structures in the algebra hierarchy. The open statement is made public using the
keyword \inline{public} to be able to re-export the names from another module.

The bundled version of the structures contains the operations of the structures,
sets and axioms. 
\begin{minted}{Agda}
record Magma c ℓ : Set (suc (c ⊔ ℓ)) where
  infixl 7 _∙_
  infix  4 _≈_
  field
    Carrier : Set c
    _≈_     : Rel Carrier ℓ
    _∙_     : Op₂ Carrier
    isMagma : IsMagma _≈_ _∙_

  open IsMagma isMagma public

  rawMagma : RawMagma _ _
  rawMagma = record { _≈_ = _≈_; _∙_ = _∙_ }

  open RawMagma rawMagma public
    using (_≉_)
\end{minted}
Above is the bundled version of \inline{IsMagma} structure. \inline{RawMagma} is
the raw version of the magma with only the operators and set. infix<l,r> denotes
the fixity and precedence of the operator. The operator with higher fixity binds
more strongly than an operator with a lower numeric value. \inline{using}
keyword is used to limit the imported components. When exporting the modules, we
may need to rename the fields to avoid having ambiguity. Keyword
\inline{renaming} is used to rename the fields.
\label{code:rename}
\begin{minted}{Agda}
  open IsMagma *-isMagma public
    using ()
    renaming
    ( ∙-congˡ  to *-congˡ
    ; ∙-congʳ  to *-congʳ
    )
\end{minted} 
In the sample code ~\ref{code:rename}, we rename \inline{∙-congˡ}  to \inline{*-congˡ}
and \inline{∙-congʳ}  to \inline{*-congʳ} thus avoiding conflict with same
elements exported by other module.

\section{Equational Proofs in Agda}
Dutch mathematician L. E. J. Brouwer discovered intuitionism to over come
Russell's paradox and that led to constructive mathematics \cite{enwiki:1122615242}.
In constructive mathematics, knowledge comes with implicit arguments.
Constructive proofs use the existence of a mathematical object is given by
giving a way to create the method. \cite{enwiki:1090644431}. In agda, a function
to and from each type is provided if there is a bijection between two types.
Agda thus allow us to check proofs through compilation and type-checking.
\cite{kidney2020finiteness}.  

Consider the example to the proposition of the associative property x ∙ (y ∙ z)
= (x ∙ y) ∙ z  for a semigroup i.e., a Magma with associative property (x ∙ (y ∙
z) = (x ∙ y) ∙ z). Tje proof can be written in Agda as:
\begin{minted}{Agda}
x∙yz≈xy∙z : ∀ x y z → x ∙ (y ∙ z) ≈ (x ∙ y) ∙ z
x∙yz≈xy∙z x y z = begin 
  x ∙ (y ∙ z) ≈⟨ sym (assoc x y z) ⟩ 
  (x ∙ y) ∙ z ∎
\end{minted}
To make proofs more readable, people have tried to emulate textual proofs, for
example, by creating "begin" and "end" syntax. \inline{begin} indicates the start
of the proof. \inline{begin} is a function that relates two objects.
\begin{minted}{Agda}
begin_ : ∀ {x y} → x IsRelatedTo y → x ∼ y
begin relTo x∼y = x∼y
\end{minted}
\inline{IsRelatedTo} is a type defined to infer arguments even if the underlying equality
evaluates. Standard step to relation is defined as \inline{step-∼}.
\begin{minted}{Agda}
step-∼ : ∀ x {y z} → y IsRelatedTo z → x ∼ y → x IsRelatedTo z
step-∼ _ (relTo y∼z) x∼y = relTo (trans x∼y y∼z)
\end{minted}
similarly, step using equality is given as
\begin{minted}{Agda}
step-≈ = Base.step-∼
syntax step-≈ x y≈z x≈y = x ≈⟨ x≈y ⟩ y≈z
\end{minted}
The termination (i.e., QED) of the proof is given using \inline{_∎} that relates object to itself.
\begin{minted}{Agda}
_∎ : ∀ x → x IsRelatedTo x
x ∎ = relTo refl
\end{minted}
Agda supports quantifiers. Universal quantifier is denoted as \(\forall\) and
existential quantifier is denoted as \(\exists\). In the example universal quantifier is used.

